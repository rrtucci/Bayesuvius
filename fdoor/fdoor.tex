\chapter{Front-door Adjustment Formula}
\label{ch-fdoor}
The front-door (FD) adjustment
formula is proven in
Chapter \ref{ch-do-calc}
from the rules of do-calculus.
The goal 
of this chapter is
to give examples
of the use of that
theorem. We will restate
the theorem in this chapter,
sans proof.
There is no need
to understand the
theorem's
proof in order to use it.
However, you
will
need to skim Chapter \ref{ch-do-calc}
in order to familiarize 
yourself with
the notation used to state the 
theorem.
This chapter also assumes
that you are comfortable 
with the  rules 
for checking for d-separation. Those rules
are covered in Chapter \ref{ch-dsep}.


\fdoordef

\begin{claim} Front-Door Adjustment Formula

\fdoorclaim

\end{claim}
\proof 
See Chapter \ref{ch-do-calc}
\qed

\section{Examples}

\begin{enumerate}
\item
\beq
\xymatrix{
&*++[F-o]{\rvc}\ar[ld]\ar[rd]
\\
\rvx\ar[r]&\rvm\ar[r]&\rvy
}
\eeq
If $\rvx.=\rvx,\rvm.=\rvm$ 
and $\rvy.=\rvy$,
then the FD criterion
is satisfied.
Can't satisfy backdoor
criterion because $\rvz.$
must be observed so
can't block
backdoor path $\rvx-\rvc-\rvy$.

\hrule\item
\beq
\xymatrix{
*++[F-o]{\rvz_1}\ar[d]\ar[dr]
&&
*++[F-o]{\rvz_2}\ar[d]\ar[dl]
\\
*++[F-o]{\rvw_1}\ar[d]
&\rvc\ar[ld]\ar[rd]
&*++[F-o]{\rvw_2}\ar[d]
\\
\rvx\ar[r]&\rvm\ar[r]&\rvy
}
\eeq
If $\rvx.=\rvx,\rvm.=\rvm$ 
and $\rvy.=\rvy$,
then the FD criterion
is satisfied.
Can't satisfy backdoor
criterion because 
to block 
backdoor path $\rvx-\rvc-\rvy$,
need to condition on $\rvc$
(i.e., need $\rvc\in \rvz.$)
but if this is true, 
then long
path 
$\rvx-\rvw_1-\rvz_1-\rvc-\rvz_2-\rvw_2-\rvy$
becomes unblocked.

\end{enumerate}