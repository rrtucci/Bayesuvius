\chapter{Expectation Maximization}

\begin{figure}[h!]
\centering
$$\begin{array}{ccc}
\xymatrix{
\ul{\theta}\ar[d]\ar[dr]
\\
\ul{\vecx}&\ul{\vech}\ar[l]
}
&=&
\xymatrix{
\ul{\theta}
\ar[d]
\ar@/_1pc/[dd]
\ar@/_1pc/[ddd]
\ar[rd]\ar[rdd]\ar[rddd]
\\
\rvx[0]
&\rvh[0]\ar[l]
\\
\rvx[1]
&\rvh[1]\ar[l]
\\
\rvx[2]
&\rvh[2]\ar[l]
}
\end{array}
$$
\caption{bnet for EM with $nsam=3$.}
\label{fig-em-bnet}
\end{figure}

This chapter is based on Wikipedia 
Ref.\cite{wiki-em}.

The bnet for Expectation
Maximization (EM)
is given by Fig.\ref{fig-em-bnet}
for $nsam=3$.
Later on in this chapter,
we will give the node transition prob matrices
for this bnet for
the special
case in which $P(x[i]\cond \theta)$
is a mixture (i.e., weighted sum)
of Gaussians.

Note that if we 
erase the $\rvh[i]$ nodes
from Fig.\ref{fig-em-bnet},
we get the bnet for naive Bayes,
which is used for classification
into the states of $\ul{\theta}$.
However, there is one big
difference. 
With naive Bayes,
the leaf nodes have
different transition prob matrices.
Here, we will assume they are i.i.d.
Naive Bayes is used for classification: i.e., 
given the states 
of the leaf nodes,
we infer the state of the root node.
EM is used for clustering; i.e.,
given many i.i.d. samples,
we fit their distribution by a weighted sum
of clustering distributions.

Let
 
$\call=$likelihood 
function.

$nsam=$ number of samples.

$\vecx=(x[0], x[1], \ldots, x[nsam-1])$,
 $x[i]\in S_\rvx$ for all $i$.

$\vech=(h[0], h[1], \ldots, h[nsam-1])$
= {\bf hidden or missing data}.
$h[i]\in S_\rvh$ for all $i$.

We assume that the samples $(x[i],h[i])$
are i.i.d. for different $i$ at fixed 
$\theta$.
What this means is that 
there are
probability distributions
$P_{\rvx|\rvh,\ul{\theta}}$
and $P_{\rvh|\ul{\theta}}$
such that

\beq
P(\vecx, \vech|\theta)=
\prod_i \left[P_{\rvx|\rvh,\ul{\theta}}
(x[i]\cond h[i], \theta)
P_{\rvh|\ul{\theta}}(h[i]\cond \theta)\right]
\;.
\eeq

Definition of likelihood functions:
\beqa
\underbrace{P(\vecx|\theta)}
_{\call(\theta;\vecx)}
&=&
\sum_{\vech}
\underbrace{P(\vecx,\vech|\theta)}
_{\call(\theta;\vecx,\vech)}
\eeqa


$\theta^*=$ maximum likelihood
estimate of $\theta$ (no prior $P(\theta)$
assumed):

\beq
\theta^*=
\argmax_\theta\call(\theta;\vecx)
\eeq

\hrule\noindent
{\bf The EM algorithm:}
\begin{enumerate}
\item{\bf Expectation step:} 
\beq
Q(\theta|\theta^{(t)})
=
E_{\vech|\vecx,\theta^{(t)}}\ln P(\vecx,\vech|\theta)
\eeq

\item{\bf Maximization step:}

\beq
\theta^{(t+1)}=\argmax_\theta
Q(\theta|\theta^{(t)})
\eeq
\end{enumerate}
Claim: $\theta^{(t)}\rarrow \theta^*$.

\;
\hrule
\section*{Motivation}

\beqa
Q(\theta|\theta)
&=&
E_{\vech|\vecx,\theta}
\ln P(\vecx,\vech|\theta)
\\
&=&
E_{\vech|\vecx,\theta}[
\ln P(\vech|\vecx, \theta) 
+\ln P(\vecx|\theta)]
\\
&=&
-H[P(\ul{\vech}|\vecx, \theta)]+
\ln P(\vecx|\theta)
\eeqa

\beqa
\partial_\theta Q(\theta|\theta)
&=&
-\sum_{\vech}\partial_\theta
P(\ul{\vech}|\vecx, \theta)
+
\partial_\theta
\ln P(\vecx|\theta)
\\
&=&
\partial_\theta
\ln P(\vecx|\theta)
\eeqa

So if $\theta^{(t)}\rarrow \theta$
and $Q(\theta|\theta)$ is max at $\theta=\theta^*$,
then $\ln P(\vecx|\theta)$
is max at $\theta=\theta^*$ too.

For a  more rigorous proof,
see Wikipedia article Ref.\cite{wiki-em}
and references therein.

\section*{EM for Gaussian mixture}

$x[i]\in \RR^d=S_\rvx$. $S_\rvh$ discrete and
not too large. $n_\rvh=|S_\rvh|$ is
number of Gaussians that we are 
going to fit the samples with.

Let
\beq
\theta = [w_h, \mu_h, \Sigma_h]_{h\in S_\rvh}
\;,
\eeq
where the
$[w_h]_{h\in S_\rvh}$ is a probability
distribution of weights, and 
where $\mu_h\in\RR^d$
and $\Sigma_h\in\RR^{d\times d}$
are the mean value vector 
and covariance matrix of
a $d$-dimensional Gaussian distribution.

The transition prob matrices, printed in blue,
for the nodes of Fig.\ref{fig-em-bnet},
for the special case
of a mixture of Gaussians, are as follows:

\beq\color{blue}
P(x[i]\cond h[i],\theta)=
\caln_d(x[i];\mu_{h[i]}, \Sigma_{h[i]})
\eeq

\beq\color{blue}
P(h[i]\cond \theta)=w_{h[i]}
\eeq

Note that

\beqa
P(x[i]\cond \theta)&=&
\sum_h P(x[i]\cond h[i]=h, \theta)
P(h[i]=h\cond\theta)
\\
&=&
\sum_hw_h\caln_d(x[i];\mu_h, \Sigma_h)
\eeqa

\beqa
P(\vecx, \vech|\theta)&=&
\prod_i \left[
w_{h[i]}
\caln_d(x[i];\mu_{h[i]}, \Sigma_{h[i]})
\right]
\\
&=&
\prod_i\prod_h
\left[w_h
\caln_d(x[i];\mu_h, \Sigma_h)\right]
^{\indi(h=h[i])}
\eeqa

{\bf Old Faithful:}
See Wikipedia Ref.\cite{wiki-em}
for an animated
gif of a  classic example
of using EM to fit
samples with a Gaussian mixture.
Unfortunately,
could
not include it
here because pdflatex
does not support animated gifs. 
It shows samples in a 2 dimensional
space
(eruption time, delay time)
from the Old Faithful geyser.
In that example, $d=2$ and $n_\rvh=2$.
Two clusters of points
in a plane are fitted
by 
a mixture of 2 Gaussians.