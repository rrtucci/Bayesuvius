\chapter{Mediation Analysis}
\label{ch-mediation}
This chapter is mostly based on 
Ref.\cite{pearl-2019review}, a 2019
 review of causality by Pearl.

To fully
understand this chapter,
the reader should first read Chapter \ref{ch-counterf}
where do and imagine operators are defined.


\begin{figure}[h!]
$$
\begin{array}{ccc}
\\
\xymatrix{
\rvu_\rvd\ar[dd]
&\rvu_\rvm\ar[d]
&\rvu_\rvy\ar[dd]
\\
&\rvm\ar[rd]
\\
\rvd\ar[ru]\ar[rr]&&\rvy
}
&\;\;\;\;&
\xymatrix{
\rvu_\rvd\ar[dd]\ar@/^2pc/@{<-->}[r]
&\rvu\ar[d]\ar@/^2pc/@{<-->}[r]
&\rvu_\rvy\ar[dd]
\\
&\rvm\ar[rd]
\\
\rvd\ar[ru]\ar[rr]&&\rvy
}
\\
\\
G_0&&G
\end{array}
$$
\caption{DEN bnets $G_0$ and $G$
are used to 
discuss mediation analysis.
In graph
$G_0$,
the external
variables are independent,
whereas in graph $G$
they are not.}
\label{fig-mediation-bnets}
\end{figure}

\begin{figure}[h!]
$$
\begin{array}{c}
\xymatrix{
&\rvh
\ar@{-->}[d]
\ar@{-->}[ldd]
\ar@{-->}[rdd]
\\
&\rvm\ar[rd]
\\
\rvd\ar[ru]\ar[rr]&&\rvy
}
\\
\\
G_{gen}
\end{array}
$$
\caption{General bnet used to 
discuss mediation analysis.
Node $\rvh$ is a hidden confounder.}
\label{fig-gen-bnet-mediation}
\end{figure}

The term ``mediation analysis" (MA)
refers
to  the analysis by Pearl
of the DEN bnet
$G$
in Fig.\ref{fig-mediation-bnets}.
We will discuss MA in terms
of DEN bnets, just as Pearl does.
However, note that much of 
what we will say applies also to 
the general (i.e., not
necessarily a DEN) bnet $G_{gen}$
show in Fig. \ref{fig-gen-bnet-mediation}.



In the DEN bnet $G$,
node $\rvd$ 
influences node
$\rvy$
both
directly
and via the mediator node $\rvm$.
The structural 
equations for $G$
are of the form:

\begin{subequations}
\label{eq-struc-eqs-med}
\beqa
\rvd&=&\rvu_\rvd
\\
\rvm&=&f_\rvm(\rvd, \rvu_\rvm)
\\
\rvy&=&f_\rvy(\rvd, \rvm, \rvu_\rvy)
\;.
\eeqa
\end{subequations}
Thus,

\beq
\rvy=f_\rvy(u_\rvd, 
f_\rvm(\rvu_\rvd, \rvu_\rvm), \rvu_\rvy)
\;.
\eeq

\begin{figure}[h!]
$$
\begin{array}{cccc}
\\
\xymatrix{
\rvu_\rvd\ar@/^2pc/@{<-->}[r]
&\rvu_\rvm\ar[d]\ar@/^2pc/@{<-->}[r]
&\rvu_\rvy\ar[dd]
\\
&\rvm\ar[rd]
\\
\rvd=5\ar[ru]
\ar[rr]&&\rvy
}
&\;\;\;\;\;&
\xymatrix{
\rvu_\rvd\ar[dd]\ar@/^2pc/@{<-->}[rr]
&&\rvu_\rvm\ar[d]\ar@/^2pc/@{<-->}[r]
&\rvu_\rvy\ar[dd]
\\
&^{\cali_\rvm{\rvd}=5}\ar[r]
&\rvm\ar[rd]
\\
\rvd
\ar[rrr]
&&&\rvy
}
\\
\\
\cald_{\rvd=5}G
&&
\cali_{\rvd\rarrow\rvm}(5)G
\end{array}
$$
\caption{Graph $G$
of Fig.\ref{fig-mediation-bnets}
after applying do operator $\cald_{\rvd=5}$
and imagine operator 
$\cali_{\rvd\rarrow\rvm}(5)$.}
\label{fig-mediation-ops-egs}
\end{figure}

If we apply
$\cald_{\rvd=5}G$
to Eqs.(\ref{eq-struc-eqs-med}), we get

\begin{subequations}
\label{eq-mediation-rho-egs}
\beqa
\rvd&=&5
\\
\rvm&=&f_\rvm(\rvd, \rvu_\rvm)
\\
\rvy&=&f_\rvy(\rvd, \rvm, \rvu_\rvy)
\;.
\eeqa
\end{subequations}
Eqs.\ref{eq-mediation-rho-egs}
are represented graphically
in Fig.\ref{fig-mediation-ops-egs}.
We will often denote the  random variable
 $\rvy$ in Eqs.(\ref{eq-mediation-rho-egs})
by the more explicit symbol 
$\rvy_{\cald_{\rvd=5}G}$.
Pearl often 
refers to
this $\rvy$ by the less explicit symbol
$Y_5$ or $Y_5(u)$ where 
$u=(u_\rvm, u_\rvy)=u_{!\rvd}$.

If we apply
$\cali_{\rvd\rarrow\rvm}(5)G$
to Eqs.(\ref{eq-struc-eqs-med}), we get

\begin{subequations}
\label{eq-mediation-kappa-egs}
\beqa
\rvd&=&\rvu_\rvd
\\
\rvm&=&f_\rvm(5, \rvu_\rvm)
\\
\rvy&=&f_\rvy(\rvd, \rvm, \rvu_\rvy)
\;.
\eeqa
\end{subequations}
Eqs.\ref{eq-mediation-kappa-egs}
are represented graphically
in Fig.\ref{fig-mediation-ops-egs}.
We will often denote the  random variable
 $\rvy$ in Eqs.(\ref{eq-mediation-kappa-egs})
by the more explicit symbol 
$\rvy_{\cali_{\rvd\rarrow\rvm}(5)G}$.
 Pearl often 
refers to
this $\rvy$ by the less explicit symbol
$Y_5$ or $Y_5(u)$ where
$u=(u_\rvd, u_\rvm, u_\rvy)$.

\hrule

\begin{figure}[h!]
$$
\begin{array}{cccc}
\\
\xymatrix{
\rvu_\rvd\ar@/^2pc/@{<-->}[r]
&\rvu_\rvm\ar[d]\ar@/^2pc/@{<-->}[r]
&\rvu_\rvy\ar[dd]
\\
&\rvm\ar[rd]
\\
\rvd=d\ar[ru]
\ar[rr]&&\rvy
}
&\;\;\;\;
&
\xymatrix{
\rvu_\rvd\ar@/^2pc/@{<-->}[r]
&\rvu_\rvm\ar@/^2pc/@{<-->}[r]
&\rvu_\rvy\ar[dd]
\\
&\rvm=m\ar[rd]
\\
\rvd=d
\ar[rr]&&\rvy
}
\\
\\
\cald_{\rvd=d}G
&&
\cald_{\rvd=d}\cald_{\rvm=m}G
\end{array}
$$
\caption{Graph $G$
of Fig.\ref{fig-mediation-bnets}
after applying the 
do operators $\cald_{\rvd=d}$
and
$\cald_{\rvd=d}\cald_{\rvm=m}$.}
\label{fig-mediation-rho}
\end{figure}
Define the Total Effect (TE),
and the
Controlled Direct Effect (CDE) by
\beqa
TE&=& E[
\rvy_{\cald_{\rvd=1}G}
-\rvy_{\cald_{\rvd=0}G}
]
\\
CDE(m)&=&
E[
\rvy_{\cald_{\rvd=1}\cald_{\rvm=m}G}
-\rvy_{\cald_{\rvd=0}\cald_{ \rvm=m}G}
]
\eeqa
The two DEN diagrams
$\cald_{\rvd=d}G$
and
$\cald_{\rvd=d}\cald_{\rvm=m}G$
used in the definitions
of $TE$ and $CDE$
are given in Fig.\ref{fig-mediation-rho}.
\hrule

\begin{figure}[h!]
$$
\begin{array}{c}
\\
\xymatrix{
\rvu_\rvd\ar@/^2pc/@{<-->}[rr]
&&\rvu_\rvm\ar[d]\ar@/^2pc/@{<-->}[r]
&\rvu_\rvy\ar[dd]
\\
&\cali_\rvm\rvd=d'\ar[r]
&\rvm\ar[rd]
\\
\cald \rvd=d\ar[rrr]
&
&&\rvy
}
\\
\\
\cald_{\rvd=d}
\cali_{\rvd\rarrow\rvm}(d')
G
\end{array}
$$
\caption{
Graph $G$
of Fig.\ref{fig-mediation-bnets}
after
applying the 
imagine operator
 $\cali$
 to arrow
$\rvd\rarrow\rvm$ and 
the do operator to node $\rvd$.}
\label{fig-mediation-kappa}
\end{figure}

Let

\beq
\caly_d^{d'}=
 E[
\rvy_{\cald_{\rvd=d}
\cali_{\rvd\rarrow\rvm}(d')
G}
]
\eeq
Fig.\ref{fig-mediation-kappa}
shows the diagram 
$
\cald_{\rvd=d}\cali_{\rvd\rarrow\rvy}(d')G$
used in
the definition of $\caly_d^{d'}$.


Now define the
Natural Direct Effect (NDE), and the
Natural Indirect Effect (NIE)
by
\beqa
NDE
&=&\caly_1^0 - \caly_0^0
\\
NIE(d)
&=&\caly_d^1 - \caly_d^0
\;.
\eeqa

Note that
\beqa
NDE+NIE(1)&=&(\caly_1^0-\caly_0^0)+(\caly_1^1 - \caly_1^0)
\\
&=&\caly_1^1-\caly_0^0
\\
&=&
TE
\;.
\eeqa

$TE$ and the ``controlled effect"
$CDE(m)$
contain do operators
but no imagine 
operators 
so one can do 
intervention 
experiments to
calculate them.
On the other hand,
the ``natural effects" $NDE$ and $NIE(d)$
contain
imagine 
operators
(and therefore
counterfactual
distributions)
so it's not
obvious how to 
calculate them,
or even whether it
is possible to do so.
The next claim
shows how to calculate
$NDE$ in certain
cases. In technical jargon,
the claim 
gives sufficient conditions
for $\cald\cali$-identifiability
of $NDE$. 
If $NDE$ is $\cald\cali$-identifiable,
then $NIE(1)$ is too,
because they add to
$TE$, 
which is identifiable.


\begin{figure}[h!]
$$
\begin{array}{c}
\\
\xymatrix{
\rvu_\rvd\ar@/^2pc/@{<-->}[rr]
&&\rvu_\rvm\ar@/^2pc/@{<-->}[r]
&\rvu_\rvy\ar[dd]
\\
\cald\rvd'=0\ar@/_1.5pc/[rr]
&\ul{\xi}.\ar[r]
&\cald\rvm\ar[rd]
\\
\cald \rvd=d\ar[rrr]
&
&&\rvy
}
\end{array}
$$
\caption{
Bnet alluded to in Claim \ref{cl-adjust-nde}
}
\label{fig-adjust-nde}
\end{figure}


\begin{claim}(Adjustment formula for NDE)
\label{cl-adjust-nde}

Suppose $\xi.$
is a multinode of the bnet.
If 
\begin{enumerate}
\item
 $\ul{\xi}.\cap de(\rvd)=\emptyset$
(i.e., $\ul{\xi}.$ contains no descendants 
of $\rvd$)
\item conditioning on $\ul{\xi}.$ blocks
all backdoor paths from $\rvd$ to $\rvy$,
\end{enumerate}
then  (see Fig.\ref{fig-adjust-nde})
\beq
NDE = \sum_m\sum_{\xi.}[
\caly_1^m(\xi.)-\caly_0^m(\xi.)]
P(m|\cald\rvd=0, \xi.)P(\xi.)
\eeq
where

\beq
\caly_d^m(\xi.)=
E_{|\xi.}[\rvy_{\cald_{\rvd=d},\cald_{\rvm=m}G}]
\;.
\eeq 
If besides 1 and 2, the following are true:
\begin{enumerate}
\item[3.]
$P(m|\cald\rvd=0, \xi.)$
is $\cald$-identifiable (i.e., expressible without do's)
\item[4.]
$\caly_d^m(\xi.)$ is $\cald$-identifiable
\end{enumerate}
then
\beq
NDE = \sum_m\sum_{\xi.}[
\caly_1^m(\xi.)-\caly_0^m(\xi.)]
P(m|\rvd=0, \xi.)P(\xi.)
\eeq
where 
\beq
\caly_d^m(\xi.)=
E_{|d, m, \ul{\xi}.}[\rvy]
\eeq
If besides 1,2,3,4, the following is true

\begin{enumerate}
\item[5.] there is no confounding 
connecting the external nodes $\rvu_\rvd, \rvu_\rvy, \rvu_\rvm$,
\end{enumerate}
then
\beq
NDE = \sum_m[
\caly_1^m-\caly_0^m]
P(m|\rvd=0)
\eeq
where 
\beq
\caly_d^m=
E_{|d, m}[\rvy]
\eeq
\end{claim}
\proof 
See references in Ref.\cite{pearl-2019review}.
\qed

