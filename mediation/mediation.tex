\chapter{Mediation Analysis}
\label{ch-mediation}
This chapter is mostly based on 
Ref.\cite{pearl-2019review}, a 2019
 review of causality by Pearl.

To fully
understand this chapter,
the reader should first read Chapter \ref{ch-counterf}
where do and imagine operators are defined.


\begin{figure}[h!]
$$
\begin{array}{ccc}
\\
\xymatrix{
\rvu_\rvt\ar[dd]
&\rvu_\rvm\ar[d]
&\rvu_\rvy\ar[dd]
\\
&\rvm\ar[rd]
\\
\rvt\ar[ru]\ar[rr]&&\rvy
}
&\;\;\;\;&
\xymatrix{
\rvu_\rvt\ar[dd]\ar@/^2pc/@{<-->}[r]
&\rvu\ar[d]\ar@/^2pc/@{<-->}[r]
&\rvu_\rvy\ar[dd]
\\
&\rvm\ar[rd]
\\
\rvt\ar[ru]\ar[rr]&&\rvy
}
\\
\\
G_0&&G
\end{array}
$$
\caption{DEN bnets $G_0$ and $G$
are used to 
discuss mediation analysis.
In graph
$G_0$,
the external
variables are independent,
whereas in graph $G$
they are not.}
\label{fig-mediation-bnets}
\end{figure}

\begin{figure}[h!]
$$
\begin{array}{c}
\xymatrix{
&\rvh
\ar@{-->}[d]
\ar@{-->}[ldd]
\ar@{-->}[rdd]
\\
&\rvm\ar[rd]
\\
\rvt\ar[ru]\ar[rr]&&\rvy
}
\\
\\
G_{gen}
\end{array}
$$
\caption{General bnet used to 
discuss mediation analysis.
Node $\rvh$ is a hidden confounder.}
\label{fig-gen-bnet-mediation}
\end{figure}

The term ``mediation analysis" (MA)
refers
to  the analysis by Pearl
of the DEN bnet
$G$
in Fig.\ref{fig-mediation-bnets}.
We will discuss MA in terms
of DEN bnets, just as Pearl does.
However, note that much of 
what we will say applies also to 
the general (i.e., not
necessarily a DEN) bnet $G_{gen}$
show in Fig. \ref{fig-gen-bnet-mediation}.



In the DEN bnet $G$,
node $\rvt$ 
influences node
$\rvy$
both
directly
and via the mediator node $\rvm$.
The structural 
equations for $G$
are of the form:

\begin{subequations}
\label{eq-struc-eqs-med}
\beqa
\rvt&=&\rvu_\rvt
\\
\rvm&=&f_\rvm(\rvt, \rvu_\rvm)
\\
\rvy&=&f_\rvy(\rvt, \rvm, \rvu_\rvy)
\;.
\eeqa
\end{subequations}
Thus,

\beq
\rvy=f_\rvy(u_\rvt, 
f_\rvm(\rvu_\rvt, \rvu_\rvm), \rvu_\rvy)
\;.
\eeq

\begin{figure}[h!]
$$
\begin{array}{cccc}
\\
\xymatrix{
\rvu_\rvt\ar@/^2pc/@{<-->}[r]
&\rvu_\rvm\ar[d]\ar@/^2pc/@{<-->}[r]
&\rvu_\rvy\ar[dd]
\\
&\rvm\ar[rd]
\\
\rvt=5\ar[ru]
\ar[rr]&&\rvy
}
&\;\;\;\;\;&
\xymatrix{
\rvu_\rvt\ar[dd]\ar@/^2pc/@{<-->}[rr]
&&\rvu_\rvm\ar[d]\ar@/^2pc/@{<-->}[r]
&\rvu_\rvy\ar[dd]
\\
&^{\cali_\rvm{\rvt}=5}\ar[r]
&\rvm\ar[rd]
\\
\rvt
\ar[rrr]
&&&\rvy
}
\\
\\
\cald_{\rvt=5}G
&&
\cali_{\rvt\rarrow\rvm}(5)G
\end{array}
$$
\caption{Graph $G$
of Fig.\ref{fig-mediation-bnets}
after applying do operator $\cald_{\rvt=5}$
and imagine operator 
$\cali_{\rvt\rarrow\rvm}(5)$.}
\label{fig-mediation-ops-egs}
\end{figure}

If we apply
$\cald_{\rvt=5}G$
to Eqs.(\ref{eq-struc-eqs-med}), we get

\begin{subequations}
\label{eq-mediation-rho-egs}
\beqa
\rvt&=&5
\\
\rvm&=&f_\rvm(\rvt, \rvu_\rvm)
\\
\rvy&=&f_\rvy(\rvt, \rvm, \rvu_\rvy)
\;.
\eeqa
\end{subequations}
Eqs.\ref{eq-mediation-rho-egs}
are represented graphically
in Fig.\ref{fig-mediation-ops-egs}.
We will often denote the  random variable
 $\rvy$ in Eqs.(\ref{eq-mediation-rho-egs})
by the more explicit symbol 
$\rvy_{\cald_{\rvt=5}G}$.
Pearl often 
refers to
this $\rvy$ by the less explicit symbol
$Y_5$ or $Y_5(u)$ where 
$u=(u_\rvm, u_\rvy)=u_{!\rvt}$.

If we apply
$\cali_{\rvt\rarrow\rvm}(5)G$
to Eqs.(\ref{eq-struc-eqs-med}), we get

\begin{subequations}
\label{eq-mediation-kappa-egs}
\beqa
\rvt&=&\rvu_\rvt
\\
\rvm&=&f_\rvm(5, \rvu_\rvm)
\\
\rvy&=&f_\rvy(\rvt, \rvm, \rvu_\rvy)
\;.
\eeqa
\end{subequations}
Eqs.\ref{eq-mediation-kappa-egs}
are represented graphically
in Fig.\ref{fig-mediation-ops-egs}.
We will often denote the  random variable
 $\rvy$ in Eqs.(\ref{eq-mediation-kappa-egs})
by the more explicit symbol 
$\rvy_{\cali_{\rvt\rarrow\rvm}(5)G}$.
 Pearl often 
refers to
this $\rvy$ by the less explicit symbol
$Y_5$ or $Y_5(u)$ where
$u=(u_\rvt, u_\rvm, u_\rvy)$.

\hrule

\begin{figure}[h!]
$$
\begin{array}{cccc}
\\
\xymatrix{
\rvu_\rvt\ar@/^2pc/@{<-->}[r]
&\rvu_\rvm\ar[d]\ar@/^2pc/@{<-->}[r]
&\rvu_\rvy\ar[dd]
\\
&\rvm\ar[rd]
\\
\rvt=t\ar[ru]
\ar[rr]&&\rvy
}
&\;\;\;\;
&
\xymatrix{
\rvu_\rvt\ar@/^2pc/@{<-->}[r]
&\rvu_\rvm\ar@/^2pc/@{<-->}[r]
&\rvu_\rvy\ar[dd]
\\
&\rvm=m\ar[rd]
\\
\rvt=t
\ar[rr]&&\rvy
}
\\
\\
\cald_{\rvt=t}G
&&
\cald_{\rvt=t}\cald_{\rvm=m}G
\end{array}
$$
\caption{Graph $G$
of Fig.\ref{fig-mediation-bnets}
after applying the 
do operators $\cald_{\rvt=t}$
and
$\cald_{\rvt=t}\cald_{\rvm=m}$.}
\label{fig-mediation-rho}
\end{figure}
Define the Total Effect (TE),
and the
Controlled Direct Effect (CDE) by
\beqa
TE&=& E[
\rvy_{\cald_{\rvt=1}G}
-\rvy_{\cald_{\rvt=0}G}
]
\\
CDE(m)&=&
E[
\rvy_{\cald_{\rvt=1}\cald_{\rvm=m}G}
-\rvy_{\cald_{\rvt=0}\cald_{ \rvm=m}G}
]
\eeqa
The two DEN diagrams
$\cald_{\rvt=t}G$
and
$\cald_{\rvt=t}\cald_{\rvm=m}G$
used in the definitions
of $TE$ and $CDE$
are given in Fig.\ref{fig-mediation-rho}.
\hrule

\begin{figure}[h!]
$$
\begin{array}{c}
\\
\xymatrix{
\rvu_\rvt\ar[dd]\ar@/^2pc/@{<-->}[rr]
&&\rvu_\rvm\ar[d]\ar@/^2pc/@{<-->}[r]
&\rvu_\rvy\ar[dd]
\\
&\cali_\rvm\rvt=b\ar[r]
&\rvm\ar[rd]
\\
\rvt
&\cali_\rvy\rvt=a\ar[rr]
&&\rvy
}
\\
\\
\cali_{\rvt\rarrow\rvy}(a)
\cali_{\rvt\rarrow\rvm}(b)G
\end{array}
$$
\caption{
Graph $G$
of Fig.\ref{fig-mediation-bnets}
after
applying the 
imagine operator
 $\cali$
 to arrows
$\rvt\rarrow\rvm$ and $\rvt\rarrow\rvy$.}
\label{fig-mediation-kappa}
\end{figure}

Let

\beq
\caly^b_a=
 E[
\rvy_{\cali_{\rvt\rarrow\rvy}(a)
\cali_{\rvt\rarrow\rvm}(b)G}
]
\eeq
Fig.\ref{fig-mediation-kappa}
shows the diagram 
$\cali_{\rvt\rarrow\rvy}(a)
\cali_{\rvt\rarrow\rvm}(b)G$
used in
the definition of $\caly^b_a$.


Now define the
Natural Direct Effect (NDE), and the
Natural Indirect Effect (NIE)
by
\beqa
NDE
&=&\caly_1^0 - \caly_0^0
\\
NIE(t)
&=&\caly_t^1 - \caly_t^0
\;.
\eeqa

Note that
\beqa
NDE+NIE(1)&=&(\caly_1^0-\caly_0^0)+(\caly_1^1 - \caly_1^0)
\\
&=&\caly_1^1-\caly_0^0
\\
&=&
TE
\;.
\eeqa


