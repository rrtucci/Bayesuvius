\chapter{Lorenz System: COMING SOON}
\label{ch-lorenz}

The  {\bf Lorenz system} arises in fluid mechanics and laser physics, and other fields.
It is given by


\begin{figure}[h!]
$$
\xymatrix@R=3pc@C=3pc{
\rvy\ar[d]|\redminus
\ar[r]
\ar@[green]@/^.8pc/[drr]|\redplus
&\bigotimes\ar[drrr]|\redplus
&\rvx\ar[l]
\ar[r]
\ar[d]|\redminus
\ar@[green]@/_.8pc/[dll]|\redplus
&\bigotimes\ar[dlll]|\redminus
&\rvz\ar[d]|\redminus
\ar[l]
\\
\dot{\rvy}
&&\dot{\rvx}
&&\dot{\rvz}
}$$
\caption{Lorentz system. \OTO\cite{OTO}}
\label{fig-lorenz-sys}
\end{figure}

\beq
\left\{
\begin{array}{l}
\dot{x} = \sigma (y - x)
\\
\dot{y} = x(\rho - z) - y
\\ 
\dot{z} = xy - \beta z
\end{array}
\right.
\eeq
where $\sigma > 0$, $\rho > 0$, and $\beta > 0$ are parameters.

%
%For certain parameter values (e.g., $\sigma = 10, \beta = 8/3, \rho \in [24, 30]$), the Lorenz system exhibits bistability and hysteresis. In this regime, the system has two coexisting attractors (such as chaotic attractors or fixed points) and can jump between them depending on initial conditions and parameter sweeps. This transition can exhibit hysteresis as $\rho$ is increased or decreased.
%
%
