\chapter{Observational
 Equivalence of DAGs}\label{ch-obs-equi}

This chapter is based on Chapter 1 of
Ref.\cite{pearl-2013book}
and on a blog post by 
Bruno Gon\c{c}alves 
(Ref.\cite{bruno-obs-equiv}).



Two DAGs are {\bf observationally
equivalent (OE)}
if they
represent
the same
probability distribution.
For example,
$\rva\rarrow\rvb$
and $\rva\larrow\rvb$
are OE because

\beq
P(a|b)P(b)=P(a,b)=P(b|a)P(a)
\label{eq-two-node-prob}
\;.
\eeq

The {\bf
skeleton}
of a DAG is its
undelying undirected graph.

A {\bf v-structure}
in
a DAG consists of
two arrows 
converging to
a node and
such 
that their tails
are not 
connected 
by a third arrow.
Fig.\ref{fig-v-strucs}
shows in red all the v-structures 
of a particular DAG.

\begin{figure}[h!]
$$\xymatrix{
\rvz_1\ar[dd]\ar[rd]
&&\rvz_2\ar@[red][dd]\ar[ld]
\\
&\rvz_3\ar[dl]\ar[dr]
\\
\rvx\ar[r]&\rvw\ar@[red][r]&\rvy
\\
&(a)
}
\xymatrix{
\rvz_1\ar[dd]\ar@[red][rd]
&&\rvz_2\ar[dd]\ar@[red][ld]
\\
&\rvz_3\ar[dl]\ar[dr]
\\
\rvx\ar[r]&\rvw\ar[r]&\rvy
\\
&(b)
}
\xymatrix{
\rvz_1\ar[dd]\ar[rd]
&&\rvz_2\ar[dd]\ar[ld]
\\
&\rvz_3\ar[dl]\ar@[red][dr]
\\
\rvx\ar[r]&\rvw\ar@[red][r]&\rvy
\\
&(c)
}$$
\caption{Example showing in red
all v-structures 
of a particular DAG.}
\label{fig-v-strucs}
\end{figure}





\begin{claim}Observational Equivalence
Theorem (by Verma and Pearl, 1990)

Two DAGs are
OE
iff they
have the same
skeletons and
the same v-structures.
\end{claim}

\hrule\noindent {\bf Example}

\begin{figure}[h!]
$$\xymatrix{
&\rvx_1
\ar[dl]
\ar[dr]
\\
\rvx_2\ar[dr]
&&\rvx_3\ar[dl]
\\
&\rvx_4\ar[d]
\\
&\rvx_5
\\
&(a)
}
\;\;\;\;\;
\xymatrix{
&\rvx_1
\ar[dr]
\\
\rvx_2\ar[dr]\ar[ur]
&&\rvx_3\ar[dl]
\\
&\rvx_4\ar[d]
\\
&\rvx_5
\\
&(b)
}
\;\;\;\;\;
\xymatrix{
&\rvx_1
\ar[dl]
\\
\rvx_2\ar[dr]
&&\rvx_3\ar[dl]\ar[ul]
\\
&\rvx_4\ar[d]
\\
&\rvx_5
\\
&(c)
}$$
\caption{These 3 DAGs are 
observational equivalent (OE).}
\label{fig-obs-equi-eg}
\end{figure}
\begin{figure}[h!]
$$
\xymatrix{
&\rvx_1
\ar@{-}[dl]\ar@{-}[dr]
\\
\rvx_2\ar[dr]
&&\rvx_3\ar[dl]
\\
&\rvx_4\ar[d]
\\
&\rvx_5
\\
&(c)
}$$
\caption{This partially
directed graph 
represents the 3 
DAGs
in Fig.\ref{fig-obs-equi-eg}.}
\label{fig-pdag1}
\end{figure}

The 3 DAGs
in Fig.\ref{fig-obs-equi-eg}
are OE. They form an equivalence
class of OE DAGs
that represent
the same probability distribution.
This
equivalence
class of DAGs
can be represented
by the partially 
directed graph 
Fig.\ref{fig-pdag1}.
These 3 DAGs 
can be proven to 
be OE
in the following 3 ways:


\begin{enumerate}
\item
Write
the generic
probability
distributions
represented by
the 3 DAGs,
and show
that they are equal, as we did 
in Eq.\ref{eq-two-node-prob}.
That is the low brow way
of proving OE.
\item
Use d-separation
(see Chapter \ref{chap-dsep}).
Consider DAG $(a)$ first.
Rename the nodes as $\ul{\tau}_j$
with $j=1, 2, \ldots$
so that the names
are in topological
order (i.e., so that 
the parents of $\ul{\tau}_j$
have
indices that 
are smaller than $j$).
The node names $\rvx_j$
of DAG $(a)$
are already in topologigal
order, so we skip this step
for DAG $(a)$.
Now write down 
its total probability
distribution
and notice
which parents
of a fully connected DAG
were omitted.

\beq
P(x_1, x_2, x_3, x_4, x_5)
=
\underbrace{P(x_5|x_4)}
_{x_3, x_2, x_1 \text{ omitted}}
\underbrace{P(x_4|x_3, x_2)}
_{x_1\text{ omitted}}
\underbrace{P(x_3|x_1)}
_{x_2\text{ omitted}}
P(x_2|x_1)P(x_1)
\eeq
The
observations
of which parents
 were omitted
can be stated in d-separation lingo as
the following 3 orthogonality
relations:\footnote{
Normally,
if we had changed
from the 
original
node names to
the  
$\ul{\tau}_j$ node
names, 
these
orthogonality
relations
would first
be stated 
in terms
of the 
$\ul{\tau}_j$
names, 
and we 
could 
translate
them
so
that
they
were
stated 
in terms
of the 
original
node 
names.
But
for DAG
$(a)$
there was no need
to
use 
the $\ul{\tau}_j$ names.
}

\begin{subequations}
\label{eq-oe-ortho}
\beqa
\rvx_3 \perp_P \rvx_2 &|& \rvx_1
\\
\rvx_4 \perp_P \rvx_1 &|& \rvx_2, \rvx_3
\\
\rvx_5 \perp_P (\rvx_1, \rvx_2, \rvx_3) &|& \rvx_4
\;.
\eeqa
\end{subequations}



Going through the
same procedure
for the other 2 DAGs yields,
for each of them,
an equivalent set of 3 
orthogonality equations.\footnote{
The
$\rvx_j$
node
names
are 
no longer
in topological
order
for DAGs $(b)$
and $(c)$
so
for them
you
should
go through
the
intermediate step
of renaming the
nodes $\ul{\tau}_j$,
and 
then,
after
obtaining
the orthogonality
relations
in terms
of the
$\ul{\tau}_j$ names, 
translating
them
back to 
the original
$\rvx_j$ names.}

This is enough to
conclude that the 3 DAGs 
of Fig.\ref{fig-obs-equi-eg}
are OE.

Note that Eqs.(\ref{eq-oe-ortho})
encompass all that
there is to say about the
observability
of DAG $(a)$. These
3 equations
can be checked empirically
to assess how well the DAG fits
the data.
For example,
one can do
OLS (ordinary least squares) regression
$x_5\sim x_1+ x_2 + x_3 + x_4$ 
on the data, i.e., try 
to fit $x_5=
\beta_0 +
\sum_{i=1}^{4}\beta_i x_i$
to the data, and find that,
to a good approximation, 
$\beta_1=\beta_2=\beta_3=0$.




\item
Use the OE Theorem. All three
DAGs have the same
skeleton,
and the same
single
v-structure
$\rvx_2\rarrow\rvx_4\larrow\rvx_3$.




\end{enumerate}

