\chapter*{Notational Conventions and Preliminaries}
\addcontentsline{toc}{chapter}{Notational Conventions and Preliminaries}

\label{ch-not-cons}
\section{Some abbreviations frequently
used throughout this book.}

\begin{itemize}
\item
bnet= B net= Bayesian Network
\item
CPT = Conditional Probabilities Table,
 same as TPM
\item
DAG = Directed Acyclic Graph
\item
i.i.d.= independent identically 
distributed.
\item
TPM= Transition Probability Matrix,
same as CPT

\end{itemize}

\section{${\cal N}(!a)$}
$\caln(!a)$ will denote 
a normalization constant that does not depend
on $a$. For example, $P(x)=\caln(!x)e^{-x}$
where $\int_0^\infty dx \;P(x)=1$.

\section{One hot}
A {\bf one hot } vector of zeros and 
ones is a vector with all entries 
zero with
the exception of a single entry which is one.
A {\bf one cold} vector has all entries
equal to one with the exception of  a
single entry which is zero.
For example, if $x^n=(x_0, x_1, \ldots,
x_{n-1})$ and
$x_i=\delta(i,0)$ then $x^n$ is one hot.


\section{Special sets}
Define $\ZZ, \RR, \CC$ to be
 the integers, real numbers
 and complex numbers, respectively. 

For $a<b$, define $I_\ZZ$ 
to be the integers in the 
interval $I$, where 
$I=[a,b],[a,b),(a,b],(a,b)$ 
(i.e, $I$ can be closed or
 open on either side).

$A_{>0}=\{k\in A: k>0\}$ for $A=\ZZ, \RR$.

\section{Kronecker 
delta function}

 For $x,y$ in discrete set $S$, 
\beq
\delta(x,y)=\left\{
\begin{array}{l}
1\;{\rm if}\; x=y
\\
0 \;{\rm if}\; x\neq y
\end{array}
\right.
\eeq

\section{Dirac delta function}
 For $x,y\in\RR$,
\beq
\int^{+\infty}_{-\infty}dx\;\delta(x-y)f(x)=f(y)
\eeq

\section{Indicator function 
(aka Truth function)}
\beq
\indi(\cals)=\left\{
\begin{array}{l}
1\;{\rm if\; \cals\; is\; true} 
\\
0 \;{\rm if \;\cals\; is \;false}
\end{array}
\right.
\eeq
For example, $\delta(x,y)=\indi(x=y)$.

\section{Underlined letters
 to indicate random variables}
Random variables will be indicated by 
underlined letters and their values 
by non-underlined letters.
 Each node of a bnet will be
 labelled by a random variable.
 Thus, $\rvx=x$ means that node 
$\rvx$ is in state $x$.

It is more
conventional to
use an upper
case letter to 
indicate
a random 
variable
and a lower case letter
for its state.
Thus, $X=x$ means that 
random variable
$X$ is in state $x$.
However,
we have
opted
in this
book to
avoid
that notation,
because
we often
want to define
certain lower
case letters 
to be random variables
or, conversely, define certain upper
case letters to 
be non-random variables.

\section{Probability distributions}
 $P_\rvx(x)=P(\rvx=x)=P(x)$ is the probability that random variable $\rvx$ equals $x\in S_\rvx$. $S_\rvx$ is the set of states (i.e., values) that $\rvx$ can assume and $n_\rvx = |S_\rvx|$ is the size (aka cardinality) of that set. Hence, 
\beq
\sum_{x\in S_\rvx}P_\rvx(x)=1
\eeq

\hrule
\beq
P_{\rvx,\rvy}(x,y)=P(\rvx=x, \rvy=y)=P(x,y)
\eeq
\beq
P_{\rvx|\rvy}(x|y)=P(\rvx=x| \rvy=y)=P(x|y)=\frac{P(x,y)}{P(y)}
\eeq



\section{Discretization
of continuous
probability distributions}

The TPM of a node 
of a bnet can be either a discrete or 
a continuous probability distribution. 
To go from continuous to discrete, one 
replaces integrals over states of a node
 by sums over new states, and Dirac delta 
functions by Kronecker delta functions.
 More precisely, consider a function 
$f: [a, b]\rarrow \RR$. Express
 $[a,b]$ as 
a union of
small, disjoint (except for
one point) closed sub-intervals (bins) of
length $\Delta x$.
Name one point
in each bin to be the representative of that bin,
and  let $S_\rvx$ be the
set of all the bin representatives. This is called 
discretization or binning. Then

\beq 
\frac{1}{(b-a)}
\int_{[a,b]} dx \; f(x)\rarrow
\frac{\Delta x}{(b-a)} \sum_{x\in S_\rvx}f(x)
=
\frac{1}{n_\rvx} \sum_{x\in S_\rvx}f(x)
 \;.
\eeq
Both sides of last equation are 1 when $f(x)=1$.
 Furthermore, if $y\in S_\rvx$, then

\beq 
\int_{[a,b]} dx \; \delta(x-y)f(x)=f(y)
\rarrow \sum_{x\in S_\rvx}\delta(x,y)f(x)
=f(y)
\;.
\eeq


\section{Samples, 
i.i.d. variables}
\beq
\vec{x}= (x[0], x[1], x[2] \ldots,
 x[nsam(\vecx)-1])=x[:]
\eeq

 $nsam(\vecx)$ is the number of samples 
 of $\vecx$. 
$\rvx[\sigma]\in S_\rvx$ are
 i.i.d. (independent identically distributed) 
samples with

 \beq
x[\sigma]\sim P_\rvx\;\;({\rm i.e.}\; P_{\ul{x[\sigma]}}=P_\rvx)
\eeq

\beq
P(\rvx=x)=\frac{1}{nsam(\vecx)}\sum_\sigma \indi(x[\sigma]=x)
\eeq 
Hence, for any $f:S_\rvx\rarrow \RR$,
\beq
\sum_x P(\rvx=x)f(x)
=\frac{1}{nsam(\vecx)}\sum_\sigma f(x[\sigma])
\eeq 


If we use two sampled variables, say $\vecx$ and $\vecy$, 
in a given bnet, their number of samples 
$nsam(\vecx)$ and $nsam(\vecy)$ need not be equal.

\hrule
\beq
P(\vecx) = \prod_\sigma P(x[\sigma])
\eeq

\beq
\sum_\vecx = \prod_\sigma\sum_{x[\sigma]}
\eeq

\beq
\partial_\vecx = 
[\partial_{x[0]}, \partial_{x[1]},\partial_{x[2]}, \dots, \partial_{x[nsam(\vecx)-1]}]
\eeq

\hrule
\beqa
P(\vecx)&\approx& [\prod_x P(x)^{P(x)}]^{nsam(\vecx)} \\
&=& e^{nsam(\vecx)\sum_x P(x)\ln P(x)}\\
&=& e^{-nsam(\vecx)H(P_\rvx)}
\eeqa


\section{Normal Distribution}
For $x, \mu, \sigma\in \RR$, $\sigma >0$

\beq 
\caln(x; \mu, \sigma^2)=
\frac{1}{\sigma\sqrt{2\pi}}
e^{-\frac{1}{2}\left(
\frac{x-\mu}{\sigma}\right)^2}
\eeq

\section{Uniform Distribution}
For $a<b$, $x\in [a,b]$

\beq
\calu(x;a,b) =
\frac{1}{b-a}
\eeq

\section{Sigmoid and logit functions}

The sigmoid function sig:$\RR\rarrow [0,1]$
is defined by
\beq
\sig(x)=
\frac{1}{1+e^{-x}}
\eeq
sig() is monotonically
increasing with $\sig(-\infty)=0$ and $\sig(+\infty)=1$.

The logit or log-odds function 
logit:$[0,1]\rarrow \RR$ is defined by

\beq
{\rm logit}(p)=\ln\frac{p}{1-p}
\eeq

logit() is the inverse of sig():
\beq
{\rm logit}[\sig(x)]=x
\eeq



\section{Expected Value and Variance}

Given a random variable
 $\rvx$ with states $S_\rvx$ and 
a function $f:S_\rvx\rarrow \RR$, define

\beq
E_\rvx[f(\rvx)]=
E_{x\sim P(x)}[f(x)] = \sum_x P(x) f(x)
\eeq

\beqa
Var_\rvx[f(\rvx)]&=& E_\rvx
\left[(f(\rvx)-E_\rvx[f(\rvx)])^2\right]
\\
&=&
E_\rvx[f(\rvx)^2]-(E_\rvx[f(\rvx)])^2 
\eeqa

\beq
E[\rvx]=E_\rvx[\rvx]
\eeq

\beq
Var[\rvx]=
Var_\rvx[\rvx]
\eeq

\section{Conditional Expected Value}

Given a random variable $\rvx$ with states $S_\rvx$, a random variable $\rvy$ with states $S_\rvy,$ and a function $f:S_\rvx\times S_\rvy\rarrow \RR$, define

\beq
E_{\rvx|\rvy}[f(\rvx, \rvy)]=
\sum_x P(x|\rvy) f(x, \rvy)
\;,
\eeq

\beq
E_{\rvx|\rvy=y}[f(\rvx, y)]=
E_{\rvx|y}[f(\rvx, y)]= \sum_x P(x| y) f(x, y)
\;.
\eeq
Note that

\beqa
E_\rvy[E_{\rvx|\rvy}[f(\rvx, \rvy)]]&=&
\sum_{x,y}P(x|y)P(y)f(x,y)
\\&=&
\sum_{x,y}P(x,y)f(x,y)
\\&=&
E_{\rvx, \rvy}[f(\rvx, \rvy)]
\;.
\eeqa

\section{Law of Total Variance}

\begin{claim}
Suppose $P:S_\rvx\times S_\rvy\rarrow [0,1]$
is a probability distribution.
Suppose $f:S_\rvx\times S_\rvy\rarrow \RR$
 and $f=f(x,y)$. Then
\beq
Var_{\rvx, \rvy}(f)=
E_\rvy[Var_{\rvx|\rvy}(f)]
+
Var_\rvy(E_{\rvx|\rvy}[f])
\;.
\eeq
In particular,
\beq
Var_{\rvx}(x)=
E_\rvy[Var_{\rvx|\rvy}(x)]
+
Var_\rvy(E_{\rvx|\rvy}[x])
\;.
\eeq

\end{claim}
\proof

Let
\beq
A=\sum_y P(y)\left(\sum_x P(x|y)f\right)^2
\;.
\eeq
Then

\beqa
Var_{\rvx, \rvy}(f)&=& \sum_{x,y}P(x,y)f^2 -
\left( \sum_{x,y} P(x,y) f\right)^2
\\
&=&
\left\{
\begin{array}{l}
\sum_{x,y}P(x,y)f^2 
-A
\\
+\left(A-\left( \sum_{x,y} P(x,y) f\right)^2\right)
\end{array}
\right.
\eeqa

\beqa
E_\rvy[Var_{\rvx|\rvy}(f)]
&=&
\sum_y P(y)\left(\sum_x P(x|y)f^2
-
\left(\sum_x P(x|y)f\right)^2
\right)
\\
&=&
\sum_{x,y}P(x,y)f^2 
-A
\eeqa

\beqa
Var_\rvy(E_{\rvx|\rvy}[f])
&=&
\sum_y P(y)
\left(\sum_x P(x|y)f\right)^2
-\left(
\sum_y P(y)\sum_xP(x|y)f
\right)^2
\\
&=&
A-\left( \sum_{x,y} P(x,y) f\right)^2
\eeqa
\qed

\section{Notation
for covariances
}
Consider two random variables $\rvx, \rvy$.

\begin{itemize}
\item
Mean value of $\rvx$
\beq
\av{\rvx}=
E_\rvx[\rvx]
\eeq

\item
Signed distance of $\rvx$ to its mean value
\beq
\Delta \rvx = \rvx - \av{\rvx}
\eeq

\item
Covariance of $(\rvx, \rvy)$
\beq
Cov(\rvx, \rvy)=\av{\rvx, \rvy}=
\av{\Delta \rvx \Delta \rvy}
=
\av{\rvx\rvy}-\av{\rvx}\av{\rvy}
\eeq

$\av{\rvx, \rvy}$ is bilinear.


\item
Variance of $\rvx$
\beq
Var(\rvx)=\av{\rvx, \rvx}
\eeq

\item
Standard deviation or $\rvx$
\beq
\sigma_\rvx=\sqrt{\av{\rvx, \rvx}}
\eeq

\item
Correlation of $(\rvx, \rvy)$
\beq
\rho_{\rvx, \rvy}=
\frac{\av{\rvx, \rvy}}
{\sqrt{\av{\rvx, \rvx}\av{\rvy, \rvy}}}
\eeq
\end{itemize}

\section{Linear regression}


$\rvy=$  true value

$\hat{\rvy}=$ estimator

$\ul{\eps}=$ residual

\beq
\hat{\rvy}=
\beta_0 +\sum_{j=1}^n\beta_j \rvx_j
\eeq

\beq
\rvy = \hat{\rvy}+\ul{\eps}
\eeq
Assume 
\beq
\av{\rvx_j, \ul{\eps}}=0
\eeq
for all $j$.

For $k=1, \ldots, n$,
\beq
\av{\rvx_k, \rvy}
=
\sum_{j=1}^n\beta_j\av{\rvx_k, \rvx_j}
\;.
\eeq
Let $\rvx^n$ and $\beta^n$ be
column vectors.
Then
\beq
\av{\rvx^n, \rvy}=
\av{\rvx^n, (\rvx^n)^T}\beta^n
\;,
\eeq

\beq
\beta^n=
\av{\rvx^n, (\rvx^n)^T}^{-1}
\av{ \rvx^n, \rvy}
\;.
\eeq






\section{Short Summary of 
Boolean Algebra.} 
See Ref.\cite{wiki-bool} for more info
about this topic.

Suppose $x, y, z\in \bool$. Define

\beq
x\text{ or }y=x\V y= x+y-xy
\;,
\eeq

\beq
x \text{ and }y=x\A y= xy
\;,
\eeq
and

\beq
\text{not }x=\ol{x}=1-x
\;,
\eeq
where we are using
normal addition and multiplication 
on the right hand sides.\footnote{Note the
difference between $\V$ and modulus
2 addition $\oplus$. 
For $\oplus$ (aka XOR): $x\oplus y=x+y-2xy$.}



\begin{table}[h!]
\centering
\begin{tabular}{|
>{\columncolor[HTML]{ECF4FF}}l |l|}
\hline
Associativity & \begin{tabular}[c]{@{}l@{}}$x \V (y \V z)=(x \V y) \V z$\\ $x \A (y \A z)=(x \A y) \A z$\end{tabular} \\ \hline
Commutativity & \begin{tabular}[c]{@{}l@{}}$x \V y=y \V x$\\ $x \A y=y \A x$\end{tabular} \\ \hline
Distributivity & \begin{tabular}[c]{@{}l@{}}$x \A (y \V z)=(x \A y) \V (x \A z)$\\ $x \V (y \A z)=(x \V y) \A (x \V z)$\end{tabular} \\ \hline
Identity & \begin{tabular}[c]{@{}l@{}}$x \V 0=x$\\ $x \A 1=x$\end{tabular} \\ \hline
Annihilator & \begin{tabular}[c]{@{}l@{}}$x \A 0=0$\\ $x \V 1= 1$\end{tabular} \\ \hline
Idempotence & \begin{tabular}[c]{@{}l@{}}$x \V x= x$\\ $x \A x= x$\end{tabular} \\ \hline
Absorption & \begin{tabular}[c]{@{}l@{}}$x \A (x \V y)= x$\\ $x \V (x \A y)= x$\end{tabular} \\ \hline
Complementation & \begin{tabular}[c]{@{}l@{}}$x \A \ol{x} = 0$\\ $x \V \ol{x}   = 1$\end{tabular} \\ \hline
Double negation & $\ol{(\ol{x})} = x$ \\ \hline
De Morgan Laws & \begin{tabular}[c]{@{}l@{}}$\ol{x} \A \ol{y} =\ol{(x \V y)}$\\ $\ol{x} \V \ol{y} = \ol{(x \A y)}$\end{tabular} \\ \hline
\end{tabular}
\caption{Boolean Algebra Identities}
\label{tab-bool-alg}
\end{table}

Actually, since
$x\A y=xy$, we can omit writing
the symbol $\A$. The symbol
$\A$ is useful to
exhibit the symmetry
of the identities, and
to remark
about
the analogous identities
for sets, where
$\A$ becomes intersection $\cap$
and $\V$ becomes union $\cup$. However,
for practical calculations,
$\A$ is an unnecessary nuisance.

Since $x\in \bool$,
\beq
P(\ol{x})=1-P(x)
\;.
\eeq

Clearly, from analyzing
the simple event space $(x,y)\in \bool^2$,
\beq
P(x\V y)= P(x) + P(y) - P(x\A y)
\;.
\eeq

\section{Entropy, Kullback-Liebler divergence}

For probabilty distributions $p(x), q(x)$ of $x\in S_\rvx$
\begin{itemize}
\item 
Entropy:
\beq
H(p)=-\sum_x p(x)\ln p(x)\geq 0
\eeq

\item
Kullback-Liebler divergence:

\beq
D_{KL}(p\parallel q)=\sum_{x} p(x)\ln \frac{p(x)}{q(x)}\geq 0
\eeq
\item 
Cross entropy:
\beqa
CE(p\rarrow q) &=& -\sum_x p(x)\ln q(x)\\
&=& H(p) + D_{KL}(p\parallel q)
\eeqa
\end{itemize}

\section{Definition of various
entropies used in Shannon Information Theory}

\begin{itemize}
\item
{\bf (plain) Entropy of $\rvx$}

\beq
H(\rvx) =
-\sum_{x} P(x)\ln P(x)
\eeq
This quantity measures the
spread of $P_\rvx$.


\item
{\bf Conditional Entropy of $\rvy$ given $\rvx$}

\beqa
H(\rvy|\rvx) &=&
-\sum_{x,y}P(x,y)\ln {P(y|x)}
\\
&=&
H(\rvy,\rvx)-H(\rvx)
\eeqa
This quantity measures  the conditional
 spread
of $\rvy$ given $\rvx$.

\item {\bf Mutual Information (MI)
of $\rvx$ and $\rvy$}

\beqa
H(\rvy:\rvx) &=&
\sum_{x,y} P(x,y) \ln \frac{P(x,y)}{P(x)P(y)}
\\
&=&
H(\rvx) + H(\rvy) - H(\rvy,\rvx)
\eeqa
This quantity measures the correlation
between $\rvx$ and $\rvy$.

\item {\bf Conditional Mutual Information 
(CMI)\footnote{CMI
can be read as ``see me".}
of $\rvx$ and $\rvy$
given $\ul{\lam}$}


\beqa
H(\rvy:\rvx|\ul{\lam})
&=&
\sum_{x,y, \lam}P(x,y, \lam) \ln
\frac{P(x,y|\lam)}{P(x|\lam)P(y|\lam)}
\\
&=&
H(\rvx|\ul{\lam}) + H(\rvy|\ul{\lam})
- H(\rvy,\rvx|\ul{\lam})
\eeqa

This
quantity measures the conditional correlation
of $\rvx$ and $\rvy$ given $\ul{\lam}$.

\item {\bf Kullback-Liebler Divergence
from $P_\rvx$ to $P_\rvy$.}

Assume random variables $\rvx$
and $\rvy$
have the same set of states
$S_\rvx=S_\rvy$. Then


\beq
D_{KL}(P_\rvx\parallel P_\rvy)=
\sum_x P_\rvx(x) \ln \frac{P_\rvx(x)}{P_\rvy(x)}
\eeq

This measures a non-symmetric distance
between the probability distributions
$P_\rvx$ and $P_\rvy$. 
$D_{KL}(P_\rvx\parallel P_\rvy)$ 
is non-negative
and equals zero iff $P_\rvx=P_\rvy$.


\end{itemize}




