\chapter{Dynamical Bayesian Networks}
\label{ch-dyn-bnets}

\begin{figure}[h!]
$$
\xymatrix{
\rvc^{(0)}\ar@[green][r]\ar@[green][rdd]
&\rvc^{(1)}\ar@[green][r]\ar@[green][rdd]
&\rvc^{(2)}
&\ldots
&\rvc^{(T-2)}\ar@[green][r]\ar@[green][rdd]
&\rvc^{(T-1)}
\\
\rvb^{(0)}\ar@[green][rd]\ar@[green][r]
&\rvb^{(1)}\ar@[green][rd]\ar@[green][r]
&\rvb^{(2)}
&\ldots
&\rvb^{(T-2)}\ar@[green][rd]\ar@[green][r]
&\rvb^{(T-1)}
\\
\rva^{(0)}\ar[u]
&\rva^{(1)}\ar[u]
&\rva^{(2)}\ar[u]
&\ldots
&\rva^{(T-2)}\ar[u]
&\rva^{(T-1)}\ar[u]
}$$
\caption{
Example of a DBN. Same
time-slice (in black) is repeated $T$ times.
Green arrows connect adjacent slices.
}
\label{fig-dyn-bnet}
\end{figure}



A {\bf dynamical bnet (DBN)} is simply
a Markov chain $\rva_0\rarrow \rva_1\rarrow\ldots \rva_{T-1}$ (see Chapter
\ref{ch-mchain})
for which each node $\rva_i$ is 
called a {\bf time-slice}.
Each time-slice 
represents
at finer resolution a sub-DAG
which has the same 
structure (but
not necessarily the same TPMs)
in every time-slice.\footnote{Sometimes,
it is convenient to
define time-slices that are influenced by
the $n$-previous time-slices.
instead of just $n=1$.
By defining a bigger time-slice that contains
$n$ of the original time-slices,
we can get bigger time-slices
that only listen to 
the adjacent previous bigger time-slice.}
 If the
TPMs are the same for all time-slices,
we call it a {\bf time-homogeneous dynamical bnet}.
Fig.\ref{fig-dyn-bnet} gives an example
of a DBN.
In that figure, each time-slice
is represented in black, and
arrows connecting adjacent time-slices
are represented in green.
In Fig.\ref{fig-dyn-bnet},
we've drawn the 3 nodes of
each time-slice vertically,
and labeled them
with a superscript ${.}^{(t)}$,
where $t\in \{
0,1 \ldots, T-1\}$ 
is the time
of the slice.
To fully 
specify the
DBN
of Fig.\ref{fig-dyn-bnet},
we would also have to specify
the TPMs 

$P(c^{(0)})$, 

$P(b^{(0)})$,

$P(a^{(0)})$,

$P(c^{(1)}|c^{(0)})$,
 
$P(b^{(1)}|b^{(0)}, a^{(1)})$

$P(a^{(1)}|b^{(0)}, c^{(0)})$, etc.

Dynamical bnets
are very common
in AI and Data Science.
Kalman filters (Chapter \ref{ch-kalman}),
Hidden Markov Models (Chapter \ref{ch-hmm})
and
Recurrent Neural Networks 
(Chapter \ref{ch-rnn})
are famous examples of DBNs.

Bnets are acyclic; they can't have cycles
(i.e, closed directed paths).
Yet feedback loops are an important
concept in Science. So what is
the equivalent of feedback loops in the
bnet world? Dynamical bnets are.
Fig.\ref{fig-dyn-bnet-compact}
represents
Fig.\ref{fig-dyn-bnet} more 
compactly using feedback loops. 
Any bnet with feedback loops
can be \enquote{unrolled} into a DBN.


\begin{figure}[h!]
$$
\xymatrix{
\rvc^{(t)}
\ar@/^4pc/@[green]@{-->}[dd]
\ar@(ul,ur)@[green]@{-->}[]
\\
\rvb^{(t)}
\ar@/^1pc/@[green]@{-->}[d]
\ar@(ul,ur)@[green]@{-->}[]
\\
\rva^{(t)}\ar[u]
}$$
\caption{
Dynamical bnet Fig.\ref{fig-dyn-bnet}
represented 
more compactly using feedback loops.
Dashed green arrows
point to the future, from nodes of the $t$ time-slice
to nodes of the $t+1$ time-slice.
}
\label{fig-dyn-bnet-compact}
\end{figure}
