\chapter{Dynamical Bayesian Networks}
\label{ch-dyn-bnet}

\begin{figure}[h!]
$$
\xymatrix{
\rvc^{(0)}\ar@[red][r]\ar@[red][rdd]
&\rvc^{(1)}\ar[r]\ar[rdd]
&\rvc^{(2)}
&\ldots
&\rvc^{(T-2)}\ar[r]\ar[rdd]
&\rvc^{(T-1)}
\\
\rvb^{(0)}\ar@[red][rd]\ar@[red][r]
&\rvb^{(1)}\ar[rd]\ar[r]
&\rvb^{(2)}
&\ldots
&\rvb^{(T-2)}\ar[rd]\ar[r]
&\rvb^{(T-1)}
\\
\rva^{(0)}
&\rva^{(1)}\ar@[red][u]
&\rva^{(2)}\ar[u]
&\ldots
&\rva^{(T-2)}\ar[u]
&\rva^{(T-1)}\ar[u]
}$$
\caption{
Example of a dynamical bnet. The
pattern of red arrows is repeated $T-1$ times.
}
\label{fig-dyn-bnet}
\end{figure}



A dynamical bnet is simply
a time homogeneous Markov chain (see Chapter
\ref{ch-mchain})
for which each node is 
called a {\bf time slice},
and each time slice 
represents
at finer resolution a sub-DAG
which is the same 
between any
2 successive time slices.
Fig.\ref{fig-dyn-bnet} gives an example
of a dynamical bnet.
In Fig.\ref{fig-dyn-bnet},
we've drawn the 3 nodes of
each time slice vertically,
and labeled them
with a superscript ${.}^{(t)}$,
where $t\in \{
0,1 \ldots, T-1\}$ 
is the time
of the slice.
To fully 
specify the
dynamical bnet
of Fig.\ref{fig-dyn-bnet},
we would also have to specify
the TPMs 

$P(c^{(0)})$, 

$P(b^{(0)})$

$P(c^{(1)}|c^{(0)})$,
 
$P(b^{(1)}|b^{(0)}, a^{(1)})$

$P(a^{(1)}|b^{(0)}, c^{(0)})$.

Dynamical bnets
are very common
in AI and Data Science.
Kalman filters (Chapter \ref{ch-kalman}),
Hidden Markov Models (Chapter \ref{ch-hmm})
and
Recurrent Neural Networks 
(Chapter \ref{ch-rnn})
are famous examples of dynamical bnets.

Bnets are acyclic; they can't have cycles
(i.e, closed directed paths).
Yet feedback loops are an important
concept in Science. So what is
the equivalent of feedback loops in the
bnet world? Dynamical bnets are.
Fig.\ref{fig-dyn-bnet-compact}
represents
Fig.\ref{fig-dyn-bnet} more 
compactly using feedback loops. 
Any bnet with feedback loops
can be ``unrolled" into a dynamical bnet.


\begin{figure}[h!]
$$
\xymatrix{
\rvc^{(t)}
\ar@/^4pc/@{~>}[dd]
\ar@(ul,ur)@{~>}[]
\\
\rvb^{(t)}
\ar@/^1pc/@{~>}[d]
\ar@(ul,ur)@{~>}[]
\\
\rva^{(t)}\ar[u]
}$$
\caption{
Dynamical bnet Fig.\ref{fig-dyn-bnet}
represented 
more compactly using feedback loops.
Wavy arrows
point to the future, from nodes of the $t$ time slice
to nodes of the $t+1$ time slice.
}
\label{fig-dyn-bnet-compact}
\end{figure}
