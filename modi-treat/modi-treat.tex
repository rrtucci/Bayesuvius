\chapter{Modified Treatment Policy (COMING SOON)}
\label{ch-modi-treat}

This  chapter
assumes that the
reader has
already read
Chapter \ref{ch-pot-out} on Potential Outcomes (PO) theory.


\url{https://sci-hub.se/10.1002/sim.5907}

\url{https://www.khstats.com/blog/lmtp/lmtp/}

\url{https://en.wikipedia.org/wiki/Generalized_linear_model}


\cite{hernan-book}

{\bf Modified Treatment Policy (MTP)}

\section{One time MTP}
Consider a typical PO bnet $G$
and its corresponding imagined bnet $G_{im2}$ (see Fig.\ref{fig-modi-po}).
A MTP adds a Bayesian prior to the
 node $\ul{\tilx}$ in $G_{im2}$.
The prior depends on the {\bf treatment
variable} (a.k.a. {\bf exposure variable}) $\rvx$ and its parents $pa(\rvx)=\rvc$.
This adds arrows $\rvc\rarrow\ul{\tilx}$
and $\rvx\rarrow \ul{\tilx}$ (see Fig.\ref{fig-unmodi-modi}).
An ATE is calculated by removing all arrows entering $\ul{\tilx}$
so as to mimic a RCT. Hence, a MTP cannot be used to calculate
an ATE or mimic an RCT.


\begin{figure}[h!]
$$
\begin{array}{ccc}
\xymatrix{
\rvc
\ar[dd]\ar[ddrr]
\\
\\
\rvx \ar[rr]
&
&\rvy
}
&
\xymatrix{
\rvc
\ar[dd]\ar[dr] \ar[drr]
\\
& \rvy(0)\ar[dr]
&\rvy(1)\ar[d]
\\
\rvx\ar[rr]
&
&\rvy
}
&
\xymatrix{
\rvc
\ar[dd]\ar[dr] \ar[drr]
\\
& \rvy(0)\ar[dr]
&\rvy(1)\ar[d]
\\
\rvx
&\ul{\tilx}=\tilx\ar[r]
&\rvy
}
\\
G
&G_{im1} =\cali_{\rvx\rarrow\rvy} G
& G_{im2}=\cali_{\ul{\tilx}\rarrow\rvy}\cali_{\rvx\rarrow\rvy}(\tilx)G
\end{array}
$$
\caption{$G$ is one of the
simplest possible bnets considered in  PO theory.
In the usual PO theory,
one only considers the bnet $G_{im1}$.
In MTP theory, we
consider the bnet $G_{im2}$.
For $G_{im2}$, $\tilx\in S_{\ul{\tilx}}=S_\rvx=\bool$.
}
\label{fig-modi-po}
\end{figure}


\begin{figure}[h!]
$$
\begin{array}{cc}
\xymatrix{
pa(\rvx) =\rvc
\ar[dd]
&
&pa'(\rvy)\ar[d]
\\
&&[\rvy(\tilx)]_{\tilx\in S_{\ul{\tilx}}}\ar[d]
\\
\rvx
&\ul{\tilx}=\tilx\ar[r]
&\rvy
}
&
\xymatrix{
pa(\rvx)=\rvc
\ar[dd]\ar[ddr]
&
&pa'(\rvy)\ar[d]
\\
&&[\rvy(\tilx)]_{\tilx\in S_{\ul{\tilx}}}\ar[d]
\\
\rvx\ar[r]
&\ul{\tilx}\ar[r]
&\rvy
}
\\
G_{im2}
 &G_{im2,mod}
\end{array}
$$
\caption{In this figure, $G_{im2}$
is the generalization of the $G_{im2}$
in Fig.\ref{fig-modi-po}.
Note that $pa'(\rvy)$ are the parents in $G$ of
$\rvy$, excluding $\rvx$.
Note that $S_{\ul{\tilx}}\subset S_\rvx$. }
\label{fig-unmodi-modi}
\end{figure}

The TPM, printed in  blue, for node $\ul{\tilx}$
of the imagined bnet $G_{im2}$ in Fig.\ref{fig-unmodi-modi}, is as follows
\beq\color{blue}
P(\ul{\tilx}=\tilx) =\delta(\tilx, x)
\eeq

The TPM, printed in  blue, for node $\ul{\tilx}$
of the modified imagined bnet $G_{im2,mod}$ in Fig.\ref{fig-unmodi-modi},
 is as follows
\beq\color{blue}
P(\ul{\tilx}=\tilx | \rvx=x, c)= \text{ prior}
\eeq

Hence, node $\ul{\tilx}$
has a non-informative (frequentist) prior in $G_{im2}$,
and it has an informative (Bayesian) prior in $G_{im2,mod}$.

Assumptions about $G_{im2, mod}$

\begin{enumerate}


\item {\bf Consistency (a.k.a. SUTVA)}

The TPM, printed in  blue, for node $\rvy$
of the modified imagined bnet $G_{im2,mod}$ in Fig.\ref{fig-unmodi-modi},
 is as follows

\beq \color{blue}
P(y\cond \tilx, \{y(\tilx')\}_{\tilx'\in S_{\ul{\tilx}}})=
\delta(y , \sum_{\tilx'\in S_{\ul{\tilx}}}
 y(\tilx')\indi(\tilx'=\tilx))
\eeq

\beqa
P(\rvy=y|\tilx, pa'(\rvy)=\xi)&=&
P(\rvy(\tilx)=y|\tilx, pa'(\rvy)=\xi)
\eeqa


\item {\bf Exchangeability (a.k.a.
conditional independence assumption (CIA))}

\beqa
P(\rvy(\tilx')=y|\tilx, pa'(\rvy)=\xi)
&=&
P(\rvy(\tilx')=y|pa'(\rvy)=\xi)
\eeqa

\item {\bf Identifiability}

We must have
\beq
S_{\ul{\tilx}} \subset S_\rvx
\eeq
in $G_{im2, mod}$
or else queries of the type
 $P(\rvy(\tilx)=y|pa'(\rvy)=\xi)$ are not identifiable.
This is clear because if $\tilx\not\in S_\rvx$,
then we have no information of the type
$P(\rvy=y|\tilx, pah(\rvy)=c)=P(\rvy(\tilx)=y|pa'(\rvy)=\xi)$.

\item{\bf Positivity}

Define the following 2 propensities
for $x\in S_\rvx$ and $\tilx \in S_{\ul{\tilx}}$:

\beq
g_{x|c}=P(\rvx=x |c)
\eeq

\beq
\tilg_{\tilx|c}=P(\ul{\tilx}=\tilx |c)
\eeq

Positivity for  $G_{im2,mod}$
is the requirement that

\beq
0<g_{\tilx|c}<1 \text{ for all } \tilx \in S_{\ul{\tilx}}
\eeq
If $g_{\tilx|c}$
is deterministic, then this requirement
is not satisfied, and we cannot do IPW (i.e.,
inverse propensity weighing).
If $g_{\tilx|c}$
is defined to equal $g_{f(\tilx)|c}$ for some
function $f:S_{\ul{\tilx}}\rarrow S_\rvx$
and $S_{\ul{\tilx}}\subset S_\rvx$, then
positivity is assured.

Note that using a MDP can allow IPW to
be performed  in cases
when the propensity $g_{x|c}$
is not defined for all $x\in S_\rvx$, but is defined
for all $x\in S_{\ul{\tilx}}$, where
$S_{\ul{\tilx}}$
is some proper subset of $S_\rvx$.

\end{enumerate}

The probability
distribution
$P(\tilx|x, c)$ (i.e., the  $\ul{\tilx}$ prior for $G_{im2,mod}$)
is called the {\bf modified treatment plan (MTP)}.
A MTP can be deterministic or probabilistic.
An MTP that depends (resp., does not depend)
 on $x$ is said to
be dynamic (resp., static), because $\tilx$ depends (resp., does not
depend) on the previous value
of $x$.

\begin{enumerate}
\item Deterministic MTP

Suppose $S_{\ul{\tilx}}\subset S_\rvx$
and we are given a function
$\tilx_c(\cdot):S_\rvx\rarrow S_{\ul{\tilx}}$.
Then let the TPM, printed
in blue, for node $\ul{\tilx}$, be as follows:

\beq\color{blue}
P(\ul{\tilx}=\tilx | \rvx=x, c)= \delta(\tilx, \tilx_c(x))
\eeq

Examples of $\tilx_c(x)$
\begin{itemize}

\item threshold for in $x$ for small $x$

Let $S_\rvx=[a,b]$ where $a<b$.
For some $\theta, u(c)\in [a,b]$,

\beq
\tilx_c(x) = \left\{\begin{array}{ll}
\theta &\text{ if }x\leq u(c)
\\
x &\text{ if }x> u(c)
\end{array}
\right.
\eeq

\item upshift in $x$ for small $x$

Let $S_\rvx=[a,b]$ where $a<b$.
For some $\Delta x >0$,
and $u(c)\in[a,b]$,

\beq
\tilx_c(x) = \left\{\begin{array}{ll}
(x +  \Delta x) &\text{ if }x< u(c)\text{ and }
(x +  \Delta x)\in[a,b]
\\
x &\text{ if }x> u(c)
\\
b & \text{ if }(x +  \Delta x)>b
\end{array}
\right.
\eeq



\end{itemize}


\item Stochastic MTP

For some
convenient, user specified,
probability
distribution
$P(\tilx| x', x, c)$,
where $\tilx\in S_{\ul{\tilx}}$,
$x,x'\in S_\rvx$,
 and $c\in S_\rvc$,
let the TPM, printed
in blue, for node $\ul{\tilx}$, be as follows:


\beq\color{blue}
P(\tilx|x,c)=
\sum_{x'\in S_\rvx} P(\tilx| x', x, c)
\underbrace{P(\rvx=x'|c)}_{\text{unmodified propensity}}
\eeq

Examples
\begin{itemize}
\item
Suppose we are given a function $x_c:S_{\ul{\tilx}}\rarrow S_\rvx$.
($x_c(\tilx)$ could be the inverse, if
 it exists, of the function $\tilx_c(x)$
defined in the deterministic TMP case.)

\beq
P(\tilx|x,c) = \frac{P(\rvx=x_c(\tilx)|c)}{\sum_\tilx numerator}
\eeq

\item
Suppose
$S_\rvx=S_{\ul{\tilx}}=\bool$, $\theta\in\RR$,
and let


\beq
P(\tilx|x, c)
=
[\pi(x,c)]^{\tilx} [1-\pi(x,c)]^{1-\tilx}
\eeq
where

\beq
\pi(x,c)=
\frac{\theta^x P(\rvx=x|c)}{\sum_x numerator}
\eeq

\end{itemize}


\end{enumerate}


\subsection{PIE}
\beqa
\caly_{|x, c}
&=&
\sum_y y P(\rvy(x)=y\cond x, c)
\\
&=& \sum_y y P(\rvy=y\cond x, c)
\eeqa

\beqa
\TIL{\caly}_{|x, c}
&=&
\sum_\tilx\sum_y y P(\rvy(\tilx)=y\cond \tilx, c)
P(\tilx|x,c)
\\
&=&
\sum_\tilx\sum_y y P(\rvy=y\cond \tilx, c)
P(\tilx|x,c)
\eeqa

\beq
\caly_{|c} = \sum_{x} \caly_{|x, c}P(x|c)
,\quad \caly =\sum_c P(c)\caly_{|c}
\eeq

\beq
\TIL{\caly}_{|c} = \sum_{x} \TIL{\caly}_{|x, c}P(x|c)
,\quad \TIL{\caly} =\sum_c P(c)\TIL{\caly}_{|c}
\eeq





PIE Estimand

\beq
\Delta_{|c} = \caly_{|c} -\TIL{\caly}_{|c},\quad \Delta = \sum_c P(c)\Delta_{|c}
\eeq

\begin{claim}
Consider the deterministic MTP case $P(\tilx|x,c)=
\delta(\tilx , \tilx_c(x))$,
where the function $\tilx=\tilx_c(x)$ is
invertible with inverse $x=x_c(\tilx)$
for all $x\in \XX\subset \RR$.
Then, for all $x\in \XX$,

\beq
\TIL{\caly}_{|x, c}=
\caly_{|x_c(\tilx), c}
\eeq
\end{claim}
\proof


\beqa
\TIL{\caly}_{|x, c} &=&
\sum_\tilx\sum_y y P(\rvy=y\cond \ul{\tilx}=\tilx, c)
\delta(\tilx, \tilx_c(x))
\\
&=&
\sum_y y P(\rvy=y\cond \ul{\tilx}=\tilx_c(x), c)
\\
&=&
\sum_y y P(\rvy=y\cond \rvx=x_c(\tilx), c)
\\
&=&
\caly_{|x_c(\tilx), c}
\eeqa
\qed



Henceforth, assume the
deterministic MTP case
$P(\tilx|x,c)=\delta(\tilx, \tilx_c(x))$.
We will also
assume that
the domain of $\tilx_c(x)$
is a union of disjoint set $\XX^j_c$
for $j=1,2, \dots, nj(c)$,
and that on each set $\XX^j_c$,
$\tilx_c(x)$ is invertible and differentiable.

\beq
\XX_c = \cup_{j=1}^{nj(c)}\XX_{c}^j
\eeq

\beq
\tilx_c(x) =\sum_{j=1}^{nj(c)}  \indi(x\in \XX_c^j)\tilx_c^j(x)
\eeq


\beq
P(\tilx|x, c)=\delta(\tilx, \tilx_c(x))=
\sum_j \indi(x\in \XX_c^j)
\delta(\tilx, \tilx^j_c(x))
\eeq

\beq
(\tilx^j_c)^{-1}(\tilx)= x^j(\tilx)
\eeq

\beq
\TIL{\XX}_c^j= \tilx^j_c(\XX^j_c)=\{\tilx^j_c(x): x\in \XX^j_c\}
\eeq

\beq
P_\XX = P(\rvx\in \XX)
\eeq

\begin{claim}

\beq
\TIL{\caly}_{|c}=
\frac{1}{P_\XX}
E_{\rvy, \ul{\tilx}|c}[
\lam_c(\tilx)\caly_{|x_c(\tilx),c}]
\;,
\eeq

\beqa
\caly_{|c}&=&
\frac{1}{P_\XX}
E_{\rvy, \ul{\tilx}|c}[
\lam_c(\tilx)y]
\eeqa
and

\beq
P_\XX =
E_{\rvy, \ul{\tilx}|c}[
\lam_c(\tilx)]
\eeq
where

\beq
\lam_c(\tilx)=\frac{1}{P_\XX}
\sum_j \indi(\tilx\in \TIL{\XX}^j_c)
\der{x^j_c(\tilx)}{\tilx}\frac{P(\rvx=x^j_c(\tilx)|c)}{P(\tilx|c)}
\eeq
\end{claim}
\proof

\begin{align}
\TIL{\caly}_{|c}
&=
\frac{1}{P_\XX}
\int_{x\in \XX_c}dx\;P(x|c)
\sum_y y P(y|\ul{\tilx}=\tilx_c(x), c)
\\
&=
\frac{1}{P_\XX}
\sum_j \int_{x\in\XX^j_c} dx\;P(x|c)
\sum_y y P(y|\ul{\tilx}=\tilx_c^j(x),c)
\\
&=
\frac{1}{P_\XX}
\sum_j
\int_{\tilx\in \TIL{\XX}^j_c}
 d\tilx\; P(\tilx|c)
\der{x^j_c(\tilx)}{\tilx}
\frac{P(\rvx=x^j_c(\tilx)|c)}{P(\tilx|c)}
\underbrace{\sum_y y P(y|\rvx=x_c^j(\tilx),c)}_{\caly_{|x_c^j(\tilx),c}}
\\
&=
\int_{\tilx\in \TIL{\XX}_c}
 d\tilx\; P(\tilx|c)
\lam_c(\tilx)
\caly_{|x_c(\tilx),c}
\end{align}

\beqa
\caly_{|c}&=&
\frac{1}{P_\XX}
\int_{\tilx\in \TIL{\XX}_c}
 d\tilx\; P(\tilx|c)
\lam_c(\tilx)
\caly_{|\tilx,c}
\\
&=&
\frac{1}{P_\XX}
\int_{\tilx\in \TIL{\XX}_c}
 d\tilx\; P(\tilx|c)
\lam_c(\tilx)
\sum_y y P(y|\rvx=\tilx,c)
\eeqa
\qed

\beq
\Delta_{|c} = \frac{E_{\rvy, \ul{\tilx}|c}\left[
\lam_c(\tilx)
(y-\caly_{|x_c(\tilx),c})\right]}
{E_{\ul{\tilx}|c}[
\lam_c(\tilx)]}
\eeq


\subsection{PIE Estimators}

\beq
N_c =\sum_\s \delta(c_\s, c)
\eeq

\beq
P(\s|c)= \frac{\delta(c_\s, c)}{N_c}
\eeq

\beq
E_{\s|c}[\xi_\s] =\sum_\s P(\s|c) \xi_\s
\eeq


\beq
\HAT{\Delta}_{|c} = \frac{E_{\s|c}\left[
\lam_{c}(\tilx_\s)
(y_\s-\caly_{|x_{c}(\tilx_\s),c})\right]}
{E_{\s|c}[
\lam_c(\tilx_\s)]}
\eeq



\beq
g({\caly}_{|x_\s,c_\s}) = \xbeta
\eeq
where

\beq
X^T= [1, x, c],\quad X_\s^T= [1, x_\s, c_\s]
\eeq

\beq
\HAT{\caly}(X^T\HAT{\beta}) = g^{-1}(X^T\HAT{\beta})
\eeq

\beq
\HAT{\Delta}_{OR}=E_\s[
\lam_c(x_\s) (y_\s - \HAT{\caly}(X^T\HAT{\beta}))]
\eeq



\section{Multi-time MTP}

Multi-time (a.k.a. longitudinal) MTP (LMTP)
g-formula for LMTP\footnote{Other formulae
that
define recursively the full
probability distribution of a bnet,
are also called g-formulae. Chapter
\ref{ch-g-formula} defines
a different g-formula.}



Consider  $t=1,2, \ldots, nt$,

For $nt=1$,

\beq
\begin{array}{cc}
\xymatrix{
\rvc_1\ar[d]\ar[dr]
\\
\rvx_1\ar[r]
&\rvy
}
 &
\xymatrix{
\rvc_1\ar[d]\ar@/^1pc/[dd]\ar[ddr]
\\
\rvx_1\ar[d]
\\
\ul{\tilx}_1\ar[r]
&\rvy
}
\end{array}
\eeq

\beq
P(y,x_1, c_1)=
P(y|x_1, c_1)
P(x_1|c_1)
P(c_1)
\eeq

\beq
P(y,\tilx_1,x_1, c_1)=
P(y|\tilx_1, c_1)
P(x_1|c_1)
P(\tilx_1|x_1, c_1)P(c_1)
\eeq
For $nt=2$,

\beq
\begin{array}{ll}
\xymatrix{
\rvc_1\ar[d]\ar[r]\ar[dr]
&\rvc_2\ar[d]\ar[rd]
\\
\rvx_1\ar[r] \ar[ru]
&\rvx_2\ar[r]
&\rvy
}
&
\xymatrix{
\rvc_1\ar[d]\ar[r]\ar@/^1pc/[dd]\ar[ddr]\ar[dr]
&\rvc_2\ar[d]\ar@/^1pc/[dd]\ar[ddr] \ar[rdd]
\\
\rvx_1\ar[d]
&\rvx_2\ar[d]
\\
\ul{\tilx}_1\ar[ru] \ar[ruu] \ar[r]
&\ul{\tilx}_2\ar[r]
&\rvy
}
\end{array}
\eeq

\beq
P(y,x_{\leq 2},c_{\leq 2})=
\left\{
\begin{array}{l}
P(y|x_2, c_2)
\\
*P(x_1|c_1) P(c_1)
\\
*P(x_2|x_1, c_{2,1})P(c_2|x_1, c_1)
\end{array}
\right.
\eeq

\beq
P(y,\tilx_{\leq 2}, x_{\leq 2},c_{\leq 2})=
\left\{
\begin{array}{l}
P(y|\tilx_2, c_2)
\\
*P(x_1|c_1)
P(\tilx_1|x_1, c_1) P(c_1)
\\
*P(x_2|\tilx_1, c_{2,1})
P(\tilx_2|x_2, c_{2,1})
P(c_2|\tilx_1, c_1)
\end{array}
\right.
\eeq


For $nt=3$,

\beq
\begin{array}{cc}
\xymatrix{
\rvc_1\ar[d]\ar[r]\ar[dr]
&\rvc_2\ar[d]\ar[r]\ar[dr]
&\rvc_3\ar[d] \ar[rd]
\\
\rvx_1\ar[r]\ar[ru]
&\rvx_2\ar[r]\ar[ru]
&\rvx_3\ar[r]
&\rvy
}
&
\xymatrix{
\rvc_1\ar[d]\ar[r]\ar[ddr]\ar[dr]\ar@/^1pc/[dd]
&\rvc_2\ar[d]\ar[r]\ar[ddr] \ar[dr]\ar@/^1pc/[dd]
&\rvc_3\ar[d]\ar[ddr] \ar@/^1pc/[dd] \ar[rdd]
\\
\rvx_1\ar[d]\ar[rd]
&\rvx_2\ar[d]\ar[rd]
&\rvx_3\ar[d]
\\
\ul{\tilx}_1\ar[ru] \ar[ruu] \ar[r]
&\ul{\tilx}_2\ar[ru]\ar[ruu]  \ar[r]
&\ul{\tilx}_3\ar[r]
&\rvy
}
\end{array}
\label{eq-modi-bnet-nt-3}
\eeq

\beq
P(y, x_{\leq 3}, c_{\leq 3})
=
\left\{
\begin{array}{l}
P(y| x_3, c_3)
\\
*P(x_1|c_1) P(c_1)
\\
*P(x_2|x_1, c_{2,1})P(c_2| x_1, c_1)
\\
*P(x_3|x_{2}, c_{3,2}) P(c_3|x_2, c_2)
\end{array}
\right.
\eeq


\beq
P(y,\tilx_{\leq 3}, x_{\leq 3}, c_{\leq 3})
=
\left\{
\begin{array}{l}
P(y|\tilx_3, c_3)
\\
*P(x_1|c_1)
P(\tilx_1|x_1, c_1)P(c_1)
\\
*P(x_2|\tilx_1, c_{2,1})
P(\tilx_2|\tilx_1, x_{2,1}, c_{2,1})  P(c_2|\tilx_1, c_1)
\\
*P(x_3|\tilx_{2}, c_{3,2})
P(\tilx_3|\tilx_{2}, x_{3,2}, c_{3,2}) P(c_3|\tilx_2, c_2)
\end{array}
\right.
\eeq

In general,

\beq
P(y,x_{\leq nt}, c_{\leq nt})=
P(y|x_{nt}, c_{nt})
\prod_{t=1}^{nt}
\{
P(x_t|x_{t-1}, c_{t, t-1} )
P(c_t|\tilx_{t-1}, c_{t-1})
\}
\eeq

\beq
P(y,\tilx_{\leq nt},x_{\leq nt}, c_{\leq nt})=
\left\{
\begin{array}{l}
P(y|\tilx_{nt},x_{nt}, c_{nt})
\\
*\prod_{t=1}^{nt}
P(x_t|\tilx_{t-1}, c_{t, t-1} )
P(\tilx_t|\tilx_{t-1}, x_{t,t-1}, c_{t, t-1})
P(c_t|\tilx_{t-1},  c_{t-1})
\end{array}
\right.
\eeq



Note that Ref.\cite{hernan-book} by Hern\'{a}n and Robins  (HR) uses the
more common notation $L_{t} = \rvc_{t}$,
$A_{t} = \rvx_{t}$
for $t=1, 2, \ldots, nt$. Hence, the $nt=2$ bnet
Eq.(\ref{eq-modi-bnet-nt-3}), written in the HR notation, is
as follows
(before and after adding the $\TIL{A}$ nodes):
\beq
\begin{array}{cc}
\xymatrix{
L_1\ar[d]\ar[r]\ar[dr]
&L_2\ar[d]\ar[r]\ar[dr]
&L_3\ar[d]\ar[rd]
\\
A_1\ar[r]\ar[ru]
&A_2\ar[r]\ar[ru]
&A_3\ar[r]
&\rvy
}
&
\xymatrix{
L_1\ar[d]\ar[r]\ar[ddr]\ar[dr]\ar@/^1pc/[dd]
&L_2\ar[d]\ar[r]\ar[ddr] \ar[dr]\ar@/^1pc/[dd]
&L_3\ar[d]\ar[ddr] \ar@/^1pc/[dd]
\\
A_1\ar[d]\ar[rd]
&A_2\ar[d]\ar[rd]
&A_3\ar[d]
\\
\TIL{A}_1\ar[ru]\ar[r]
&\TIL{A}_2\ar[ru]\ar[r]
&\TIL{A}_3\ar[r]
&\rvy
}
\end{array}
\eeq
HR doesn't actually display the $\TIL{A}$
nodes in his DAGs. Furthermore, in HR,
 nodes $A_t$ and $L_t$
have parent nodes
that are time steps $0,1,2,3,\ldots $ in the past.
I, on the other hand, will
not draw or consider parents that are 2 or more
time steps in the past,
because I will assume the effect of those parents is
negligible to first approximation.
