\chapter{Finite State Machine}
\label{ch-finite-state}

This chapter is mostly based on Refs.\cite{fsm-brilliant}, \cite{wiki-det-automaton} and
\cite{wiki-nondet-automaton}.

A {\bf Finite State Machine} (FSM) (a.k.a. {\bf Finite Automaton} (FA))  can be explained as a 
special case of a Petri net, and that
is the approach that we take in this chapter.
Petri nets are discussed in Chapter \ref{ch-petri}. We will assume
that the reader has read that chapter before tackling this one.

{\bf Some nomenclature:}

Let $|S|$ denote the number of elements
in a set $|S|$.
The {\bf power set of set $S$}, denoted by $2^S$,
is the set of all the subsets of $S$, including the empty set.
Note that $|2^S|=2^{|S|}$. Indeed,
consider the binomial expansion

\beq
(1 + 1)^{|S|}=\sum_{i=0}^{|S|}{|S| \choose i}
\eeq
Now note that  ${|S|\choose i}$
is the number of subsets of $S$ with $i$ elements.

Consider a function
$f:A\rarrow B$. 

The {\bf domain} of $f$ is $A$,
and its {\bf range} is $B$.

The {\bf image} of $f$
is $Im(f)=\{f(a): a\in A\}$. 

$f$ is {\bf 1-1} if $f(a_1)=f(a_2)$
implies $a_1=a_2$.


$f$ is {\bf onto} if $Im(f)=B$.

\section{Deterministic FSM}

\subsection{Example}

\begin{figure}[h!]
$$
\begin{array}{cccc}
\xymatrix@C=3pc@R=3pc{
\ar[d] 
\\
\PinkCircle{\rvp_1}\ar@/^1pc/[r]|a
&
\Circle{\rvp_2}\ar@/^1pc/[l]|b
\ar[d]|c \ar@(ru,lu)[]_a
\\
\Circle{\rvp_3}\ar@/^1pc/[r]|b
\ar[u]|a
&\DCircle{\rvp_4}\ar@/^1pc/[l]|d
}&
\xymatrix@C=3pc@R=3pc{
\ar[d] 
\\
\Circle{\rvp_1}\ar@/^1pc/[r]|a
&
\PinkCircle{\rvp_2}\ar@/^1pc/[l]|b
\ar[d]|c \ar@(ru,lu)[]_a
\\
\Circle{\rvp_3}\ar@/^1pc/[r]|b
\ar[u]|a
&\DCircle{\rvp_4}\ar@/^1pc/[l]|d
}&
\xymatrix@C=3pc@R=3pc{
\ar[d] 
\\
\Circle{\rvp_1}\ar@/^1pc/[r]|a
&
\PinkCircle{\rvp_2}\ar@/^1pc/[l]|b
\ar[d]|c \ar@(ru,lu)[]_a
\\
\Circle{\rvp_3}\ar@/^1pc/[r]|b
\ar[u]|a
&\DCircle{\rvp_4}\ar@/^1pc/[l]|d
}&
\xymatrix@C=3pc@R=3pc{
\ar[d] 
\\
\Circle{\rvp_1}\ar@/^1pc/[r]|a
&
\Circle{\rvp_2}\ar@/^1pc/[l]|b
\ar[d]|c \ar@(ru,lu)[]_a
\\
\Circle{\rvp_3}\ar@/^1pc/[r]|b
\ar[u]|a
&\PinkDCircle{\rvp_4}\ar@/^1pc/[l]|d
}
\end{array}
$$
\caption{Example of a deterministic FSM. Pink nodes carry the single pink token.
The pink token's movements trace out the string \qt{aac}.}
\label{fig-det-fsm}
\end{figure}
Fig.\ref{fig-det-fsm}
gives an example of a deterministic
FSM. A FSM contains exactly one {\bf pink place node}. We will 
imagine that this pink place node contains a single pink token. This token moves around with each step.
Each step appends a string to the {\bf string history} of the pink token.

\subsection{Precise Definition}

Recall the following terms used for pnets:

$\calx =\{\rvx_i\}_{i=1}^{nx}$ are the {\bf transition nodes} of the pnet.

$\calp =\{\rvp_i\}_{i=1}^{np}$ are the {\bf place (or state) nodes} of the pnet.\footnote{In this
chapter, since we are treating FSM as a special case of pnets, we use the words \qt{state} and
\qt{place} interchangeably.}

$\calx(\rvp\rarrow)=\{\rvx: \rvp\rarrow\rvx \text{ is an arrow of the pnet}\}$.

\hrule
New terms and notation not used for pnets:

$\Sigma=$ {\bf alphabet}, a finite set of symbols
from which to create strings. For example,
$\Sigma$ might be $\{0, 1\}$ or $\{a, b, c\}$.

$\lam:\calx\rarrow\Sigma$, is the {\bf transition labelling function}. $\lam(\rvx)=a$ means alphabet symbol $a$ is assigned to transition $\rvx$.  


$\lam_\rvp:\calx(\rvp\rarrow)\rarrow\Sigma$, this is the {\bf restriction of $\lam()$}
to the domain $\calx(\rvp\rarrow)$.


$\rvp(0)\in \calp$, {\bf starting place}. Indicated by arrow starting from nothing.



$\calp_{acc}\subset \calp$ are the 
{\bf acceptor places}. The pink token must end in an acceptor place, but it may visit
an acceptor place multiple times.


$\Delta:\calx\times\calp\rarrow \calp$ is the 
{\bf transition function}.\footnote{It's more common to call the function $\delta:\Sigma\times\calp\rarrow \calp$, with $\delta(\lam(\rvx), \rvp)=\Delta(\rvx, \rvp)$, 
the
transition function $\delta$.}
It maps each place node $\rvp(t)\in\calp$ and transition $\rvx(t)\in\calx$ to the next place node $\rvp(t+1)\in\calp$ in a {\bf place history} $\rvp(0), \rvp(1), \ldots$. We will write this as

\beq
\ket{\rvp(t+1)}=\rvx(t)\ket{\rvp(t)}
\eeq
The place history is the places visited by the pink token. Each transition of
the pink token  between places appends a string
to the pink tokens's {\bf string history}.




\hrule
Recall from Chapter \ref{ch-petri} that a Petri net is defined as $\Phi_{pnet} = \av{\calp, \calx, W,\ket{b(0)}_\calp}$.
A {\bf Finite Automaton (FA)}
is defined by adding extra stuff to
$\Phi_{pnet}$, namely by
$\Phi_{FA}=\av{\Phi_{pnet},\calp_{acc}, \Sigma, \lam, \Delta}$\footnote{This definition has some redundancy.  
$\av{\calp, \Sigma, p(0), \calp_{acc}, \delta}$
is sufficient.}.
$\ket{b(0)}_\calp$ in the definition
of $\Phi_{pnet}$ is replaced by
$\rvp(0)\in\calp$.

Let $h=a_1a_2a_3\ldots a_n$ be a string
over the alphabet $\Sigma$.
The automaton $\Phi_{FA}$ {\bf accepts (or generates) the string} $h$
if a sequence of places $\rvp(0), \rvp(1), \ldots, \rvp(T)$ exists
in $\calp$ such that 
\begin{enumerate}
\item $\rvp(0)$  is the starting place
\item $\ket{\rvp(t+1) }= \rvx(n)\ket{\rvp(t)}$
\item $\rvp(T)\in \calp_{acc}$
\end{enumerate}
If the automaton does not accept the string $h$,
it is said to {\bf reject} it. The set $\call(\Phi_{FA})$ of all strings 
accepted by the FA is called its {\bf language}.

\begin{mdframed}[hidealllines=true,backgroundcolor=blue!10]
For a {\bf deterministic FA}:
\begin{itemize}
\item For each $\rvp\in\calp$, the function  $\lam_\rvp$ is 1-1.\footnote{If it weren't 1-1,
it would have to choose 
between 2 outgoing paths
that appended the same string to the string history.
This choice would make
the FSM non-deterministic.}

\item
we will often indicate a node for 
which $\lam_{\rvp}(\rvx_i)=a$ either by $\rvx_i=a$
or simply by $a$. 
\end{itemize}\end{mdframed}

\section{Non-deterministic FSM}
\subsection{Example}

\begin{figure}
$$
\xymatrix{
&&\Circle{\rvp_1}\ar[r]|0
&\Circle{\rvp_2}\ar[r]|1
&\DCircle{\rvp_3}
\\
\ar[r]
&\PinkCircle{\rvp_4}\ar@(ru,lu)[]_{0,1}
\ar[ru]|\emptyset
\ar[rd]|\emptyset
\\
&&\Circle{\rvp_5}\ar[r]|1
&\Circle{\rvp_6}\ar[r]|0
&\DCircle{\rvp_7}
}
$$
\caption{
Example of a 
non-deterministic FSM.}
\label{fig-non-det-fsm}
\end{figure}
Fig.\ref{fig-non-det-fsm}
gives an example of a non-deterministic
FSM.


\subsection{Precise Definition}
For a non-deterministic FSM,
everything we said for a deterministic FSM,
except for the shaded box, is valid. The shaded 
box must be replaced by the following shaded box:

\begin{mdframed}[hidealllines=true,backgroundcolor=blue!10]
For a {\bf non-deterministic FA}:
\begin{itemize}
\item
$\lam_\rvp:\calx(\rvp\rarrow)\rarrow 2^\Sigma$ 
instead of 
$\lam_\rvp:\calx(\rvp\rarrow)\rarrow \Sigma$ (i.e., the range of $\lam_\rvp$
is the power set of $\Sigma$ 
instead of $\Sigma$).
The function $\lam_\rvp$ is not necessarily
1-1.
Note that the new range contains the empty set.
If $\lam_\rvp(\rvx)=\emptyset$, we say there is a  {\bf null transition} for arrow 
$\rvp\rarrow \rvx$.\footnote{The empty set is sometimes denoted by $\eps$ 
when discussing FSM.} A null transition 
transfers the node across without adding anything to the string history. 
\item
we will often indicate a node for 
which $\lam_{\rvp}(\rvx_i)=\{a,b, \ldots\}\subset\Sigma$ either by $\rvx_i=a, b, \ldots$
or simply by $a, b, \ldots$. 
\end{itemize}\end{mdframed}

\section{Equivalency of deterministic and non-deterministic FSM}

At first glance, it might
seem that non-deterministic FSM
are more general than  deterministic ones.
However, they are equivalent
in the sense that they generate identical languages (i.e., collections of strings). Their equivalence can be
established by using the transformations described in Figs.\ref{fig-shrink-null-tra},
\ref{fig-split-x-y}
and \ref{fig-merge-two-x}
to reduce  a non-deterministic FSM
to a deterministic one.


\begin{figure}[h!]
$$
\xymatrix@R=.1pc{
& a\ar[dl]
\\
\Circle{\rvp_1}\ar[rd]
\ar[dd]
\\
& b
\\
\emptyset\ar[dd]
\\
&c\ar[dl]
\\
\Circle{\rvp_2}\ar[rd]
\\
&d
\save "2,1"."6,1"*++[F--]\frm{}
\restore
}
$$
\caption{Place nodes $\rvp_1, \rvp_2$ connected by
a null transition can be merged into a single
place node. This will eliminate the null transition.}
\label{fig-shrink-null-tra}
\end{figure}

\begin{figure}[h!]
$$
\begin{array}{ccc}
\xymatrix@R=.3pc{
& a\ar[dl]
\\
\Circle{\rvp_1}\ar[rd]
\ar[dd]
\\
& b
\\
x,y\ar[dd]
\\
&c\ar[dl]
\\
\Circle{\rvp_2}\ar[rd]
\\
&d
}
&
\xymatrix
{
\\
\\
\implies}
&
\xymatrix@R=.3pc{
& a\ar[dl]
\\
\Circle{\rvp_1}\ar[rd]
\ar[ddd]|x
\ar@/_2pc/[ddddd]|y
\\
& b
\\
\\
\Circle{\rvp_2'}\ar[rdd]
&c\ar[ldd]\ar[l]
\\
\\
\Circle{\rvp_2''}\ar[r]
&d
}
\end{array}
$$
\caption{A place node $\rvp_2$
that receives two alphabet 
symbols $x, y$ at once, 
can be cloned into two place nodes
$\rvp_2'$ and $\rvp_2''$ that receive
 $x$ and $y$ respectively.
}
\label{fig-split-x-y}
\end{figure}


\begin{figure}[h!]
$$
\begin{array}{ccc}
\xymatrix@R=.3pc{
& a\ar[dl]
\\
\Circle{\rvp_1}\ar[rd]
\ar[ddd]|x
\ar@/_2pc/[ddddd]|x
\\
& b
\\
\\
\Circle{\rvp_2'}\ar[rdd]
&c\ar[ldd]\ar[l]
\\
\\
\Circle{\rvp_2''}\ar[r]
&d
}&
\xymatrix
{
\\
\\
\implies}
&
\xymatrix@R=.3pc{
& a\ar[dl]
\\
\Circle{\rvp_1}\ar[rd]
\ar[dd]
\\
& b
\\
x\ar[dd]
\\
&c\ar[dl]
\\
\Circle{\rvp_2}\ar[rd]
\\
&d
}
\end{array}
$$
\caption{Two place nodes $\rvp_2'$ and $\rvp_2''$
that receive the same symbol $x$, can be
merged into a single place node $\rvp_2$.}
\label{fig-merge-two-x}
\end{figure}