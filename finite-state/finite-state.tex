\chapter{Finite State Machine}
\label{ch-finite-state}

{\bf Finite State Machine} (FSM) (a.k.a. {\bf Finite Automaton}, (FA))

$\calx =\{\rvx_i\}_{i=1}^{nx}$ are the {\bf transition nodes} of the pnet.

$\calp =\{\rvp_i\}_{i=1}^{np}$ are the {\bf place (or state) nodes} of the pnet.

$\calx(\rvp\rarrow)=\{\rvx: \rvp\rarrow\rvx \text{ is an arrow of the pnet}\}$.

\hrule
New terms and notation:

$\calp_{acc}\subset \calp$ are the 
{\bf accepting states}. States which, if reached, terminate a trajectory.



$\Sigma=$ {\bf alphabet}, a finite set of symbols
from which to create strings. For example,
$\Sigma$ might be $\{0, 1\}$ or $\{a, b, c\}$.


$f_\rvp:\calx(\rvp\rarrow)\rarrow \Sigma$ for each
$\rvp\in \calp$,  is the 
{\bf transition function} for place node $\rvp$. 
\begin{mdframed}[hidealllines=true,backgroundcolor=blue!10]We will often write in a 
FSM diagram, $\rvx_i=a$ if $f_{\rvp}(\rvx_i)=a$. \end{mdframed}



$f_\calp =
\{f_\rvp: \rvp\in \calp\}$





$\rvp(0)\in \calp$, {\bf starting state (i.e., starting place)}



$\Phi=\av{\calp,\calp_{acc}, \calx, \Sigma, f_\calp, \rvp(0) }$ is a {\bf Finite Automaton (FA)}\footnote{Recall from Chapter \ref{ch-petri} that a Petri net is defined as $\Phi = \av{\calp, \calx, W,\ket{b(0)}_\calp}$.
}

\hrule
For a {\bf deterministic FA}, the functions
$f_\rvp$ for all $\rvp\in \calp$
must all be 1-1 and onto (i.e., bijections).
Furthermore, we must have 
$\emptyset\not\in \Sigma$. 

For a {\bf non-deterministic FA},  the functions
$f_\rvp$ need not be bijective.
Furthermore, we must have 
$\emptyset\in \Sigma$. If $f_\rvp(\rvx)=\text{empty set}=\emptyset$, we say there is a  {\bf null transition} for arrow 
$\rvp\rarrow \rvx$.\footnote{The empty set is sometimes denoted by $\eps$ 
when discussing FSM.}