\chapter{Finite State Machine}
\label{ch-finite-state}

Let $|S|$ denote the number of elements
in a set $|S|$.
The {\bf power set of a set $S$}, denoted by $2^S$,
is the set of all its subsets, including the empty set.
Note that $|2^S|=2^{|S|}$. Indeed
consider the binomial expansion

\beq
(1 + 1)^{|S|}=\sum_{i=0}^{|S|}{|S| \choose i}
\eeq
Now note that  ${|S|\choose i}$
is the number of subsets of $S$ with $i$ elements.

\section{Deterministic FSM}

\subsection{Example}
\subsection{Precise Definition}
{\bf Finite State Machine} (FSM) (a.k.a. {\bf Finite Automaton}, (FA))

$\calx =\{\rvx_i\}_{i=1}^{nx}$ are the {\bf transition nodes} of the pnet.

$\calp =\{\rvp_i\}_{i=1}^{np}$ are the {\bf place (or state) nodes} of the pnet.

$\calx(\rvp\rarrow)=\{\rvx: \rvp\rarrow\rvx \text{ is an arrow of the pnet}\}$.

\hrule
New terms and notation:

$\calp_{acc}\subset \calp$ are the 
{\bf accepting states}. States which, if reached, terminate a trajectory.

$\Sigma=$ {\bf alphabet}, a finite set of symbols
from which to create strings. For example,
$\Sigma$ might be $\{0, 1\}$ or $\{a, b, c\}$.

$f_\rvp:\calx(\rvp\rarrow)\rarrow \Sigma$ for each
$\rvp\in \calp$,  is the 
{\bf transition function} for place node $\rvp$. For each $\rvp\in \calp$,
 function
$f_\rvp$
must be 1-1. (i.e., 
$f_\rvp(\rvx_1)=f_\rvp(\rvx_2)=a\in\Sigma$
implies $\rvx_1=\rvx_2$)

$f_\calp =
\{f_\rvp: \rvp\in \calp\}$

$\rvp_{i(0)}\in \calp$, {\bf starting state (i.e., starting place)}
\hrule
Recall from Chapter \ref{ch-petri} that a Petri net is defined as $\Phi_{pnet} = \av{\calp, \calx, W,\ket{b(0)}_\calp}$.
A {\bf Finite Automaton (FA)}
is defined by adding extra stuff to
$\Phi_{pnet}$, namely by
$\Phi_{FA}=\av{\Phi_{pnet},\calp_{acc}, \Sigma, f_\calp}$.
$\ket{b(0)}_\calp$ in the definition
of $\Phi_{pnet}$ is replaced by
$\rvp_{i(0)}\in\calp$.

\begin{mdframed}[hidealllines=true,backgroundcolor=blue!10]
For a {\bf deterministic FA}:
\begin{itemize}
\item For each $\rvp\in\calp$, the function $f_\rvp:\calx(\rvp\rarrow)\rarrow \Sigma$ is onto (i.e.,
its image\footnote{Recall that the image
$Im(f)$ of a function
$f:A\rarrow B$ with range $B$
is the set $Im(f)=\{f(a): a\in A\}$,
and $f$ is onto if $Im(f)=B$.}

is the full alphabet)

\item
we will often indicate a node for 
which $f_{\rvp}(\rvx_i)=a$ either by $\rvx_i=a$
or simply by $a$. 
\end{itemize}\end{mdframed}

\section{Non-deterministic FSM}
\subsection{Simple Example}
\subsection{Precise Definition}

\begin{mdframed}[hidealllines=true,backgroundcolor=blue!10]
For a {\bf non-deterministic FA}:
\begin{itemize}
\item
$f_\rvp:\calx(\rvp\rarrow)\rarrow 2^\Sigma$ 
instead of 
$f_\rvp:\calx(\rvp\rarrow)\rarrow \Sigma$ (i.e., the range of $f_\rvp$
is the power set of $\Sigma$ 
instead of $\Sigma$).
The function $f_\rvp$ need not be onto.
Note that the new range contains the empty set.
If $f_\rvp(\rvx)=\emptyset$, we say there is a  {\bf null transition} for arrow 
$\rvp\rarrow \rvx$.\footnote{The empty set is sometimes denoted by $\eps$ 
when discussing FSM.} 
\item
we will often indicate a node for 
which $f_{\rvp}(\rvx_i)=\{a,b\}\subset\Sigma$ either by $\rvx_i=a, b$
or simply by $a, b$. 
\end{itemize}\end{mdframed}
