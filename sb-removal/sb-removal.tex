\chapter{Selection Bias Removal}
This chapter 
is based on Ref.\cite{bare-sb-removal}.

Selection bias (SB)
occurs when one 
samples from an
atypical subset
of a
population,
producing a biased dataset.
Are such biased 
datasets
useless? No. 
It is possible to 
add additional auxiliary features
to the biased dataset, and to 
sample those new features
in an unbiased way,
 from the whole population.
Then
one can merge
the original
 biased dataset with the
auxiliary, unbiased one,
to obtain an enhanced dataset.
It is sometimes
possible to do this so that the enhanced
dataset is provably 
unbiased.
The theory of Causal
Inference tell us 
WHEN this is possible,
and HOW to do it
when it is possible.

Consider the bnet 
Fig.\ref{fig-bs-removal-basic}.

\begin{figure}[h!]
$$
\xymatrix{
\rvs=1\ar[dr]|?\ar[d]_?\ar[r]^?
&\rvA\ar[d]
&
\\
\rvx\ar[r]\ar@/_.9pc/[ru]
&\rvy
}
$$
\caption{Bnet considered for 
selection bias (SB) removal.
Arrows with question marks
may or may not be present.}
\label{fig-bs-removal-basic}
\end{figure}

Let

$\rvs\in\bool$
is the {\bf selection node}.
$\rvs=1$ means there is 
SB
in the node.
$\rvs=0$ or no $\rvs$ parent
means there 
is no SB in the node.


$\rvx=$ {\bf class features}.\footnote{
A feature is the same as a node in a bnet.}

$\rvy=$ {\bf target feature}.

$\rvA=$ {\bf auxiliary features}
(Ref.\cite{bare-sb-removal} 
calls this $T-\{\rvx\}$) 

$\rvE=\{\rvy, \rvx\}\cup\rvA=$ 
{\bf Enhanced feature set}
(Ref.\cite{bare-sb-removal} calls this $M$) 

$\Sigma=$ population of individuals $\s$

$OD=\{(\s,\rvx^\s,  \rvy^\s,\rvs^\s=1):\s\in\Sigma\}=$ 
{\bf Original Dataset}, dataset for $(\rvx,\rvy)$ features
with $\rvs=1$. 
Gives empirical
distribution $\color{red}{P(y|x, \rvs=1)}$.
(Ref.\cite{bare-sb-removal} 
calls this dataset the {\bf biased study}.)

$AD=\{(\s, \rvx^\s,\rvA^\s):\s\in\Sigma\}=$ 
{\bf Auxiliary Dataset}, dataset for $(\rvx, \rvA)$ features.
Gives empirical
distribution $\color{red}{P(A|x)}$.
(Ref.\cite{bare-sb-removal} 
calls this dataset the 
{\bf population level study}.)


$ED=\{(\s,\rvx^\s, \rvA^\s, \rvy^\s,\rvs^\s=1):\s\in\Sigma\}=$ 
{\bf Enhanced Dataset}, dataset for $(\rvx,\rvy, \rvA)$ features
for $\rvs=1$.
Obtained by merging $OD$ and $AD$.
Gives empirical
distribution $\color{red}{P(y|x, A, \rvs=1)}$.
\section{Removing SB from 
passive query $P(y|x)$}

The {\bf  passive query} $P(y|x, \rvs=1)$
is {\bf SB-recoverable}
if it is independent of $\rvs$
for some bnet $G$.

\begin{enumerate}
\item {\bf No SB.}
{\bf Query $P(y|x)$ is
SB-recoverable}
with $\rvA= \emptyset$; SB can be removed
by conditioning on $\rvx$.

If $\rvy\perp\rvs|\rvx$, then
\beq
P(y|x, \rvs=1)=P(y|x)
\;.
\eeq
For example,
$\rvy\perp\rvs|\rvx$ in the following bnet:

\beq
\begin{array}{c}
\xymatrix{
\rvs\ar[d]
\\
*++[o][F*:yellow]{\rvx}\ar[r]&\rvy
}
\\
\rvy\perp\rvs|\rvx
\end{array}
\eeq



\item {\bf Query $P(y|x)$ is
SB-recoverable}
via $\rvxi$; SB can be removed
by conditioning on $\rvx$
and $\rvxi$.
Here $\rvxi$
is a nonempty
subset of $\rvA$

\begin{claim}\label{cl-sb-recov}
There exists $\rvxi\subset \rvA$
 such that
 $\rvy\perp\rvs|(\rvx,\rvxi)$
and $\rvxi\perp\rvs|\rvx$
iff
\beq
P(y|x, \rvs=1)
=
\sum_\xi 
\underbrace{P(y|x, \xi, \rvs=1)}_
{P(y|x,\xi)}
\underbrace{P(\xi|x, \rvs=1)}_{P(\xi|x)}
=P(y|x)
\eeq

\beq
\xymatrix{
\rvs=1\ar[rd]
\\
x\ar[r]
&y
}
\xymatrix{\\=}
\xymatrix{
&\sum \xi\ar[d]
\\
x\ar[r]\ar[ru]
&y}
\xymatrix{
\\=}
\xymatrix{\\
x\ar[r]&y}
\eeq
\end{claim}
\proof

The $\Rightarrow$ part of this 
claim is obvious. For a proof
of the $\Leftarrow$ part, see
 Ref.\cite{bare-sb-removal}.
\qed

The simplest example
of a bnet to which
Claim \ref{cl-sb-recov}
applies is this:

\beq
\xymatrix{
\rvs\ar[d]
&{\rvxi}\ar[d]
\\
{\rvx}\ar[r]
\ar[ru]
&\rvy
}
\eeq

As a second example
of Claim \ref{cl-sb-recov},
consider the following bnet:
\beq
\xymatrix{
\rvs\ar[d]\ar@/^1pc/[rr]
&{\rvz}\ar[d]\ar[r]
&{\rvw}
\\
{\rvx}\ar[r]
\ar[ru]
&\rvy
}
\eeq

$\rvy\perp\rvs|(\rvx,\rvxi)$
and $\rvxi\perp\rvs|\rvx$
with $\rvxi=\rvz$
for this bnet:

\beq
\begin{array}{cc}
\xymatrix{
\rvs\ar[d]\ar@/^1pc/[rr]
&*++[o][F*:yellow]{\rvz}\ar[d]\ar[r]
&{\rvw}
&
\\
*++[o][F*:yellow]{\rvx}\ar[r]
\ar[ru]
&\rvy
}
&
\xymatrix{
\rvs\ar[d]\ar@/^1pc/[rr]
&{\rvz}\ar[d]\ar[r]
&{\rvw}
&
\\
*++[o][F*:yellow]{\rvx}\ar[r]
\ar[ru]
&\rvy
}
\\
(a)\quad 
\rvy\perp\rvs|(\rvx, \rvz)
&(b)\quad
\rvz\perp\rvs|\rvx
\end{array}
\label{eq-x-z-w-recov}
\eeq
Therefore

\beqa
P(y|x, \rvs=1)
&=&
\xymatrix{
&\sum {z}\ar[d]\ar[r]
&\sum{w}
&
\\
{x}\ar[r]
\ar[ru]
&y
}
\\
\eeqa

\item {\bf Query $P(y|x)$ is not
SB-recoverable}; SB cannot be removed.

\beq
\xymatrix{
\rvs\ar[d]\ar[dr]
\\
\rvx\ar[r]&\rvy
}
\quad\quad
\xymatrix{
\rvs\ar[dr]
\\
\rvx\ar[r]&\rvy
}
\quad\quad
\xymatrix{
\rvs\ar[r]\ar[d]
&\rvz\ar[dl]
\\
\rvx\ar[r]
&\rvy\ar[u]
}
\eeq


\end{enumerate}

\section{Removing SB from active query 
$P(y|{\cal D}x)$}

The {\bf active
query (i.e., do query)}
$P(y|\cald\rvx=x, \rvs=1)$
is 
\begin{enumerate}[(a)]
\item {\bf SB-recoverable}
if it equals $P(y|\cald\rvx=x)$
for some 
bnet $G$,
\item
{\bf do-identifiable}
if it equals
$P(y|x, \rvs=1)$ 
for some bnet $G$,
\item
both
{\bf SB-recoverable
and do-identifiable}
if it equals 
$P(y|x)$
for the same bnet $G$.
\end{enumerate}

Examples
\begin{itemize}
\item

\beq
\xymatrix{
\rvs\ar[r]
&\rvxi\ar[d]
\\
\rvx\ar[r]\ar[ru]
&\rvy}
\quad
\begin{array}{l}
\text{SB-recoverable: NO}
\\
\text{do-identifiable: YES}
\end{array}
\label{eq-sb-removal-ex-1}
\eeq
For bnet Eq.(\ref{eq-sb-removal-ex-1}),
$P(y|\cald\rvx=x, \rvs=1)$
is do-identifiable
because there are no unobserved nodes.
It's not SB-recoverable because
to block information
from flowing from $\rvy$ to $\rvs$,
one must condition on $\rvxi$.
But if one conditions on $\rvxi$,
then  info flows from $\rvx$ to $\rvs$,
which is forbidden by Claim
\ref{cl-sb-recov}.
\item

\beq
\xymatrix{
\rvs\ar[d]
&*++[F-o]{\rvxi}\ar[d]\ar[ld]
\\
\rvx\ar[r]
&\rvy
}\quad
\begin{array}{l}
\text{SB-recoverable: YES}
\\
\text{do-identifiable: NO}
\end{array}
\label{eq-sb-removal-ex-2}
\eeq
\end{itemize}


Let

$V=$ set of nodes in graph

$V^{<\rvx}=$ non-descendants 
of $\rvx$ (excluding $\rvx$)

$V^{>\rvx}=$ descendants
of $\rvx$ (excluding $\rvx$)

$\rvz^{<\rvx} = \rvz\cap V^{<\rvx}$

$\rvz^{>\rvx} = \rvz\cap V^{>\rvx}$


Suppose $\rvz \cup\{\rvx,\rvy\} \subset E$
and $\rvz\subset \rvA$.
We say $\rvz$ satisfies the {\bf 
selection bias (SB) 
backdoor criterion} 
with respect to $(\rvx, \rvy)$
if

\begin{enumerate}
\item all backdoor
paths from $\rvx$ to
$\rvy$ are blocked by conditioning on $\rvz^{<\rvx}$
\item $\rvz^{>\rvx} \perp \rvy | (\rvx,\rvz^{<\rvx})$
\item $\rvy\perp\rvs|(\rvx,\rvz)$
\end{enumerate}

 \begin{claim}(SB Backdoor
 Adjustment Formula)

If $\rvz$ satisfies the SB backdoor
criterion relative to $(\rvx, \rvy)$,
then

\beq
P(y|\cald\rvx=x, \rvs=1)
=
\sum_z P(y|x,z)P(z)=P(y|x)
\eeq

\beq
\xymatrix{
\rvs=1
\ar[rd]
\\
\cald\rvx=x\ar[r]
&y
}
\xymatrix{\\=}
\xymatrix{
&\sum z \ar[d]
\\
x\ar[r]
&y}
\\
\xymatrix{\\=}
\xymatrix{
\\
x\ar[r]&y
}
\eeq

\end{claim}
\proof

If
$z$
satisfies the
SB backdoor
criterion
relative
to
$(\rvx, \rvy)$,
then
$\rvx, \rvy, \rvz$
might
have the following 
structure.


\beq
\xymatrix{
\rvs\ar[d]\ar[r]\ar@/^1pc/[rr]
&\rvz^{<\rvx}\ar[ld]\ar[d]\ar[r]
&\rvz^{>\rvx}
\\
\rvx\ar[rru]\ar[r]
&\rvy
}
\label{eq-sb-bdoor-special}
\eeq

See Claim \ref{cl-decSBBackDoor}
for a proof of this claim
for the
special case Eq.(\ref{eq-sb-bdoor-special}).
\qed


