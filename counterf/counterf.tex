\chapter{Counterfactual Reasoning:
 COMING SOON}

This chapter 
assumes that the reader has read 
Chapter \ref{ch-linear-sys}  
on linear 
deterministic systems
with exogenous noise.

Let
us repeat
Eq.(\ref{eq-mat-fully-conn})
from Chapter \ref{ch-linear-sys}.


\beq
\rvx = A \rvx+\rvu
\label{eq-mat-fully-conn-1}
\eeq
This equation
represents the 
structural equations
for a 
fully connected PA diagram.

Actually, 
we want to consider 
$nsam$ copies of equation 
Eq.(\ref{eq-mat-fully-conn-1})

\beq
\rvx[s] = A \rvx[s]+\rvu[s]
\label{eq-mat-fully-conn-nsam}
\;,
\eeq
where each $s=0, 1, \ldots, nsam-1$.
Each $s$ represents a separate
individual or ``unit"
in a population.



\hrule
\begin{enumerate}
\item Solve
 Eqs.\ref{eq-mat-fully-conn-nsam} 
for $\rvu$.

\beq
\rvu(\rvx,A)=(1-A)\rvx
\eeq
\item
Modify Eqs.\ref{eq-mat-fully-conn-nsam}
by replacing equation 
for $\rvx_{i^*}$ by

\beq
\rvx_{i^*}=a
\eeq
for some $a\in S_{\rvx_{i^*}}$. 
Modify the PA diagram
correspondingly
by deleting
all arrows 
entering
node $\rvx_{i^*}$.
This is an
intervention
similar to
those done in do-calculus.

Define $A^*$ by

\beq
A^*_{r,c}=\left\{
\begin{array}{ll}
A_{r,c}&\text{if row $r\neq i^*$}
\\
0& \text{if row $r=i^*$}
\\
\end{array}\right.
\eeq


Define $\rvu^*$ by
\beq
\rvu^*_{r}=\left\{
\begin{array}{ll}
\rvu_{r}(\rvx, A)&\text{if row $r\neq i^*$}
\\
a& \text{if row $r=i^*$}
\\
\end{array}\right.
\eeq

Define new random variables $\rvx^*$
that satisfy

\beq
\rvx^* = A^* \rvx^* +\rvu^*
\;.
\eeq
Thus,

\beq
\rvx^*=
(1-A^*)^{-1}\rvu^*
\eeq
\end{enumerate}
\hrule
\beq\color{blue}
P(x^*_i|x^*_{<i})=
\indi(
x^*_i=\sum_{k<i}\alp^*_{i|k}x^*_k
 + u^*_i)
\;.
\eeq

\beqa
P(x^*.)=
\prod_i P(x^*_i|x^*_{<i})
\eeqa

