\chapter{Transformer Networks}
\label{ch-transformer}

This chapter in based on Refs.\cite{joshi-trans}
and \cite{wiki-transformer}.

Transformer Networks (TN)
have been taking the field of
Natural Language Processing (NLP)
by storm in recent years.
They were introduced in 2017 and already
are the basis of
BERT (Bidirectional Encoder
Representations from Transformers)
and GPT (Generative Pre-trained Transformer),
two TN libraries that
have been trained with
huge databases such as all of
English Wikipedia (2,500M words).

Recurrent Neural Nets (RNNs)
are discussed in Chapter \ref{ch-rnn}.
TNs are quickly displacing RNNs, 
an older method, in NLP.  TNs are better than RNNs 
for doing NLP in several important ways. Whereas
 RNNs analyze the tokens (words) of a sentence 
sequentially (like a Kalman Filter), 
TNs analyze them in parallel, and thus are more
 amenable to parallel computing. Also, because
 RNNs analyze the words of a sentence sequentially, 
they tend to give more importance to the end 
of a sentence than to its beginning. That's because 
RNNs start forgetting the beginning of a sentence
 by the time they reach its end, like a patient 
with Alzheimer's. TNs do not suffer from this malady.

Dynamical bnets are discussed in Chapter \ref{ch-dyn-bnet}.
In Chapter \ref{ch-rnn},
we showed that RNNs
are dynamical bnets.
The goal of
this chapter
is to define TNs,
and to show that they too are
dynamical bnets.

Let

$\cals$ be the
set of words in a sentence,

$h^t_i\in \RR^{nh}$ be
an $nh$ dimensional column vector
for word $i\in \cals$ at time $t$.

$Q^t, K^t, V^t\in \RR^{nh\times nh}$
be the  {\bf prior-attention matrices for time
slice $t$}.
These matrices are learned by training
the net.
The letters $Q,K,V$ stand for
 Query, Key and Value,
respectively.


\begin{figure}[h!]
$$
\xymatrix{
&\rvV^t\ar[d]
&\rvQ^t\ar[d]
&\rvK^t\ar[d]
\\
&\{\rvv^t_i\}_{i=0,1,2}
&\{\rvq^t_i\}_{i=0,1,2}
&\{\rvk^t_i\}_{i=0,1,2}
\\
&\rvv_0^t\ar[rrd]
\ar[rrdddd]\ar[rrddddddd]
\\
\rvh_0^t \ar[ru]\ar[r]\ar[rd]
&\rvq_0^t\ar[rr]
&&\rvh_0^{t+1}
\\
&\rvk^t_0\ar[rru]
\ar[rrdd]\ar[rrddddd]
\\
%%%%%
&\rvv_1^t\ar[rrd]
\ar[rrdddd]\ar[rruu]
\\
\rvh_1^t\ar[ru]\ar[r]\ar[rd]
&\rvq_1^t\ar[rr]
&&\rvh_1^{t+1}
\\
&\rvk^t_1\ar[rru]
\ar[rrdd]\ar[rruuuu]
\\
%%%%%
&\rvv_2^t\ar[rrd]
\ar[rruu]\ar[rruuuuu]
\\
\rvh_2^t \ar[ru]\ar[r]\ar[rd]
&\rvq_2^t\ar[rr]
&&\rvh_2^{t+1}
\\
&\rvk^t_2\ar[rru]
\ar[rruuuu]\ar[rruuuuuuu]
}
$$
\caption{Time slice $t$
of dynamical bnet for
 a transformer network (TN)
of a 3-word sentence.
The following arrows were not
drawn
explicitly for clarity:
arrows pointing from node
$\rvV^t$ to nodes $\rvv^t_i$
for $i=0,1,2$,
from node
$\rvQ^t$ to nodes $\rvq^t_i$
for $i=0,1,2$,
and from node
$\rvK^t$ to nodes $\rvk^t_i$
for $i=0,1,2$.
Note that $k^t_i$
for all $i$
points to $\rvh^{t+1}_j$ for all $j$.
Likewise,
$\rvv^t_i$
for all $i$
points to $\rvh^{t+1}_j$ for all $j$.
}
\label{fig-transformer}
\end{figure}

Fig.\ref{fig-transformer}
represents a TN 
of a 3-word sentence as a dynamical bnet.
The TPMs,
printed in blue,
for bnet
Fig.\ref{fig-transformer},
are as follows:

\begin{subequations}
\label{eq-vqk-priors}
\beq\color{blue}
P(V^t)=\delta(V^t, V^t_0)
\eeq

\beq\color{blue}
P(Q^t)=\delta(Q^t, Q^t_0)
\eeq

\beq\color{blue}
P(K^t)=\delta(K^t, K^t_0)
\eeq
\end{subequations}

\beq\color{blue}
P(v^t_i|V^t, h^t_i)=
\indi(\;\;\;
v^t = V^t h^t_i
\;\;\;)
\eeq

\beq\color{blue}
P(q^t_i|Q^t, h^t_i)=
\indi(\;\;\;
q^t = Q^t h^t_i
\;\;\;)
\eeq

\beq\color{blue}
P(k^t_i|K^t, h^t_i)=
\indi(\;\;\;
k^t= K^t h^t_i
\;\;\;)
\eeq

\beq\color{blue}
P(h^{t+1}_i|v^t_.,q^t_i,
 k^t_.)
=
\indi(\;\;\;
h_i^{t+1}=
\underbrace{
\sum_{j\in\cals}
v^t_{j}
\underbrace{
\frac{e^{(q^t_i)^T k^t_j}}
{\sum_{j'\in \cals}
e^{(q^t_i)^Tk^t_{j'}}}
}_{w_{j|i}=
[\softmax((q_i^t)^T k^t_j)]_j}\;
}_{E_{\rvj|i}[v^t_\rvj]={\bf Attention}}
\;\;\;)
\eeq

In Eqs.\ref{eq-vqk-priors}, 
the 3 root nodes $\rvV^t, \rvQ^t, \rvK^t$ 
have delta function priors.
To improve stability
of the dynamical bnet,
it is more common instead to use
for these priors,
multiple delta functions 
(a.k.a. {\bf multi-heads})
labeled $c=0,1, \ldots, nc-1$, as follows.

\begin{subequations}
\beq\color{blue}
P(V^t|c)=
\delta(V^t, V^t_c)
\eeq

\beq\color{blue}
P(Q^t|c)=
\delta(Q^t, Q^t_c)
\eeq

\beq\color{blue}
P(K^t|c)=
\delta(K^t, K^t_c)
\eeq
\end{subequations}
for $c=0,1, \ldots, nc-1$.
Even better,
one can go fully
Bayesian, 
and make
these 3 prior distributions 
arbitrary categorical
distributions  
to be determined by net training,
instead of mere
delta functions.