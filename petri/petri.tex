\chapter{Petri Nets COMING SOON}
\label{petri}

\newcommand{\rpetriar}[1]{\ar@{.>}@/_1pc/[r]|*++[o][F-]{#1}}

\newcommand{\lpetriar}[1]{\ar@{.>}@/_1pc/[l]|*++[o][F-]{#1}}

\xymatrix@C=5pc{
\rvx\rpetriar{8}
\ar[r]
&\rvy\lpetriar{9}
}


\section{Conventional Petri Nets}
\subsection{Simple Example of Petri Net}

\beq
\xymatrix{
&&*++[o][F-]{0}\ar[rd]
\\
*++[o][F-]{3}\ar[r]_<<<{\rvp_1}
&\rvx_1\ar[ur]_>{\rvp_2}
\ar[dr]
&&\rvx_2\ar[r]_>>{\rvp_4}
\ar@/_2pc/[lll]
&*++[o][F-]{1}
\\
&&*++[o][F-]{2}\ar[ur]_<{\rvp_3}
}
\eeq

\beq
\xymatrix{
&&*++[F-]{\phantom{0}}\ar[rd]
\\
*++[F-]{\bullet\bullet\bullet}\ar[r]_<<<{\rvp_1}
&\rvx_1\ar[ur]_>{\rvp_2}
\ar[dr]
&&\rvx_2\ar[r]_>>{\rvp_4}
\ar@/_2pc/[lll]
&*++[F-]{\bullet}
\\
&&*++[F-]{\bullet\bullet}\ar[ur]_<<{\rvp_3}
}
\eeq

\beq
W^{\rvx\larrow}=
\begin{array}{c|cc}
&\rvx_1&\rvx_2
\\ \hline
\rvp_1&1&0
\\
\rvp_2&0&1
\\
\rvp_3&0&1
\\
\rvp_4&0&0
\end{array}
,\;
W^{\rvx\rarrow}=
\begin{array}{c|cc}
&\rvx_1&\rvx_2
\\ \hline
\rvp_1&0&1
\\
\rvp_2&1&0
\\
\rvp_3&1&0
\\
\rvp_4&0&1
\end{array}
,\;
W=
W^{\rvx\rarrow}
-W^{\rvx\larrow}=
\begin{array}{c|cc}
&\rvx_1&\rvx_2
\\ \hline
\rvp_1&-1&1
\\
\rvp_2&1&-1
\\
\rvp_3&1&-1
\\
\rvp_4&0&1
\end{array}
\eeq


%0
%p2
%3  p1
%/x1
%@
%x2 p4
%t / 1
%2 p3
%@
%(74.1)
%p2
%• • •   p1
%/x1
%>
%
%x2 p4
%s / •
%•• p3
%>
%(74.2)
%Wx← =
%x1 x2
%p1 1 0
%p2 0 1
%p3 0 1
%p4 0 0
%, Wx→ =
%x1 x2
%p1 0 1
%p2 1 0
%p3 1 0
%p4 0 1
%, W = Wx→ −Wx← =
%x1 x2
%p1 −1 1
%p2 1 −1
%p3 1 −1
%p4 0 1
%(74.3)
%553
%If
%p =
%p1
%p2
%p3
%p4
%, x =
%"
%x1
%x2
%#
%(74.4)
%then
%p(t) = p(0) +Wx =
%p1(0) − x1 + x2
%p2(0) + x1 − x2
%p3(0) + x1 − x2
%p4(0) + x2
%(74.5)
%p2
%
%p3
%x ~ 2
%p ~   1 p4
%p1 p2 p3 p4
%x2 1 −1 −1 1
%(74.6)
\subsection{Precise Definition of Petri Net}
%Let
%Z>0 = {1, 2, 3 . . .} natural numbers.
%Z≥0 = {0, 1, 2, 3 . . .} natural numbers and zero.
%pi = ith place (a.k.a. buffer, token holder) node. This node holds either
%an inxeger ≥ 0 or that number of tokens (tokens are represenxed by bullet poinxs).
%xi = ith transition node
%P = {pi}np
%i=1 set of all places
%X = {xi}nx
%i=1 set of all transitions
%x → y an arrow (a.k.a. directed arc). Note that there are two type of arrows:
%those that go from an transition to a place node and those which go from a place to
%a transition node. Let x → y = y ← x = (x, y)
%A = the set of all arrows. A ⊂ A = (P × X) ∪ (X × P) = {p → x or x → p :
%p ∈ P, x ∈ X}.
%A flow is a non-directed path of arrows from A.
%μ : A → Z>0 = multiplicity of each arrow. If unstated, assume μ(a) = 1
%for all a ∈ A. It’s also possible to extend the domain of μ() to A and define μ(a) = 0
%for all a ∈ A − A.
%p ∈ Znp×1
%≥0 markings column vector. pi is the number of tokens in place pi.
%We will use M : P → Z≥0 with M(pi) = pi.
%554
%t ∈ Znx×1
%>0 . (column vector). How many times each transition occurs
%p(0) = initial markings
%Wx← ∈ {0, 1}np×nx. Wx←
%i,j = 1 if there is an arrow pi
%→ xj (enxering xj);
%else Wx←
%i,j = 0. More generally, in the case a multiplicity function μ() is given, let
%Wx←
%i,j = μ(pi
%→ xj).
%Wx→ ∈ {0, 1}np×nx. Wx→
%i,j = +1 if there is an arrow pi
%← xj (leaving xj);
%else Wx→
%i,j = 0. More generally, in the case a multiplicity function μ() is given, let
%Wx→
%i,j = μ(pi
%← xj).
%W = Wx→ − Wx← ∈ {−1, 0, 1}np×nx This is called the incidence matrix.
%More generally, if a multiplicity function is given, W = Wx→ −Wx← ∈ Z.
%{◦ → x} = {p : p → x ∈ A}: input places (a.k.a. places pre-set)
%{x → ◦} = {p : x → p ∈ A}: output places (a.k.a. places post-set)
%{◦ → p} = {x : x → p ∈ A}: input transitions (a.k.a. transitions
%preset)
%{p → ◦} = {x : p → x ∈ A}: output transitions (a.k.a. transitions
%post-set)
%Φ = ⟨P,X,W, p(0)⟩ is a Petri net.
\subsection{Execution of a Petri net}
%Consider a Petri Net Φ = ⟨P,X,W, p(0)⟩.
%By firing step of a transition x, we mean the action of removing μ(p → x)
%tokens from all p such that p → x ∈ {◦ → x}, and adding μ(x → p′) tokens to all p′
%such that x → p′ ∈ {x → ◦}.
%A transition x is enabled, i.e., it may fire if there are enough tokens in
%the input places of x for a firing to be possible; i.e., if M(p) ≥ μ(p → x) for all p.
%t ∈ Z>0 is the firing time.
%p(t) = markings at firing time t.
%x(t) = how many times each transition occurs at time t.
%p(t) = p(0) +Wx(t) (74.7)
%p(0), p(1), p(2), . . . is the sequence of firings
%A marking p∗ is reachable from a marking p in n = 1, 2, . . . steps if
%p can be transformed to p∗ by executing n firings. We write p Φ / p∗ and say a
%marking p∗ is reachable from a marking p if p∗ is reachable from p in a finite
%number of steps.
%The reachability set of a Petri Net Φ with initial markings p(0) is defined as
%R(Φ) = {p∗ : p(0) Φ / p∗ } (74.8)
%555
\section{Generalizations}
\subsection{Continuous Petri net}
%Replace pi → dxi
%dt = x˙i(t) for all i = 1, 2, . . . , np.
%x˙(t) = x˙(0) +Wx(t) (74.9)
\subsection{Colored Petri Nets}
\subsection{Stochastic Petri Nets}
\subsection{Petri-Bayes Net}
%Petrifying a Bnet
%x / y =⇒ x /
%3
%&
%f 2
%y (74.10)
%We will refer to the place between x and y as px,y.
%a
%b 􀀀 /
%c
%􀀀
%d
%=⇒
%a
%􀀀
%3
%v
%6
%5
%V
%b /
%2
%j *
%1
%U
%c
%􀀀
%5
%v
%6
%d
%(74.11)
{\bf Petri-Bayes Net}
%a
%3
%U
%b
%􀀀
%2
%u
%5
%x
%3
%U
%􀀀
%4
%u
%5
%c d
%pa→x px→c px→d
%· · ·
%x −1 1 1 0
%(74.12)
%556
%a
%3
%U
%b
%􀀀
%2
%v
%5
%x
%3
%U
%􀀀
%4
%u
%6
%c d
%pa→x px→b
%· · ·
%x −1 1 0
%a
%3
%U
%b
%
%2
%u
%5
%x
%3
%U
%􀀀
%4
%u
%5
%c y d
%pa→x px→b
%· · ·
%x −1 1 0
%(74.13)
%a
%3
%U
%b
%􀀀
%2
%u
%5
%x
%3
%U
%􀀀
%4
%u
%5
%c d
%pc→x px→a px→b px→d
%· · ·
%x −1 1 1 1 0
%a
%3
%U
%b
%􀀀
%2
%v
%5
%x
%3
%U
%􀀀
%4
%u
%6
%c d
%pc→x
%· · ·
%x −1 0
%(74.14)
%557
%
