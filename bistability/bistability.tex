\chapter{Bistability in 2D and 3D systems: COMING SOON}
\label{ch-bistability}

Bistability and hysteresis are common phenomena in nonlinear dynamical systems. Below, we'll give examples of systems with 2 or 3 dofs 
(degrees of freedom) with these properties.


\section{Lorentz System}
A classic example of a 3D system with bistability and hysteresis is the {\bf Lorenz system} (in certain parameter regimes). It is given by:

\beq
\xymatrix{
\rvy\ar[d]\ar[r]\ar@{-->}[drr]
&\bigotimes\ar[drrr]
&\rvx\ar[l]\ar[r]\ar[d]\ar@{-->}[dll]
&\bigotimes\ar[dlll]
&\rvz\ar[d]\ar[l]
\\
\dot{\rvy}
&&\dot{\rvx}
&&\dot{\rvz}
}
\eeq

\beq
\left\{
\begin{array}{l}
\dot{x} = \sigma (y - x)
\\
\dot{y} = x(\rho - z) - y
\\ 
\dot{z} = xy - \beta z
\end{array}
\right.
\eeq

where $\sigma > 0$, $\rho > 0$, and $\beta > 0$ are parameters.

For certain parameter values (e.g., $\sigma = 10, \beta = 8/3, \rho \in [24, 30]$), the Lorenz system exhibits bistability and hysteresis. In this regime, the system has two coexisting attractors (such as chaotic attractors or fixed points) and can jump between them depending on initial conditions and parameter sweeps. This transition can exhibit hysteresis as $\rho$ is increased or decreased.



\section{FitzHugh-Nagumo model}
A classic 2D example is the {\bf FitzHugh-Nagumo model}, a reduced form of the Hodgkin-Huxley model for neuronal dynamics:

\beq
\xymatrix{
\rvx^3\ar[dr]
&\rvx\ar[d] \ar[dr]\ar[l]
& \rvy\ar[d]\ar[dl]
\\
1\ar[r]
&\dot{\rvx}
&\dot{\rvy}
}
\eeq

\beq
\left\{
\begin{array}{l}
\dot{x} = x - x^3 - y + I
\\
\dot{y} = \epsilon (x - \gamma y)
\end{array}
\right.
\eeq

where $x$ represents a fast variable (e.g., membrane potential), $y$ represents a slow recovery variable, $I$ is an external input, and $\epsilon > 0$, $\gamma > 0$ are parameters.

- For appropriate parameter values, the FitzHugh-Nagumo model exhibits bistability (e.g., a stable fixed point and a limit cycle).

- Hysteresis arises as the input $I$ is slowly increased or decreased, causing the system to switch between the two attractors (resting state or oscillations) with a delay.


\section{Genetic Toggle Switch}
A biochemical system that exhibits bistability and hysteresis is the {\bf genetic toggle switch} (e.g., two genes repressing each other's expression):

\beq
\xymatrix{
\rvx\ar[dr]\ar[d]
&\rvy\ar[dr]\ar[d]
&\rvz\ar[dll]\ar[d]
\\
\dot{\rvx}
&\dot{\rvy}
&\dot{\rvz}
}
\eeq

\beq
\left\{
\begin{array}{l}
\dot{x} = \frac{\alpha_1}{1 + z^n} - x
\\
 \dot{y} = \frac{\alpha_2}{1 + x^m} - y
 \\ 
 \dot{z} = \frac{\alpha_3}{1 + y^p} - z,
\end{array}
\right.
\eeq

where $x$, $y$, and $z$ represent concentrations of proteins, $\alpha_1, \alpha_2, \alpha_3 > 0$ are production rates, and $n, m, p > 1$ control the steepness of the repression.

- The system can exhibit two stable steady states for specific parameter values.

- Hysteresis can occur as parameters like $\alpha_1$, $\alpha_2$, or $\alpha_3$ are varied.
