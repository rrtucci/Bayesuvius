\chapter{Binary Decision Diagrams}\label{ch-binarydd}

\begin{figure}[h!]
$$
\begin{array}{c}
\xymatrix@C=.1pc{
&&&&&&&\stackrel{\rvA}{x_1?}\ar@{-->}[dllll]
\ar[drrrr]
\\
&&&\stackrel{\rvA_0}{x_2?}\ar@{-->}[dll]\ar[drr]
&&&&&&&&\stackrel{\rvA_1} {x_2?}\ar@{-->}[dll]\ar[drr]
\\
&\stackrel{\rvA_{00}}{ x_3?}\ar@{-->}[dl]\ar[dr]
&&&&\stackrel{\rvA_{01}} {x_3?}\ar@{-->}[dl]\ar[dr]
&&&&\stackrel{\rvA_{10}}{ x_3?}\ar@{-->}[dl]\ar[dr]
&&&&\stackrel{\rvA_{11}}{x_3?}\ar@{-->}[dl]\ar[dr]
\\
\rvA_{000}\ar[d]
&&\rvA_{001}\ar[d]
&&\rvA_{010}\ar[d]
&&\rvA_{011}\ar[d]
&&\rvA_{100}\ar[d]
&&\rvA_{101}\ar[d]
&&\rvA_{110}\ar[d]
&&\rvA_{111}\ar[d]
\\
\Rect{1}
&&\Rect{0}
&&\Rect{0}
&&\Rect{1}
&&\Rect{0}
&&\Rect{0}
&&\Rect{1}
&&\Rect{1}
}
\\
\begin{array}{ccc|l}
x_1&x_2&x_3&f(A_{x_1,x_2,x_2})
\\ \hline\hline
0
&0
&0
&1
\\ \hline
0
&0
&1
&0
\\ \hline
0
&1
&0
&0
\\ \hline
0
&1
&1
&1
\\ \hline
1
&0
&0
&0
\\ \hline
1
&0
&1
&0
\\ \hline
1
&1
&0
&1
\\ \hline
1
&1
&1
&1
\\ \hline
\end{array}
\end{array}
$$
\caption{Example of a binary tree and its truth table.
The truth table gives the values of $
f(x_1, x_2,x_3)=
\bar{x}_1(x_2+\bar{x}_3)  + x_1 x_2 $ }
\label{fig-bdd-tree}
\end{figure}

This chapter is based
on Wikipedia article Ref.\cite{wiki-bdd}
and other references.

A {\bf Binary Decision Diagram} (BDD) is a 
graph which represents 
in a more concise manner
the information in
a binary tree and
its accompanying truth table.

Fig.\ref{fig-bdd-tree} shows an
example of a binary tree and its 
truth table.
We will  convert this tree 
 into a BDD in this chapter. Each node asks a question with a binary (Boolean) answer. An answer of 0 (resp., 1 ) is indicated by a dashed (resp., full) arrow. The same question is asked by all nodes at the same level (i.e., depth) of the tree. In addition to the question, nodes are labeled uniquely by the random variable $\rvA_{x_1, x_2, \ldots, x_n}$,
where $n$ is the level of the node
and $x_i$ is the answer to the question $x_i?$.
The leaves of
the tree are square boxes that report 

\beq
f(x_1, x_2,x_3)=
\bar{x}_1(x_2+\bar{x}_3)  + x_1 x_2
\label{eq-bdd-truth-table}
\eeq
\begin{figure}[h!]
$$
\xymatrix@C=.1pc{
&&&&&&&x_1?\ar@{-->}[dllll]
\ar[drrrr]
\\
&&&x_2?\ar@{-->}[dll]\ar[drr]
&&&&&&&&x_2?\ar@{-->}[dll]\ar[drr]
\\
&x_3?\ar@{-->}[dl]\ar[dr]
&&&&x_3?\ar@{-->}[dl]\ar[dr]
&&&&x_3?\ar@{-->}[dl]\ar[dr]
&&&&x_3?\ar@{-->}[dl]\ar[dr]
\\
\Rect{1}
&&\Rect{0}
&&\Rect{0}
&&\Rect{1}
&&\Rect{0}
&&\Rect{0}
&&\Rect{1}
&&\Rect{1}
}
$$
\caption{The same tree as in Fig.\ref{fig-bdd-tree}, after dropping 
some labels that are not needed
for our discussion of BDDs.}
\label{fig-bdd-tree-simp}
\end{figure}


Fig.\ref{fig-bdd-tree-simp} shows
the same tree as in  Fig.\ref{fig-bdd-tree}, after dropping 
some labels that are not needed
for our discussion of BDDs.
Note that the {\bf question ordering} of the tree is
$x_1<x_2<x_3$. Other question
orderings such as $x_2< x_1 < x_3$
are possible. For a given
truth table, some question 
orderings lead to a BDD 
that has the full  
exponential complexity $2^n$
of the tree, where $n$ is 
the number of levels. Other question orderings
might lead to BDDs that have lower (such as linear in $n$) 
complexity.



\subsection{Conversion of Binary Tree into BDD}

A BDD is obtained
from a binary tree by 
successive application of the
following
3 rules:

\begin{enumerate}
\item Merge equivalent leaves (EL)
\item Merge isomorphic nodes (IN)
\item Eliminate parallel 0/1 arrows (PA) by merging source
and target nodes of the parallel  arrows.
\end{enumerate}

Fig.\ref{fig-el-pa-example} 
gives an example
of the application of rules EL and PA.
Fig.\ref{fig-example-in-rule}
gives an example of the 
application of the IN rule.


\begin{figure}[h!]
$$
\xymatrix{
\\\\
\stackrel{EL}{\implies}}
\xymatrix@C=.1pc{
&&&&&&&x_1?\ar@{-->}[dllll]
\ar[drrrr]
\\
&&&x_2?\ar@{-->}[dll]\ar[drr]
&&&&&&&&x_2?\ar@{-->}[dll]\ar[drr]
\\
&x_3?\ar@{-->}[drrrrrrr]\ar[drrrrr]
&&&&x_3?\ar@{-->}[dr]\ar[drrr]
&&&&x_3?\ar@{-->}[dlll]\ar@/_1pc/[dlll]
&&&&x_3?\ar@{-->}[dlllll]\ar@/_1pc/[dlllll]
\\
&&
&&
&&\Rect{0}
&&\Rect{1}
&&
&&
&&
}
\xymatrix{
\\\\\stackrel{PA}{\implies}}
\xymatrix@C=.1pc{
&&&&&&x_1?\ar@{-->}[dllll]
\ar[d]
\\
&&x_2?\ar@{-->}[dll]\ar[drr]
&&&&x_2?\ar@{-->}[ddll]\ar[dd]
\\
x_3?\ar@{-->}[drrrrrr]\ar[drrrr]
&&&&x_3?\ar@{-->}[d]\ar[drr]
\\
&&
&&\Rect{0}
&&\Rect{1}
&&
&&
&&
}
$$
\caption{This is the 
result of applying the
the EL rule, followed by the PA
rule, to Fig.\ref{fig-bdd-tree-simp}.}
\label{fig-el-pa-example}
\end{figure}

\begin{figure}[h!]
$$
\begin{array}{c}
\xymatrix@C=.1pc{
&&&&&&x_1?\ar@{-->}[dllll]
\ar[drrr]
\\
&&x_2?\ar@{-->}[dll]\ar[drr]
&&&&&&&x_2?\ar@{-->}[ddlllll]\ar[ddlll]
\\
x_3?\ar[drrrrrr]\ar@{-->}[drrrr]
&&&&x_3?\ar@{-->}[d]\ar[drr]
\\
&&
&&\Rect{0}
&&\Rect{1}
&&
&&
&&
}
\xymatrix{\\\\
\stackrel{IN}{\implies}}
\xymatrix@C=.1pc{
&&&&&&x_1?\ar@{-->}[dllll]
\ar[drrr]
\\
&&x_2?\ar@{-->}[dll]\ar@/_1pc/[dll]
&&&&&&&x_2?\ar@{-->}[ddlllll]\ar[ddlll]
\\
x_3?\ar[drrrrrr]\ar@{-->}[drrrr]
&&&&
\\
&&
&&\Rect{0}
&&\Rect{1}
&&
&&
&&
}
\\
\xymatrix{\\\\\stackrel{PA}{\implies}}
\xymatrix@C=.1pc{
&x_1?\ar@{-->}[dl]\ar[dr]
\\
x_2?\ar@{-->}[d]\ar[drr]
&&x_2?\ar[d]\ar@{-->}[dll]
\\
\Rect{0}
&&\Rect{1}
}
\xymatrix{\\\\\stackrel{IN}{\implies}}
\xymatrix@C=.1pc{x_1?\ar@{-->}[d]\ar@/_1pc/[d]
\\
x_2?\ar@{-->}[d]\ar[dr]
\\
\Rect{0}
&\Rect{1}
}
\xymatrix{\\\\\stackrel{PA}{\implies}}
\xymatrix@C=.1pc{
\\
x_1?\ar@{-->}[d]\ar[dr]
\\
\Rect{0}
&\Rect{1}
}\end{array}
$$
\caption{An example 
to illustrate the application of the
IN rule}
\label{fig-example-in-rule}
\end{figure}





\section{Equivalent Bnet}

\begin{figure}[h!]
$$
\begin{array}{cc}
\xymatrix@C=.1pc{
&&&&&&&x_1?\ar@{-->}[dllll]
\ar[dr]
\\
&&&x_2?\ar@{-->}[dll]\ar[drr]
&&&&&x_2?\ar@{-->}[ddll]\ar[dd]
\\
&x_3?\ar@{-->}[drrrrrrr]\ar[drrrrr]
&&&&x_3?\ar@{-->}[dr]\ar[drrr]
\\
&&
&&
&&\Rect{0}
&&\Rect{1}
&&
&&
&&
}
&
\xymatrix@C=.1pc{
&&&&&&&\Circle{\rva, x_1?}\ar@{-->}[dllll]|{\ol{x_1}}
\ar[dr]|{x_1}
\\
&&&\Circle{\rvb, x_2?}\ar@{-->}[dll]|{\ol{x_2}}\ar[drr]|{x_2}
&&&&&\Circle{\rvc, x_2?}\ar[dd]|{x_2}
\\
&\Circle{\rvd, x_3?}\ar@{-->}[drrrrrrr]|{\ol{x_3}}
&&&&\Circle{\rve, x_3?}\ar[drrr]|{x_3}
\\
&&
&&
&&
&&\Rect{\ul{y}}
&&
&&
&&
}
\\
(a) & (b)
\end{array}
$$
\caption{
$(a)$ 
is the BDD 
at the end of the
conversion chain in  Fig.\ref{fig-el-pa-example}.
$(b)$ 
is a LD (Linear Deterministic) bnet that
is equivalent to the BDD of
$(a)$. $(b)$ is
obtained by keeping a subset of the arrows in $(a)$. The remaining
arrows are given gain $x_i$
(resp., $\bar{x}_i$) if they are full 
(resp., dashed) and originate from
a node with question $x_i?$.
}
\label{fig-bdd-to-bnet}
\end{figure}




In Fig.\ref{fig-bdd-to-bnet}, $(a)$
is a BDD and 
$(b)$ is a LD (Linear Deterministic) bnet equivalent to $(a)$.
The structural equations of the LD bnet, printed in blue,
are as follows:

\beq\color{blue}
\rvy=  x_2 \rvc + x_3 \rve + \bar{x}_3\rvd
\eeq

\beq\color{blue}
\rvd=\bar{x}_2\rvb
\eeq

\beq\color{blue}
\rve=x_2\rvb
\eeq

\beq\color{blue}
\rvb = \bar{x}_1\rva
\eeq

\beq\color{blue}
\rvc=x_1\rva
\eeq
Therefore

\beq
\rvy = [\bar{x}_1(x_2 x_3 +\bar{x}_2\bar{x}_3)
+ x_1 x_2
]\rva
\label{eq-3-histories-coeff}
\eeq

The equivalence of $(a)$ and $(b)$
follows from the following transformation
of Eq.(\ref{eq-bdd-truth-table})

\begin{align}
f(x_1, x_2,x_3) =&
\bar{x}_1(x_2+\bar{x}_3)  + x_1 x_2
\\
=&
\bar{x}_1[x_2(x_3+\bar{x}_3)+\bar{x}_3
(x_2+ \bar{x}_2)]  + x_1 x_2
\;\text{(because $x+\bar{x}=1$ in base 2)}
\\
=&
\bar{x}_1[x_2(x_3)+\bar{x}_3
(\bar{x}_2)]  + x_1 x_2\;, \text{
(because $2x_2\bar{x}_3=0$ in base 2)
}
\label{eq-bdd-truth-table-equiv}
\end{align}
Note that the right 
hand side of Eq.(\ref{eq-bdd-truth-table-equiv})
gives the same 3 \qt{Feynman histories}
as the coefficient of $\rva$ in
 Eq.(\ref{eq-3-histories-coeff}).

\begin{figure}[h!]
$$
\begin{array}{ccc}
\xymatrix{
\Circle{\rva}
\ar[d]|{x_1}
\\
\Circle{\rvb}
\ar[d]|{x_2}
\\
\Circle{\rvc}
}
&\;\;\;&
\xymatrix{
&\Circle{\rva}\ar[dl]|1
\ar[dr]|1
\\
\Circle{\rvb}\ar[dr]|{x_1}
&&\Circle{\rvc}\ar[dl]|{x_2}
\\
&\Circle{\rvd}
}
\\
\rvc=x_2\rvb=x_2x_1\rva
&&
\rvc= (x_1 + x_2)\rva
\end{array}
$$
\caption{Expressing the sum or product 
of two boolean variables $x_1$, $x_2$ 
as a LD bnet}
\label{fig-bdd-and-or}
\end{figure}

It's easy to  express the sum or product 
of two boolean variables $x_1$, $x_2$ 
as a LD bnet.
(see Fig.\ref{fig-bdd-and-or})
In general, any boolean polynomial
can be expressed as a LD bnet. In particular, the sum of products and product of sums
canonical forms
of any boolean expression can be expressed 
thus.