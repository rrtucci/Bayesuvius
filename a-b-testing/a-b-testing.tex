\chapter{A/B testing}
\label{ch-a-b-testing}

{\bf A/B testing} in its simplest
form is just frequentist {\bf hypothesis testing (HT)}.
The formulae used for HT
will not be  derived here. 
 The reader can consult a statistics book  for that.
In this chapter, we will merely state them,
and draw a causal bnet describing them.


\begin{figure}[h!]

{\entrymodifiers={++[F-:<3pt>]}

$$\xymatrix@R=15pt{
\vv{p}\ar[dd]
\ar[ddr]
\ar[dr]
\ar[drr]
&\vv{n}\ar[d]
&*{}
&(\alp, \beta)\ar[d]
\ar[ddr]
&*{}
&*{}
\\
*{}
&\vv{SE}\ar[d]\ar[r]
&Z\ar[dr]
&(Z_{\alp/2}, Z_\beta)\ar[dlll]
\ar[dll]
&*{}
&*{}
\\
n
&CI
&*{}
&p\ar[r]
&\text{decision}
}$$
}
\caption{Bnet for A/B testing (a.k.a. 
hypothesis testing.)}
\label{fig-bnet-hypo-test}
\end{figure}

Fig.\ref{fig-bnet-hypo-test}
gives a bnet for HT. The node names 
of the bnet are defined as follows

$\Phi()$: Cummulative Distribution
of Standard Normal Distribution
\beq
\Phi (x)=\frac{1}{\sqrt{2\pi}}
\int_{-\infty}^x dt\; e^{
-\frac{t^2}{2}}
\eeq

Conversions: The number of successful actions (e.g., purchases, sign-ups). 

$\xi\in\{A,B\}$: Variant. A is the control group, B is 
the treated group.

$p_\xi$: Conversion rate, the proportion of users who converted (for variant $\xi$).

$\vv{p}=(p_A, p_B)$

$n_\xi$: Sample size, number of visitors
(for variant $\xi$).
   
$\vv{n}=(n_A, n_B)$.

$(\alp, \beta)$: tail areas. One minus tail area = confidence level. $\alp=P(0|1)=$ probability of type 1 error. $\beta=P(1|0)=$
probability of type 2 error


$SE_\xi$: Standard error, measuring the variability of the conversion rate
(for variant $\xi$).
Called $\s_\xi/\sqrt{n_\xi}$ in Statistics.
    
$\vv{SE}=(SE_A, SE_B)$

$Z_\alp$: Z-score for confidence level
 $CL_\alp$ (e.g., 1.96 for 95\%).  

$Z_\beta$: Z-score for power level $\beta$ (e.g., 0.84 for 80\%).  

$n$: Minimum required sample size.
We require $n_A, n_B\geq n$.

$CI$: Confidence interval.  

$Z$: Z-score, measuring how different the two conversion rates are in standard deviation units.

$p$: p-value

decision: whether the result is statistically significant or not.

The structure equations of the nodes
of the bnet are defined as follows:


\begin{enumerate}

\item {\bf Confidence Levels (CL)}

\beq
CL_\alp = 
\left\{
\begin{array}{ll}
1-\frac{\alp}{2}
& \text{if two tailed}
\\
1-\alp & \text{if one-tailed}
\end{array}
\right.
\eeq

\beq
\alp = 
\left\{
\begin{array}{ll}
2(1- CL_\alp)
& \text{if two tailed}
\\
1- CL_\alp & \text{if one-tailed}
\end{array}
\right.
\eeq
Same with $\alp$ replaced by $\beta$.


\item {\bf Conversion Rate (a.k.a., proportion)(p)}  

\beq
p_\xi = \frac{\text{Conversions for $\xi$}}{\text{Total Visitors for $\xi$}}
\eeq
for $\xi=A,B$.

\item {\bf Standard Error (SE)}  

\beq
SE_\xi = \sqrt{\frac{p_\xi  (1 - p_\xi)}{n_\xi}}
\eeq
for $\xi=A,B$.


\item {\bf Z-Score (for hypothesis testing)}  

\beq
Z = \frac{p_A - p_B}{\sqrt{SE_A^2 + SE_B^2}}
\eeq

\item {\bf P-Value}  

\beq
p=
\left\{
\begin{array}{ll}
1 -\Phi(|Z|)
&\text{if one-tail testing}
\\
2[1 -\Phi(|Z|)]
&\text{if two-tailed testing }
\end{array}
\right.
\eeq



\item {\bf Decision}

Say $\alp=0.05$ (95\% confidence level)

If $p< \alp$,  overlap between Gaussians is small, reject null hypothesis,
a statistically significant result

If $p \geq \alp$, overlpa between Gaussians is large, fail to reject null hypothesis, not a statistically significant result

\item {\bf $Z_\alp, Z_\beta$  scores}

\beq
Z_\alp=
\Phi^{-1}(1 -\alp)
\eeq
Same with $\alp$ replaced by $\beta$.
Use $Z_{\alp/2}$ for two-tailed testing, $Z_\alp$ for one-tailed testing

\item {\bf Confidence Interval for Difference in Conversion Rates} 

\beq
a_\pm = (p_A - p_B) \pm Z_{\alpha/2} \sqrt{SE_A^2 + SE_B^2}
\eeq

\beq
CI = [a_-, a_+]
\eeq



\item {\bf Sample Size 
(for Proportion Tests)}
\beq
n = \frac{(Z_{\alpha/2} + Z_{\beta})^2 [p_A(1-p_A) + p_B(1-p_B)]}{(p_A - p_B)^2}
\eeq
We require $n_A, n_B\geq n$.

\end{enumerate}
