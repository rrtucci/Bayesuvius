\chapter{Green's function: COMING SOON}
\label{ch-greens-fun}


A {\bf Green's function} (GF)
(a.k.a. {\bf impulse response
 function, propagator}) is a solution to an ODE 
 with a Dirac delta function driving term and satisfying certain
 initial conditions or boundary conditions. 
 
GFs are very common in Physics and there is much
to say about them. In this chapter, we will confine ourselves
to explaining their connection to bnets  and causality.
For much more info about GFs, see Ref.\cite{wiki-greens-fun}.

\begin{figure}[h!]
$$
\xymatrix{
f(s_1)\ar[d]|{G(t_1, s_1)}
\ar[dr]\ar[drr]
&f(s_2)\ar[dl]\ar[d]\ar[dr]
&f(s_3)\ar[dll]\ar[dl]\ar[d]
\\
u(t_1)
&u(t_2)
&u(t_3)
}$$
\caption{bnet for 
$u(t)=  \int ds\; G(t, s)f(s)$}
\label{eq-bnet-greens-fun}
\end{figure}

For simplicity, assume that $t, s\in \RR$,
although. As it will become clear,
all of what we say
about GF in this chapter,
can be trivially 
generalized to $t, s\in \RR^n$

\beq
\call G(t, s)= \delta(t-s)
\eeq

\beq
\underbrace{\int ds\; f(s)\call G(t, s)}_{\call\int ds\; G(t, s)f(s)}= 
\underbrace{\int ds\; f(s)\delta(t-s)}_{f(t)}
\eeq

\beq
u(t)=  \int ds\; G(t, s)f(s)
\eeq
satisfies

\beq 
\call u(t) = f(t)
\eeq

\beq
G(t, s) = G(t-s)
\eeq

\beq
u(t)=  \int ds\; G(t-s)f(s)
\implies u(\omega) = G(\omega) f(\omega)
\eeq






\begin{claim}
For a damped harmonic oscillator,
$\call$ is

\beq
\call=\partial_t^2 +2\gamma\partial_t + \omega^2_0
\eeq
and the GF is

\beq
G(\tau)=\indi(t>0)
\left\{
\begin{array}{ll}
e^{-\gamma \tau} \, \frac{\sin(\omega \tau)}{\omega}
\text{ where }\omega =\sqrt{\omega_0^2 -\gamma^2}
& \text{if $\gamma < \omega_0$}
\\
e^{-\gamma \tau} t
&\text{if $\gamma =\omega_0$}
\\
e^{-\gamma \tau} \, \frac{\sinh(\omega \tau)}{\omega}
\text{ where }\omega =\sqrt{\gamma^2 -\omega_0^2}
&\text{if $\gamma > \omega_0$}
\end{array}
\right.
\eeq
where $\tau = t-s$
\end{claim}