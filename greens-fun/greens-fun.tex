\chapter{Green's function}
\label{ch-greens-fun}


A {\bf Green's function} (GF)
(a.k.a. {\bf impulse response
 function, propagator}) is a solution to an ODE 
 with a Dirac delta function driving term and satisfying certain
 initial conditions or boundary conditions. 
 
GFs are very common in Physics/Engineering and there is much
to say about them. In this chapter, we cover
just a minute fraction of the subject. We confine ourselves
to explaining their connection to bnets  and causality.
For much more info about GFs, see Ref.\cite{wiki-greens-fun}.

\begin{figure}[h!]
$$
\xymatrix@R=3pc{
f(s_1)\ar[d]|{G(t_1, s_1)\Delta s}
\ar[dr]\ar[drr]
&f(s_2)\ar[dl]\ar[d]\ar[dr]
&f(s_3)\ar[dll]\ar[dl]\ar[d]
\\
u(t_1)
&u(t_2)
&u(t_3)
}$$
\caption{bnet for 
$u(t)=  \int ds\; G(t, s)f(s)$
after discretizing $t$ and $s$.}
\label{eq-bnet-greens-fun}
\end{figure}

For simplicity, assume that $t, s\in \RR$,
although, as will soon become clear to the reader,
much of what we say
about GF in this chapter,
can be trivially 
generalized to $t, s\in \RR^n$.

Suppose $\call$ is a 
linear differential operator such as $\partial^2_t + a\partial_t + b$ for $a,b\in\RR$.
A Green's function $G(t,s)$ satisfies

\beq
\call G(t, s)= \delta(t-s)
\eeq
Multiplying both sides of the last equation by $\int ds\; f(s)$,
and using the linearity of $\call$, we get


\beq
\underbrace{\int ds\; f(s)\call G(t, s)}_{\call\int ds\; G(t, s)f(s)}= 
\underbrace{\int ds\; f(s)\delta(t-s)}_{f(t)}
\eeq
Thus, if we define $u(t)$ by

\beq
u(t)=  \int ds\; G(t, s)f(s)
\eeq
then $u(t)$ satisfies

\beq 
\call u(t) = f(t)
\eeq

If the coefficients of the derivatives in $\call$
are constants, the system will be {\bf translationally
invariant} (i.e. the ODE does not change
form if you change $t$ to $t-a$ where
$a\in\RR$ is a constant).
For translational invariant systems,
the GF depends only on $t$ differences, so

\beq
G(t, s) = G(t-s)
\eeq
Hence, 

\beq
u(t)=  \int ds\; G(t-s)f(s)
\implies u(\omega) = G(\omega) f(\omega)
\eeq

Physical systems are always {\bf causal}. 
This means the future cannot influence the
present. If $t, s$ represent time, the GP of a causal system must satisfy

\beq
G(t, s)=0\quad\text{if } t<s
\eeq


\begin{claim}
For a damped harmonic oscillator,
$\call$ is

\beq
\call=\partial_t^2 +2\gamma\partial_t + \omega^2_0
\eeq
and the GF is

\beq
G(\tau)=\indi(t>0)
\left\{
\begin{array}{ll}
e^{-\gamma \tau} \, \frac{\sin(\TIL{\omega}_0 \tau)}{\TIL{\omega}_0}
\text{ where }\TIL{\omega}_0 =\sqrt{\omega_0^2 -\gamma^2}
& \text{if $\gamma < \omega_0$}
\\
e^{-\gamma \tau} t
&\text{if $\gamma =\omega_0$}
\\
e^{-\gamma \tau} \, \frac{\sinh(\TIL{\gamma} \tau)}{\TIL{\gamma}}
\text{ where }\TIL{\gamma} =\sqrt{\gamma^2 -\omega_0^2}
&\text{if $\gamma > \omega_0$}
\end{array}
\right.
\eeq
where $\tau = t-s$
\end{claim}
\proof

Find solution of $\call G(\tau)=\delta(\tau)$ in standard way with Laplace transforms. The standard way is to take the Laplace transform
of both sides, solve for
the Laplace transform $G(s)$ of $G(t)$, then invert the Laplace Transform.  

Fourier transforms
are for getting steady state solutions,
not transient ones like this one. They cannot accommodate the initial condition $G(\tau)=0$
for $\tau<0$.
\qed