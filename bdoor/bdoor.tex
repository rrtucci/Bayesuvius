\chapter{Backdoor Adjustment Formula}
\label{ch-bdoor}

The backdoor (BD) adjustment
formula is proven in
Chapter \ref{ch-do-calc}
from the rules of Do Calculus.
The goal 
of this chapter is
to give examples
of the use of that
theorem.
We will restate
the theorem in this chapter,
sans proof.
There is no need
to understand the
theorem's
proof in order to use it.
However, you
will
need to skim Chapter \ref{ch-do-calc}
in order to familiarize 
yourself with
the notation used to state the 
theorem.
This chapter also assumes
that you are comfortable 
with the  rules 
for checking for d-separation. Those rules
are covered in Chapter \ref{ch-dsep}.



\bdoordef
\begin{claim} {\bf Backdoor Adjustment
 Formula}

\bdoorclaim
\end{claim}
\proof 
See Chapter \ref{ch-do-calc}.
\qed



\section{Examples}

\hrule
\begin{enumerate}
\item
\beq
\xymatrix{
&\rvz\ar[dr]
\\
\rvx\ar[rr]\ar[ru]&&\rvy
}
\eeq

If
$\rvx.=\rvx, \rvy.=\rvy, 
\rvz.=\rvz$, BD criterion not satisfied because 
$\rvz\in de(\rvx)$.
However, 

\beq
P(y|\cald \rvx=x)=
\xymatrix{
&\sum z\ar[dr]
\\
x\ar[rr]\ar[ru]&&y
}=P(y|x)
\eeq

\hrule
\item
\beq
\xymatrix{
&\rvz
\\
\rvx\ar[rr]\ar[ru]&&\rvy\ar[ul]
}
\eeq

If
$\rvx.=\rvx, \rvy.=\rvy, 
\rvz.=\rvz$, BD criterion not satisfied because 
$\rvz\in de(\rvx)$.
However, 

\beq
P(y|\cald \rvx=x)=
\xymatrix{
&\sum z
\\
x\ar[rr]\ar[ru]&&y\ar[ul]
}
= P(y|x)
\eeq

\hrule\item
\beq
\xymatrix{
&\rvz\ar[dl]\ar[dr]
\\
\rvx\ar[rr]&&\rvy
}
\eeq
BD criterion satisfied if
$\rvx.=\rvx, \rvy.=\rvy, \rvz.=\rvz$.
In fact,

\beq
P(y|\cald\rvx=x)=
\xymatrix{
&\EE z\ar[dr]
\\
x\ar[rr]&&y
}
= P(y|x)
\eeq

Note that 
here the backdoor formula adjusts
the parents  of $\rvx.$.
More complicated examples might adjust for non-parents.



\hrule\item
\beq
\xymatrix{
&\rvz\ar[dl]\ar[dr]
\\
\rvx\ar[r]&\rvm\ar[r]&\rvy
}
\eeq
BD criterion satisfied if
$\rvx.=\rvx, \rvy.=\rvy, \rvz.=\rvz$.
In fact,

\beq
P(y|\cald\rvx=x)=
\xymatrix{
&\EE z\ar[dr]
\\
x\ar[r]&\sum m\ar[r]&y
}
=
P(y|x)
\eeq

\hrule\item

\beq
\xymatrix{
&\rvz_1\ar[ddl]\ar[ddr]\ar[d]
\\
&\rvz_2\ar[dl]\ar[dr]
\\
\rvx\ar[rr]
&&\rvy
}
\eeq
BD criterion satisfied if
$\rvx.=\rvx, \rvy.=\rvy, \rvz.=(\rvz_1, \rvz_2)$.
In fact,

\beq
P(y|\cald \rvx=x)=
\xymatrix{
&\EE z_1\ar[ddr]\ar[d]
\\
&\sum z_2\ar[dr]
\\
x\ar[rr]
&&y
}= P(y|x)
\eeq

\hrule\item
\beq
\xymatrix{
\rvz_1\ar[dd]\ar[rd]
&&\rvz_3\ar[dd]\ar[dl]
\\
&\rvz_2
\\
\rvx\ar[rr]
&&\rvy
}
\eeq

BD criterion satisfied if
$\rvx.=\rvx, \rvy.=\rvy$
and  $\rvz.=\rvz_1$.
In fact,

\beq
\xymatrix
{\\
P(y|\cald \rvx=x)=}
\xymatrix{
\EE z_1\ar[rd]
&&\sum z_3\ar[dd]\ar[dl]
\\
&\sum z_2
\\
x\ar[rr]
&&y
}
\xymatrix{\\
=P(y|x)
}
\eeq




%\begin{multicols}{4}
%\begin{itemize}
%\item $ \emptyset$
%\item $\rvz_1$
%\item $\rvz_3$
%\item $\rvz_1, \rvz_2$
%\item $\rvz_2,\rvz_3$
%\item $\rvz_1, \rvz_3$
%\item $\rvz_1, \rvz_2, \rvz_3$
%\end{itemize}
%\end{multicols}

\hrule\item
\beq
\xymatrix{
&*++[F-o]{\rvz}\ar[dl]\ar[dr]
\\
\rvx\ar[r]&\rvm\ar[r]&\rvy
}
\eeq
First sum over $\rvm$. BD criterion is
not satisfied if 
$\rvx.=\rvx, \rvy.=\rvy, \rvz. =\rvz$
because $\rvz$ is hidden and can't be conditioned on.
However, the frontdoor criterion can be
satisfied. See Chapter
\ref{ch-fdoor}.

\hrule\item
\beq
\xymatrix{
*++[F-o]{\rvw}\ar[d]\ar[r]
&\rvz\ar[d]
\\
\rvx\ar[r]&\rvy
}
\eeq

First sum over $\rvz$ node. Now BD criterion not satisfied if\footnote{To be totally correct,
the backdoor criterion should ask that
all nodes besides $\rvx., \rvy., \rvz.$ be summed over.
}
$\rvx.=\rvx, \rvy.=\rvy, \rvz.=\rvw$,
because $\rvw$ is hidden and can't be 
averaged over.
This query is not identifiable.

\hrule\item
\beq
\xymatrix{
*++[F-o]{\rve}\ar[d]\ar[r]
&\rvz\ar[dl]\ar[dr]
&\rva\ar[d]\ar[l]
\\
\rvx\ar[rr]&&\rvy
}
\eeq

%Conditioning
%on $\rvz$
%blocks 
%backdoor path
%$\rvx-\rvz-\rvy$, 
%but 
%opens path $\rvx-\rve-\rvz-\rva-\rvy$
%because $\rvz$ is a collider
%for that path. That
%path is blocked
%if we also
%condition on $\rva$, 
%which is possible
%because $\rva$ is
%observed.
%In conclusion,
%the BD criterion is satisfied if
%$\rvx.=\rvx$, 
%$\rvy.=\rvy$
%and 
%$\rvz.=(\rvz, \rva)$.
%
%Conditioning on 
%the parents of 
%$\rvx.$
%is often
%enough
%to block
%all
%backdoor paths.
%However, sometimes
%some of the 
%parents are unobserved 
%and one must 
%condition on other
%nodes that
%are not parents of $\rvx.$
%in order to satisfy
%the BD criterion. 

First sum over the $\rvz, \rva$ nodes. Now BD criterion not satisfied if
$\rvx.=\rvx, \rvy.=\rvy, \rvz.=\rve$,
because $\rve$ is hidden and can't be averaged over.
This query is not identifiable.


\hrule\item
\beq
\xymatrix{
\rvz\ar[d]&&\rvt\ar[ll]\ar[d]
\\
\rvw&\rvx\ar[r]\ar[l]&\rvy
}
\eeq
BD criterion not satisfied. In fact,

\beq
\xymatrix{\\
P(y|\cald \rvx=x)=
}
\xymatrix{
\sum z\ar[d]&&\sum t\ar[ll]\ar[d]
\\
\sum w&x\ar[r]\ar[l]&y
}
\eeq



%\hrule
%\item
%Discuss what to do if
%several sets $\rvz.$
%satisfy the BD criterion.
%\begin{itemize}
%\item
%Can evaluate $P(y.|\cald \rvx.=x.)$
%multiple ways and compare the results.
%This is a test that the causal bnet 
%is correct.
%\item
%It might 
%be easier or 
%less expensive to get data for
%some $\rvz.$ 
%more than for others.
%\end{itemize}

\hrule
\item
(Taken from online course notes 
Ref.\cite{ethz-causality})

Consider the bnet

\beq
\xymatrix{
\rvx_2\ar[d]\ar[r]
&\rvx_3&
\rvx_4\ar[d]\ar[l]
\\
\rvx\ar[d]\ar[r]
&\rvx_6\ar[d]\ar[r]
&\rvy\ar[d]
&\rvx_7\ar[l]
\\
\rvx_8&\rvx_9&\rvx_{10}
}
\eeq
Find all  {\bf conditioning node sets} that block all
non-direct paths
 between $\rvx$ and $\rvy$.

Ans:
\begin{multicols}{4}
\begin{itemize}
\item $ \emptyset$
\item $\rvx_2$
\item $\rvx_4$
\item $\rvx_2, \rvx_4$
\item $\rvx_2,\rvx_3$
\item $\rvx_3, \rvx_4$
\item $\rvx_2, \rvx_3, \rvx_4$
\end{itemize}
\end{multicols}
Add $\rvx_7$
to each of the previous 7 possible
$\rvz.$. This gives
 a total of 14 possible
 conditioning sets. 


\end{enumerate}