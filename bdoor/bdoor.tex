\chapter{Backdoor Adjustment}
\label{chap-bdoor}

The backdoor (BD) adjustment
theorem is proven in 
Chapter \ref{chap-do-calc}
from the rules of do-calculus.
The goal 
of this chapter is
to give examples
of the use of that
theorem.
We will restate
the theorem in this chapter,
sans proof.
There is no need
to understand the
theorem's
proof in order to use it.
However, you
will
need to skim Chapter \ref{chap-do-calc}
in order to familiarize 
yourself with
the notation used to state the 
theorem.
This chapter also assumes
that you are comfortable 
with the  rules 
for checking for d-separation. Those rules
are covered in Chapter \ref{chap-dsep}.



\bdoordef
\begin{claim} {\bf Backdoor Adjustment
 Theorem}

\bdoorclaim
\end{claim}
\proof 
See Chapter \ref{chap-do-calc}
\qed

Examples:
\begin{enumerate}
\hrule\item
\beq
\xymatrix{
&\rvz\ar[dr]
\\
\rvx\ar[rr]\ar[ru]&&\rvy
}
\eeq

BD criterion satisfied if
$\rvx.=\rvx, \rvy.=\rvy, \rvz.=\emptyset$.
 No adjustment necessary.

\beq
P(y|\rho \rvx=x)=P(y|x)
\eeq

\hrule\item
\beq
\xymatrix{
&\rvz\ar[dl]\ar[dr]
\\
\rvx\ar[rr]&&\rvy
}
\eeq
BD criterion satisfied if
$\rvx.=\rvx, \rvy.=\rvy, \rvz.=\rvz$.

Note that 
here the backdoor formula adjusts
the parents  of $\rvx.$.

\hrule\item
\beq
\xymatrix{
&\rvz\ar[dl]\ar[dr]
\\
\rvx\ar[r]&\rvm\ar[r]&\rvy
}
\eeq
BD criterion satisfied if
$\rvx.=\rvx, \rvy.=\rvy, \rvz.=\rvz$.

This bnet
is also used
to demonstrate
the front-door criterion.

\hrule\item
\beq
\xymatrix{
*+[F]{\rvw}\ar[d]\ar[r]
&\rvz\ar[d]
\\
\rvx\ar[r]&\rvy
}
\eeq

BD criterion satisfied if
$\rvx.=\rvx, \rvy.=\rvy, \rvz.=\rvz$.
Note that 
here the backdoor formula cannot
adjust the single parent $\rvw$
of $\rvx$ because it is hidden, 
but we are able to 
block the backdoor path 
by conditioning on $\rvz$ 
instead.


\hrule\item
\beq
\xymatrix{
*+[F]{\rve}\ar[d]\ar[r]
&\rvz\ar[dl]\ar[dr]
&\rva\ar[d]\ar[l]
\\
\rvx\ar[rr]&&\rvy
}
\eeq

Conditioning
on $\rvz$
blocks 
backdoor path
$\rvx-\rvz-\rvy$, 
but 
opens path $\rvx-\rve-\rvz-\rva-\rvy$
because $\rvz$ is a collider
for that path. That
path is blocked
if we also
condition on $\rva$, 
which is possible
because $\rva$ is
observed.
In conclusion,
the BD criterion is satisfied if
$\rvx.=\rvx$, 
$\rvy.=\rvy$
and 
$\rvz.=(\rvz, \rva)$.

Conditioning on 
the parents of 
$\rvx.$
is often
enough
to block
all
backdoor paths.
However, sometimes
some of the 
parents are unobserved 
and one most 
condition on other
nodes that
are not parents of $\rvx.$
in order to satisfy
the BD criterion. 


\hrule\item
\beq
\xymatrix{
\rvz\ar[d]&&\rvt\ar[ll]\ar[d]
\\
\rvw&\rvx\ar[r]\ar[l]&\rvy
}
\eeq

No need to control
anything 
because only possible
backdoor path is blocked by collider $\rvw$.
Hence,

\beq
P(y|\rho\rvx=x)=P(y|x)
\;.
\eeq

However, 
if for some reason 
we want to control
$\rvw$, we
can block the  path
by controlling $\rvt$ too.
Thus, the
BD criterion is
satisfied if
 $\rvx.=\rvx$,
$\rvy.=\rvy$ and 
$\rvz.=(\rvw, \rvt)$.
Therefore, 

\beq
P(y|\rho \rvx=x)=
\sum_{t,w} P(y|x, t, w)P(t, w)
\label{eq-bdoor-tw-sum}
\;.
\eeq

Alternatively,
can condition
on $\rvw$ a priori,
and satisfy the
BD criterion with
$\rvx.=\rvx$, $\rvy.=\rvy$
and $\rvz.=\rvt$;
thus,

\beq
P(y|\rho \rvx=x, w)=
\sum_t P(y|x, t, w)P(t|w)
\label{eq-bdoor-t-sum}
\;.
\eeq
Multiplying
Eq.(\ref{eq-bdoor-t-sum})
by $P(w)$
and summing
over $w$
gives Eq.(\ref{eq-bdoor-tw-sum}).

\item
Discuss reasons why 
multiple possible sets $\rvz.$
that satisfy the BD criterion
can be advantageous.
\begin{itemize}
\item
Can evaluate $P(y.|\rho \rvx.=x.)$
multiple ways and compare the results.
This is a test that the causal bnet 
is correct.
\item
Some $\rvz.$ might 
be easier or 
less expensive to get data on.
\end{itemize}
  



\end{enumerate}