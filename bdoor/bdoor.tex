\chapter{Backdoor Adjustment Formula}
\label{ch-bdoor}

The backdoor (BD) adjustment
formula is proven in
Chapter \ref{ch-do-calc}
from the rules of Do Calculus.
The goal 
of this chapter is
to give examples
of the use of that
theorem.
We will restate
the theorem in this chapter,
sans proof.
There is no need
to understand the
theorem's
proof in order to use it.
However, you
will
need to skim Chapter \ref{ch-do-calc}
in order to familiarize 
yourself with
the notation used to state the 
theorem.
This chapter also assumes
that you are comfortable 
with the  rules 
for checking for d-separation. Those rules
are covered in Chapter \ref{ch-dsep}.



\bdoordef
\begin{claim} {\bf Backdoor Adjustment
 Formula}

\bdoorclaim
\end{claim}
\proof 
See Chapter \ref{ch-do-calc}.
\qed

\section{Examples}
\begin{enumerate}
\item
\beq
\xymatrix{
&\rvz\ar[dr]
\\
\rvx\ar[rr]\ar[ru]&&\rvy
}
\eeq

BD criterion satisfied if
$\rvx.=\rvx, \rvy.=\rvy, \rvz.=\emptyset$.
 No adjustment necessary.

\beq
P(y|\cald \rvx=x)=P(y|x)
\eeq

\hrule\item
\beq
\xymatrix{
&\rvz\ar[dl]\ar[dr]
\\
\rvx\ar[rr]&&\rvy
}
\eeq
BD criterion satisfied if
$\rvx.=\rvx, \rvy.=\rvy, \rvz.=\rvz$.

Note that 
here the backdoor formula adjusts
the parents  of $\rvx.$.

\hrule\item
\beq
\xymatrix{
&\rvz\ar[dl]\ar[dr]
\\
\rvx\ar[r]&\rvm\ar[r]&\rvy
}
\eeq
BD criterion satisfied if
$\rvx.=\rvx, \rvy.=\rvy, \rvz.=\rvz$.

\hrule\item
\beq
\xymatrix{
&*++[F-o]{\rvz}\ar[dl]\ar[dr]
\\
\rvx\ar[r]&\rvm\ar[r]&\rvy
}
\eeq
BD criterion is
impossible to satisfy if
$\rvx.=\rvx, \rvy.=\rvy$.
However, the frontdoor criterion can be
satisfied. See Chapter
\ref{ch-fdoor}.

\hrule\item
\beq
\xymatrix{
*++[F-o]{\rvw}\ar[d]\ar[r]
&\rvz\ar[d]
\\
\rvx\ar[r]&\rvy
}
\eeq

BD criterion satisfied if
$\rvx.=\rvx, \rvy.=\rvy, \rvz.=\rvz$.
We are able to 
block the backdoor path 
by conditioning on $\rvz$.


\hrule\item
\beq
\xymatrix{
*++[F-o]{\rve}\ar[d]\ar[r]
&\rvz\ar[dl]\ar[dr]
&\rva\ar[d]\ar[l]
\\
\rvx\ar[rr]&&\rvy
}
\eeq

Conditioning
on $\rvz$
blocks 
backdoor path
$\rvx-\rvz-\rvy$, 
but 
opens path $\rvx-\rve-\rvz-\rva-\rvy$
because $\rvz$ is a collider
for that path. That
path is blocked
if we also
condition on $\rva$, 
which is possible
because $\rva$ is
observed.
In conclusion,
the BD criterion is satisfied if
$\rvx.=\rvx$, 
$\rvy.=\rvy$
and 
$\rvz.=(\rvz, \rva)$.

Conditioning on 
the parents of 
$\rvx.$
is often
enough
to block
all
backdoor paths.
However, sometimes
some of the 
parents are unobserved 
and one must 
condition on other
nodes that
are not parents of $\rvx.$
in order to satisfy
the BD criterion. 


\hrule\item
\beq
\xymatrix{
\rvz\ar[d]&&\rvt\ar[ll]\ar[d]
\\
\rvw&\rvx\ar[r]\ar[l]&\rvy
}
\eeq

No need to control
anything 
because only possible
backdoor path is blocked by
not conditioning on collider $\rvw$.
Hence,

\beq
P(y|\cald\rvx=x)=P(y|x)
\;.
\eeq

However, 
if for some reason 
we want to control
$\rvt$, we
can do so. We  can't
control
$\rvw$ though, 
because $\rvw\in de(\rvx)$.
Thus, the
BD criterion is
satisfied if
 $\rvx.=\rvx$,
$\rvy.=\rvy$ and 
$\rvz.=\rvt$.
Therefore, 

\beq
P(y|\cald \rvx=x)=
\sum_{t} P(y|x, t)P(t)
\label{eq-bdoor-t-sum}
\;.
\eeq


\hrule
\item
Discuss what to do if
several sets $\rvz.$
satisfy the BD criterion.
\begin{itemize}
\item
Can evaluate $P(y.|\cald \rvx.=x.)$
multiple ways and compare the results.
This is a test that the causal bnet 
is correct.
\item
It might 
be easier or 
less expensive to get data for
some $\rvz.$ 
more than for others.
\end{itemize}

\hrule
\item (Taken from online course notes 
Ref.\cite{ethz-causality})

Consider the bnet

\beq
\xymatrix{
\rvx_2\ar[d]\ar[r]
&\rvx_3&
\rvx_4\ar[d]\ar[l]
\\
\rvx_1\ar[d]\ar[r]
&\rvx_6\ar[d]\ar[r]
&\rvx_5\ar[d]
&\rvx_7\ar[l]
\\
\rvx_8&\rvx_9&\rvx_{10}
}
\eeq
If $\rvx.=\rvx_1$ and 
$\rvy.=\rvx_5$, find
all possible 
adjustment multinodes $\rvz.$ that 
satisfy the BD criterion.
Ans:
\begin{multicols}{4}
\begin{itemize}
\item $ \emptyset$
\item $\rvx_2$
\item $\rvx_4$
\item $\rvx_2, \rvx_4$
\item $\rvx_2,\rvx_3$
\item $\rvx_3, \rvx_4$
\item $\rvx_2, \rvx_3, \rvx_4$
\end{itemize}
\end{multicols}
Add $\rvx_7$
to each of the previous 7 possible
$\rvz.$. This gives
 a total of 14 possible 
adjustment multinodes $\rvz.$. 


  



\end{enumerate}