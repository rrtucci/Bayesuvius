\chapter{Thermodynamics, Causal Perspective}
\label{ch-thermo}

For a summary of Thermodynamics, see
see \cite{wiki-thermo}.

Modern day books on Thermodynamics
derive its 3 laws from either classical
or quantum statistical mechanics, or using classical or quantum stochastic  equations
(see chapter\ref{ch-stochastic-diff-eqns}). However
the 3 laws 
were originally derived from causal type
	arguments and experimentation, in much
	the same way that one  derives a bnet
	as a hypothesis which is then tested.
Fig.\ref{fig-texnn-for-thermo} is a bnet for thermo that
captures some of those causal arguments.
The structure equations for the bnet are printed in blue.

	


\begin{figure}[h!]\centering
	$$\xymatrix{
		&*+[F*:SpringGreen]{\underline{V}}\ar@/_1pc/[dddd]\ar[ddddr]\ar[dl]&*+[F*:SpringGreen]{\underline{\{N_i\}}}\ar[ddddr]\ar[dll]&
		\\
		*+[F*:SpringGreen]{\underline{U}}\ar@/_1pc/[dddrrr]\ar[ddd]\ar[dddr]\ar[dddrr]&&*+[F*:Lavender]{\underline{S}}\ar[dddr]\ar@/_1pc/[dddll]\ar[dddl]\ar[ll]&
		\\
		&&&
		\\
		&*+[F*:Lavender]{\underline{T}}\ar[drr]\ar[dl]\ar[d]&*+[F*:Lavender]{\underline{p}}\ar[dl]\ar[d]&*+[F*:Lavender]{\underline{\{\mu_i\}}}\ar[d]
		\\
		*+[F*:SkyBlue]{\underline{F}}&*+[F*:SkyBlue]{\underline{G}}&*+[F*:SkyBlue]{\underline{H}}&*+[F*:SkyBlue]{\underline{\Phi}}
	}$$
	\caption{Thermodynamics, a causal perspective. Extrinsic variables in green, Intrinsic ones in pink, and Legendre  transforms of $U$ in blue.}
	\label{fig-texnn-for-thermo}
\end{figure}

\begin{subequations}
	
	\begin{equation}\color{blue}
		\Phi = U-TS-\sum_i\mu_iN_i\;\;\text{(Grand Potential)}
		\label{eq-X-fun-thermo}
	\end{equation}
	
	\begin{equation}\color{blue}
		\{\mu_i\} = \pder{U}{\{N_i\}}\;\;\text{(chemical potential for species $i$)}
		\label{eq-m-fun-thermo}
	\end{equation}
	
	\begin{equation}\color{blue}
		\{N_i\} = {\rm prior}\;\;\text{(number of particles of species $i$)}
		\label{eq-N-fun-thermo}
	\end{equation}
	
	\begin{equation}\color{blue}
		F = U-TS\;\;\text{(Helmholtz free energy)}
		\label{eq-F-fun-thermo}
	\end{equation}
	
	\begin{equation}\color{blue}
		G = U+pV-TS\;\;\text{(Gibbs free energy)}
		\label{eq-G-fun-thermo}
	\end{equation}
	
	\begin{equation}\color{blue}
		H = U+pV\;\;\text{(enthalpy)}
		\label{eq-H-fun-thermo}
	\end{equation}
	
	\begin{equation}\color{blue}
		p = -\;\pder{U}{V}\;\;\text{(pressure)}
		\label{eq-p-fun-thermo}
	\end{equation}
	
	\begin{equation}\color{blue}
		S = {\rm prior}\;\;\text{(entropy)}
		\label{eq-S-fun-thermo}
	\end{equation}
	
	\begin{equation}\color{blue}
		T = \pder{U}{S}\;\;\text{(temperature)}
		\label{eq-T-fun-thermo}
	\end{equation}
	
	\begin{equation}\color{blue}
		U = U(S, V, \{N_i\})\;\;\text{(internal energy)}
		\label{eq-U-fun-thermo}
	\end{equation}
	
	\begin{equation}\color{blue}
		V = {\rm prior}\;\;\text{(volume)}
		\label{eq-V-fun-thermo}
	\end{equation}
	
\end{subequations}
