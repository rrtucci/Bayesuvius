\chapter{D-Separation}
\label{ch-dsep}
Before reading this chapter,
I  recommend
that you
read
Chapter \ref{ch-bnet-def}.


A path $\gamma$ that
isn't a loop can have 
3 types of intermediate
 nodes $\rvx$ (an intermediate
 node of $\gamma$
 is a node in $\gamma$ that 
isn't one
of the two end nodes).
Suppose $\rva,\rvb\in \gamma$
are the two neighbors of $\rvx$. Then
the 3 possible cases are:
\begin{enumerate}
\item {\bf$\rvx$ is a mediator node:}
$(\rva\larrow\rvx\larrow \rvb)$
or
$(\rva\rarrow\rvx\rarrow \rvb)$
\item {\bf $\rvx$ is a fork node:}
$(\rva\larrow\rvx\rarrow \rvb)$
\item {\bf $\rvx$ is a collider node:}
$(\rva\rarrow\rvx\larrow \rvb)$
\end{enumerate}

We say that a non-loop path 
$\gamma$ 
from $\rva$ to $\rvb$ (i.e., with
end nodes $\rva, \rvb$)
is {\bf blocked}
by conditioning on a multinode $\rvz.$
if one or more 
of the following
statements is true:

\begin{enumerate}
\item 
There is a node $\rvz_0\in \rvz.$
which is a mediator 
or a fork of $\gamma$.
\item
$\gamma$ contains a collider
node $\rvc$
and 
$(\rvc\cup de(\rvc))\cap\rvz.=\emptyset$
(i.e., neither 
$\rvc$ nor 
any of the descendants of $\rvc$
is contained in $\rvz.$)
\end{enumerate}

This definition of a blocked 
path\footnote{Note
that we speak of 
blocked paths or info,
not of blocked nodes.
Nodes are not blocked;
rather they are either 
conditioned upon
or not.}
 is easy to remember
if one thinks 
of the following analogy
with pipes carrying water.
Think of path
$\gamma$ as if it
were a pipe
carrying water.
Think of
the nodes 
of $\gamma$ as junctions in the pipe.
If $\rvz.$
intersects $\gamma$
at either a mediator
or a fork junction,
that acts like a stone
that blocks the pipe flow.
A collider junction $\rvc$
is like a t-joint or hole in the pipe
causing a jet of water to leak away from the main flow.
Its presence
diminishes or totally 
prevents the main flow of  water
as long
as neither
$\rvc$
nor any of
the descendants 
of $\rvc$
are in $\rvz.$.
If,
on the 
other hand,
$\rvc\in\rvz.$,
or $\rvc'\in \rvz.$
where $\rvc'\in de(\rvc)$,
then the stone
produces
a complete
(in the case of $\rvc\in\rvz.$)
or a partial 
(in the case of $\rvc'\in\rvz.$)
plug of the leak,
preventing egress from the main flow.

See Fig.\ref{fig-blocked-paths}
for some examples of
paths that are blocked or not blocked
by conditioning on
a multinode $\rvz.$.

\begin{figure}[h!]
\beqa
\xymatrix{
\circ\ar[r]
&\circ\ar[r]
&\circ\ar[r]
&\circ\ar[r]
&\circ
}&\text{Not Blocked}
\\
\xymatrix{
\circ\ar[r]
&\color{yellow}\bullet\ar[r]
&\circ\ar[r]
&\circ\ar[r]
&\circ
}&\text{Blocked}
\\
\xymatrix{
\circ
&\circ\ar[l]\ar[r]
&\circ\ar[r]
&\circ\ar[r]
&\circ
}&\text{Not Blocked}
\\
\xymatrix{
\circ
&\color{yellow}\bullet\ar[l]\ar[r]
&\circ\ar[r]
&\circ\ar[r]
&\circ
}&\text{Blocked}
\\
\xymatrix{
\circ\ar[r]
&\circ\ar[r]
&\circ
&\circ\ar[l]\ar[r]
&\circ
}&\text{Blocked}
\\
\xymatrix{
\circ\ar[r]
&\circ\ar[r]
&\color{yellow}\bullet
&\circ\ar[l]\ar[r]
&\circ
}&\text{Not Blocked}
\\
\xymatrix{
\circ\ar[r]
&\circ\ar[r]
&\circ\ar[d]
&\circ\ar[l]\ar[r]
&\circ
\\
&&\color{yellow}\bullet
}&\text{Not Blocked}
\eeqa
\caption{Examples of 
paths that are blocked
or not blocked
by conditioning on a multinode $\rvz.$. Nodes
belonging to 
$\rvz.$
are colored yellow.}
\label{fig-blocked-paths}
\end{figure}

Given 3 
disjoint multinodes 
$\rva., \rvb., \rvz.$
of a graph $G$,
we write \qt{
$\rva.\perp_G \rvb.|\rvz.$}
or say \qt{ {\bf$\rva.$ and
$\rvb.$ are d-separated
by $\rvz.$ in $G$} }
iff there exists 
no path
$\gamma$ from
$\rva_0\in \rva.$,
to
$\rvb_0\in\rvb.$
which is not 
blocked by conditioning on
$\rvz.$.\footnote{
$\rvz.$ are the nodes
we are \qt{conditioning on}.
Unmeasured (i.e., hidden,  unobserved) 
nodes cannot be
conditioned on, because
that would entail
measuring them.
} 

The minimal 
Markov blanket (see Chapter
\ref{ch-mblanket})
of a node $\rva$
is the smallest 
multinode $\rvz.$
such that $\rva\perp_G\rvb|\rvz.$
for all $\rvb\notin \rva\cup\rvz.$.

We are finally ready
to state the d-separation
theorem, without proof.

A
 probability
distribution
{\bf $P$
is 
compatible 
with a DAG $G$}
if $P$ and $G$ 
have the same
random variables, and they 
can be
combined to form a bnet
without
contradictions;
i.e.,
one can calculate 
all
the TPMs from $P$
and multiply
them 
together to
obtain $P$ again.

\begin{claim}(d-separation Theorem)

Suppose
$\rva., \rvb., \rvz.$
are disjoint multinodes
of a DAG  $G$.

If 
$\rva.\perp_G \rvb.|\rvz.$, then
$P(b.|a., z.)=P(b.|z.)$
for all $b.,a., z.$,
for all $P$
compatible with $G$.

\end{claim}
\proof
\footnote{The full converse
of the theorem can also be 
proven.}

Motivation for D-separation Theorem

\begin{itemize} 

\item {\bf mediator $\rvz$ in path $\rva-\rvz-\rvb$}

\beq\xymatrix{
\rva\ar[r] 
&{\rvz}\ar[r]
&\rvb
}
\eeq

\beq
P(a, b, z)=P(b|z)P(z|a)P(a)
\eeq

\beq
P(a|z)=\frac{P(z|a)P(a)}{P(z)}
\eeq

\beq
P(a|z)P(b|z)=\frac{P(a,b,z)}{P(z)}=P(a,b|z)
\eeq

Hence, conditioning
on $\rvz$ makes $\rva$ and $\rvb$ independent.

\item {\bf fork $\rvz$ in path $\rva-\rvz-\rvb$}

\beq
\xymatrix{
\rva
&{\rvz}\ar[r]\ar[l]
&\rvb
}\eeq

\beq
P(a,b,z)=
P(a|z)P(z)P(b|z)
\eeq

\beq
P(a|z)P(b|z)=P(a,b|z)
\eeq
Hence, conditioning
on $\rvz$ makes $\rva$ and $\rvb$ independent.

\item {\bf collider $\rvz$ in path $\rva-\rvz-\rvb$}

\beq
\xymatrix{
\rva\ar[r] 
&{\rvz}
&\rvb\ar[l]
}\eeq

\beq
P(a,b,z)= P(z|a,b)P(a)P(b)
\eeq

\beq
P(a, b)=  P(a)P(b)
\eeq
Hence, if instead of conditioning
on $\rvz$, we sum over it,  
$\rva$ and $\rvb$ become independent.


\item {\bf collider $\rvz$ 
with descendant $\rvxi$ in path $\rva-\rvz-\rvb$}
\beq
\xymatrix{
\rva\ar[r] 
&{\rvz}\ar[d]
&\rvb\ar[l]
\\
&\rvxi
}
\eeq 

\beq
P(a,b, z, \xi)=
P(\xi|z)P(z|a,b)P(a)P(b)
\eeq

\beq
P(a, b, \xi)=
\underbrace{\sum_z [P(\xi|z)P(z|a,b)]}_
{P(\xi|a,b)}P(a)P(b)
\eeq
Hence, summing over the collider $\rvz$
makes the collider descendant $\rvxi$
look like it's the collider even though the  
 real collider is $\rvz$.
\qed

\end{itemize}

Often, the right hand side
of this theorem is stated as 
\qt{$\rva.\perp_P \rvb.|\rvz.$
for all $P$}.
Then the theorem is stated:
\qt{If
$\rva.\perp_G \rvb.|\rvz.$, then
$\rva.\perp_P \rvb.|\rvz.$ for all $P$.}

\hrule
Note that 
the following are equivalent:
\begin{itemize}
\item
$P(b.|a., z.)=P(b.|z.)$ for all $b., a., z.$.
\item
$\rva.\perp_P \rvb.|\rvz.$
\item
$H(\rva.:\rvb.|\rvz.)=0$
(see Chapter \ref{ch-conventions}
for definition of
 conditional mutual information (CMI))
\end{itemize}
\hrule\noindent
{\bf Extra stuff: mostly only for 
 pure mathematicians}

Below, we will use
the notation $nde(\rva)$
to denote
all non-descendants,
including $\rva$ itself, 
of a node $\rva$
in a DAG $G$; i.e.,
all nodes of $G$ that are not
in $de(\rva)\cup \rva$, where
$de(\rva)$
is defined in Chapter \ref{ch-bnet-def}.

Given a DAG $G$, define 
the following
sets of d-separations:\footnote{
Note that
$(\rva.\perp_G
nde(\rva.)\cond pa(\rva.))$ and
$(\rva.\perp_G
nde(\rva.)-pa(\rva.)\cond pa(\rva.))$
are
equivalent
because
$H(\rva:\rvb, \rvc|\rvc)=
H(\rva:\rvb|\rvc)$.
}
\beq
DS(G)=\{(\rva.\perp_G\rvb.\cond\rvz.):
\text{ $\rva.,\rvb.,\rvz.$ are multinodes of $G$\}}
\;.
\eeq
\beq
DS_{min}(G)=\{(\rva.\perp_G
nde(\rva.)\cond pa(\rva.)):
\text{ $\rva.$ is a multinode of $G$\}}
\;.
\eeq

See Chapter \ref{ch-obs-equi}
for an example
where set $DS_{min}(G)$
is calculated for 
a particular DAG $G$.

\begin{claim}
For all DAGs $G$, $DS(G)=DS_{min}(G)$.
\end{claim}

Given a probability distribution  $P$, 
define the following
set of conditional independencies:

\beq
CI(P)=\{(\rva.\perp_P\rvb.\cond \rvz.):
\text{ $\rva.,\rvb.,\rvz.$ are multinodes of $P$\}}
\;,
\eeq


For a DAG $G$
and a probability
distribution $P$
compatible with $G$,
define  a map $\phi$
by
\beqa
\phi:DS_{min}(G) &\rarrow& CI(P)
\\
\phi: \rva.\perp_G nde(\rva.)\cond pa(\rva.)
&\mapsto&
\rva.\perp_P nde(\rva.)\cond pa(\rva.)
\eeqa
In general, this map
is 1-1 but not onto.


\begin{claim}
For a bnet 
with a DAG $G$
and a total probability distribution $P$,
the map $\phi$ is a bijection.
\end{claim}

$DS(G)$
does not fully specify a DAG.
DAGs with the same 
$DS(G)$ are said to be
{\bf d-separation equivalent}.
See Chapter \ref{ch-obs-equi}
for more info about 
d-separation equivalence.