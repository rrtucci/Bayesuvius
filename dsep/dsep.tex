\chapter{D-Separation}
\label{chap-dsep}
Before reading this chapter,
I  recommend
that you
read
Chapter \ref{ch-bnet-def}
on the definition of bnets.


A path $\gamma$ that
isn't a loop can have 
3 types of intermediate nodes $\rvx$ (
an intermediate node is one  which 
isn't either
of the two end nodes).
Suppose $\rva$ and $\rvb$
are the two neighbors of $\rvx$. Then
the 3 possible cases are:
\begin{enumerate}
\item {\bf mediator node:}
$(\rva\larrow\rvx\larrow \rvb)$
or
$(\rva\rarrow\rvx\rarrow \rvb)$
\item {\bf fork node:}
$(\rva\larrow\rvx\rarrow \rvb)$
\item {\bf collider node:}
$(\rva\rarrow\rvx\larrow \rvb)$
\end{enumerate}

We say that a non-loop path 
$\gamma=
(\rva:\rvb)$ with
end nodes $\rva, \rvb$
is {\bf blocked} 
or is {\bf d-separated}
by a multinode $\rvZ.$
if one or more 
of the following
statements is true:

\begin{enumerate}
\item 
There is a node $\rvx\in \rvZ.$
which is a mediator 
or a fork of $\gamma$.
\item
$\gamma$ contains a collider
node $\rvc$
and 
$(\rvc\cup de(\rvc))\cap\rvZ.=\emptyset$
(i.e., neither 
$\rvc$ nor 
any of the descendants of $\rvc$
is contained in $\rvZ.$)
\end{enumerate}

This definition of a blocked 
path is easy to remember
if one thinks 
of the following analogy
with pipes carrying a fluid.
Think of path
$\gamma$ as if it
were a pipe
carrying a fluid.
Think of
the nodes 
of $\gamma$ as junctions in the pipe.
If $\rvZ.$
intersects $\gamma$
at either a mediator
or a fork junction,
that blocks the pipe flow.
A collider junction $\rvc$
is like a blackhole 
or a huge leak.
Its presence
blocks passage
of the fluid
as long
as neither
$\rvc$
nor any of
the descendants 
of $\rvc$
are in $\rvZ.$.
If,
on the 
other hand,
$\rvc\in\rvZ.$,
or $\rvc'\in \rvZ.$
where $\rvc'\in de(\rvc)$,
then
that acts
as a complete
(in the case of $\rvc\in\rvZ.$)
or a partial 
(in the case of $\rvc'\in\rvZ.$)
bridge across the blackhole.

See Fig.\ref{fig-blocked-paths}
for some examples of
paths that are blocked or not blocked
by a multinode $\rvZ.$.

\begin{figure}[h!]
\beqa
\xymatrix{
\circ\ar[r]
&\circ\ar[r]
&\circ\ar[r]
&\circ\ar[r]
&\circ
}&\text{Not Blocked}
\\
\xymatrix{
\circ\ar[r]
&\color{red}\bullet\ar[r]
&\circ\ar[r]
&\circ\ar[r]
&\circ
}&\text{Blocked}
\\
\xymatrix{
\circ
&\circ\ar[l]\ar[r]
&\circ\ar[r]
&\circ\ar[r]
&\circ
}&\text{Not Blocked}
\\
\xymatrix{
\circ
&\color{red}\bullet\ar[l]\ar[r]
&\circ\ar[r]
&\circ\ar[r]
&\circ
}&\text{Blocked}
\\
\xymatrix{
\circ\ar[r]
&\circ\ar[r]
&\circ
&\circ\ar[l]\ar[r]
&\circ
}&\text{Blocked}
\\
\xymatrix{
\circ\ar[r]
&\circ\ar[r]
&\color{red}\bullet
&\circ\ar[l]\ar[r]
&\circ
}&\text{Not Blocked}
\\
\xymatrix{
\circ\ar[r]
&\circ\ar[r]
&\circ\ar[d]
&\circ\ar[l]\ar[r]
&\circ
\\
&&\color{red}\bullet
}&\text{Not Blocked}
\eeqa
\caption{Examples of 
paths that are blocked
or not blocked
by a multinode $\rvZ.$. Nodes
belonging to 
$\rvZ.$
are colored red.}
\label{fig-blocked-paths}
\end{figure}

Given 3 
disjoint multinodes 
$\rvA., \rvB., \rvZ.$
of a graph $G$,
we write 
$\rvA.\perp \rvB.|\rvZ.$
in $G$
iff there exists 
no path
$\gamma=(\rva:\rvb)$
where $\rva\in \rvA.$,
and
$\rvb\in\rvB.$
which is not 
blocked by $\rvZ.$.

The minimal 
Markov blanket (see Chapter
\ref{ch-mblanket})
of a node $\rva$
is the smallest 
multinode $\rvZ.$
such that $\rva\perp\rvb|\rvZ.$
for all $\rvb\notin \rva\cup\rvZ.$.

We are finally ready
to state the d-separation
theorem, without proof.

\begin{claim}
Suppose
$\rvA., \rvB., \rvZ.$
are disjoint multinodes
of a graph $G$.

If 
$\rvA.\perp \rvB.|\rvZ.$
in $G$, then
$P(B.|A., Z.)=P(B.|Z.)$.

\end{claim}

