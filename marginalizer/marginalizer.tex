\chapter{Marginalizer Nodes}
\label{ch-marginalizer}

Suppose we have a bnet node $\rvx$
that has multiple components.
For instance, suppose
$\rvx=(\rvx_1, \rvx_2, \rvx_3)$.
Then, we can define 
3 {\bf marginalizer nodes}
with TPMs, shown in blue, as follows

\beq\color{blue}
P(\rvx_i=\xi_i|\rvx=(x_1,x_2,x_3))= \delta(\xi_i, x_i)
\eeq
for $\xi_i\in S_{\rvx_i}$
and $i=1,2,3$.

Figs.\ref{fig-marginalizer}
and \ref{fig-marginalizer-lden}
show 3 different
styles for representing
marginalizer nodes graphically.
In this book,
we will use styles $(b)$ or $(c)$.
Style $(b)$ is the least 
ambiguous.

\begin{figure}[h!]
$$
\begin{array}{ccc}
\xymatrix@C=4pc{
&\rvy_1
\\
\rvx\ar[ru]
\ar[r]
\ar[rd]
&\rvy_2
\\
&\rvy_3
}
&
\xymatrix@C=4pc{
&\rvy_1
\\
\rvx\ar[ru]|-\circ^{\rvx_1}
\ar[r]|-\circ^{\rvx_2}
\ar[rd]|-\circ^{\rvx_3}
&\rvy_2
\\
&\rvy_3
}
&
\xymatrix@C=4pc{
&\rvy_1
\\
\rvx\ar[ru]|-<\circ
\ar[r]|-<\circ
\ar[rd]|-<\circ
&\rvy_2
\\
&\rvy_3
}
\\
(a)&(b)&(c)
\end{array}
$$
\caption{3 styles
for representing marginalizer nodes
in an arbitrary bnet.}
\label{fig-marginalizer}
\end{figure}


\begin{figure}[h!]
$$
\begin{array}{ccc}
\xymatrix@C=4pc{
&\rvy_1
\\
\rvx\ar[ru]_>>>{\alpha_1}
\ar[r]_>>>{\alpha_2}
\ar[rd]_>>>{\alpha_3}
&\rvy_2
\\
&\rvy_3
}
&
\xymatrix@C=4pc{
&\rvy_1
\\
\rvx\ar[ru]_>>>{\alpha_1}|-\circ^{\rvx_1}
\ar[r]_>>>{\alpha_2}|-\circ^{\rvx_2}
\ar[rd]_>>>{\alpha_3}|-\circ^{\rvx_3}
&\rvy_2
\\
&\rvy_3
}
&
\xymatrix@C=4pc{
&\rvy_1
\\
\rvx\ar[ru]_{\alpha_1}|-<\circ
\ar[r]_{\alpha_2}|-<\circ
\ar[rd]_{\alpha_3}|-<\circ
&\rvy_2
\\
&\rvy_3
}
\\
(a)&(b)&(c)
\end{array}
$$
\caption{3 styles
for representing marginalizer nodes
in an LDEN bnet (see Chapter \ref{ch-LDEN}).}
\label{fig-marginalizer-lden}
\end{figure}

