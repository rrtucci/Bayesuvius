\chapter{Markov Chains}

A Markov Chain is simply
a bnet with the graph structure 
of a chain. For example,
Fig.\ref{fig-mchain}
shows a chain with $n=4$ nodes.

\begin{figure}[h!]
\centering
$$\xymatrix{
\rvx_0\ar[r]
&\rvx_1\ar[r]
&\rvx_2\ar[r]
&\rvx_3
}$$
\caption{Markov chain with $n=4$ nodes.}
\label{fig-mchain}
\end{figure}

Because of its
 graph structure,
the transition prob matrix of each node
only depends on the state of the previous 
node:

\beq
P(x_t|(x_a)_{a\neq t})=P(x_t|x_{t-1})
\;,
\eeq
where $(x_a)_{a\neq t}$ are all
 the nodes except $x_t$ itself and
$t=1, 2, \dots, n-1$.

If there
exists a single
transition prob matrix $P_{\rvx_1|\rvx_0}$
such that

\beq
P(x_t|x_{t-1})=P_{\rvx_1|\rvx_0}
(x_t|x_{t-1})
\;
\eeq
for $t=1, 2,\dots, n-1$, 
then
we say 
that the Markov chain
is {\bf time homogeneous}.