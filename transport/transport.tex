\chapter{Transportability
of Causal Knowledge: COMING SOON}
\label{ch-transport}





Ref.\cite{pearl2011trans}

book by Brady Neal \cite{book-brady-neal}


\begin{figure}[h!]
$$
\begin{array}{ccc}
\xymatrix{
&\rvx\ar[dl]\ar[dr]
\\
\rvy\ar[rr]
&&\rvz
}
&
\xymatrix{
&\rvx\ar[dl]\ar[dr]
&\rvs=0\ar[l]\ar[d]
\\
\rvy\ar[rr]
&&\rvz
}
&
\xymatrix{
&\rvx\ar[dl]\ar[dr]
&\rvs=1\ar[l]\ar[d]
\\
\rvy\ar[rr]
&&\rvz
}
\\
\\
G&G_0=G
&G_1=G^*
\\
\\
P(x,y,z)
&P(x,y,z|\rvs=0)
&P(x,y,z|\rvs=1)
\\
&
=P(x,y,z)
&
=P^*(x,y,z)
\end{array}
$$
\caption{Example of selection bnets
$G_0=G$ and $G_1=G^*$.
$\rvs=0$ corresponds to population $\Sigma$
and $\rvs=1$ to population $\Sigma^*$.
For $G^*$, 
$P^*(x)\neq P(x)$, $P^*(y|x)=P(y|x)$
and $P^*(z|x,y)\neq P(z|x,y)$.}
\label{fig-sel-dia}
\end{figure}


transfer causal knowledge from
a {\bf source population} $\Sigma$ to a 
{\bf target population} $\Sigma^*$.

Given a bnet $G$, define a
{\bf selection diagram} $G_s$ for $s=0,1$
as a bnet 
formed by adding to $G$ a new root node $\rvs=s$
and new arrows pointing
from {\bf selection node}
$\rvs$ to one
or more {\bf target nodes} of $G$.
We'll call the set 
of target nodes 
of $\rvs$ the {\bf target set} $T_\rvs$.
$\rvs=0$ corresponds to population $\Sigma$
and $\rvs=1$ to population $\Sigma^*$.
Fig.\ref{fig-sel-dia}
shows an example
of a selection diagram $G_s$.
In that figure, the target set 
of $\rvs$ is 
$\{\rvx, \rvz\}$.
The TPM for the nodes in target set 
depends on the value of $\rvs$.
Nodes of $G_s$ not in
the target set have the same TPM as in $G$,
independent of the value of $\rvs$.
All this can be generalized so as to 
have more than one selection node,
with the target sets
of the selection nodes being disjoint, and such that a
selection node
can have more than 2 states.


{\bf Observational data} is data that
determines a probability distribution
without do operators in it.
 {\bf Experimental data} 
determines a probability distribution
 with do's in it.




{\bf Transport formulae}

Below, we will
use the following conventions
for selection diagrams
with a selection node $\rvs\in\bool$.
\selectionGraphs

\begin{itemize}
\item {\bf Trivial Transportability}

\begin{claim}
\decTransportTrivial
\end{claim}
\proof
See Claim \ref{cl-decTransportTrivial}.
\qed

\item{\bf Direct Transportability 
(aka External Validity)}

\begin{claim}
\decTransportDirect
\end{claim}
\proof
See Claim \ref{cl-decTransportDirect}.
\qed



\item {\bf S-Admissible Transportability}


\begin{claim}
\decTransportBox
\end{claim}
\proof
See Claim \ref{cl-decTransportBox}
\qed

\end{itemize}

More examples of transport formulae.

\begin{claim}
\decTransportOne
\end{claim}
\proof
See Claim \ref{cl-decTransportOne}.
\qed

\begin{claim}
\decTransportTwo
\end{claim}
\proof
See Claim \ref{cl-decTransportTwo}.
\qed

\begin{claim}
\decTransportThree
\end{claim}
\proof
See Claim \ref{cl-decTransportThree}.
\qed