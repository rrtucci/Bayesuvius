\chapter{Back Propagation
 (Automatic Differentiation)}

\section{General Theory}

\subsection{Jacobians}

Suppose
 $f:\RR^{nx}\rarrow \RR^{nf}$
and 

\beq
y=f(x)
\;.
\eeq
Then the Jacobian $\pder{y}{x}$
is defined as the matrix with entries\footnote{
Mnemonic for remembering 
order of indices: $i$ in numerator/$j$ in denominator
becomes index $i/j$ of Jacobian matrix.}

\beq
\left[\pder{y}{x}\right]_{i,j}=
\pder{y_i}{x_j}
\;.
\eeq




Jacobian of function composition.
Suppose $f:\RR^{nx}\rarrow\RR^{nf}$,
$g:\RR^{nf}\rarrow \RR^{ng}$. If

\beq
y=g\circ f (x)
\;,
\eeq
then

\beq
\pder{y}{x}
=
\pder{g}{f}
\pder{f}{x}
\;.
\eeq
Right hand side
of last equation 
is a product
of two matrices
so order of matrices is important.

\subsection{Bnets for function
composition, 
forward propagation and back propagation}



\begin{figure}[h!]
\centering
$$
\xymatrix{
\rvf^4
&
\rvf^3\ar[l]
&
\rvf^2\ar[l]
&
\rvf^1\ar[l]
&
\rvf^0\ar[l]
\\
&&\text{$(a)$ Composition}
\\
\ul{\pder{f^4}{x}}
&
\ul{\pder{f^3}{x}}\ar[l]
&
\ul{\pder{f^2}{x}}\ar[l]
&
\ul{\pder{f^1}{x}}\ar[l]
&
\ul{1}\ar[l]
\\
&&\text{$(b)$ Forward-p}
\\
\ul{1}\ar[r]
&
\ul{\pder{y}{f^3}}\ar[r]
&
\ul{\pder{y}{f^2}}\ar[r]
&
\ul{\pder{y}{f^1}}\ar[r]
&
\ul{\pder{y}{f^0}}
\\
&&\text{$(c)$ Back-p}
}
$$
\caption{bnets for function
composition,
forward propagation and back propagation
for $nf=5$ nodes.}
\label{fig-backp-abc}
\end{figure}


Let
\beq
y=f^4\circ f^3 \circ f^2 \circ f^1 (x)
\;.
\eeq
This function composition chain can be
represented by the bnet
Fig.\ref{fig-backp-abc}$(a)$
with TPMs

\beq\color{blue}
P(f^\mu|f^{\mu-1})=
\indi(f^\mu=f^\mu(f^{\mu-1}))
\;
\eeq
for $\mu=1, 2,3,4$.


Note that
\beqa
\pder{y}{x}&=&
\pder{y}{f^3}
\pder{f^3}{f^2}
\left[
\pder{f^2}{f^1}
\pder{f^1}{x}
\right]
\\
&=&
\pder{y}{f^3}
\left[
\pder{f^3}{f^2}
\pder{f^2}{x}
\right]
\\
&=&
\left[
\pder{y}{f^3}
\pder{f^3}{x}
\right]
\\
&=&
\pder{y}{x}
\;.
\eeqa
This forward propagation can be
represented by the bnet
Fig.\ref{fig-backp-abc}$(b)$
with node TPMs

\beq\color{blue}
P(\pder{f^{\mu+1}}{x}\cond
 \pder{f^{\mu}}{x})=
\indi(\pder{f^{\mu+1}}{x}=
\pder{f^{\mu+1}}{f^{\mu}}
\pder{f^{\mu}}{x} 
)\;
\eeq
for $\mu=1,2,3$.

Note that

\beqa
\pder{y}{x}&=&
\left[
\pder{y}{f^3}
\pder{f^3}{f^2}
\right]
\pder{f^2}{f^1}
\pder{f^1}{x}
\\
&=&
\left[
\pder{y}{f^2}
\pder{f^2}{f^1}
\right]
\pder{f^1}{x}
\\
&=&
\left[
\pder{y}{f^1}
\pder{f^1}{x}
\right]
\\
&=&
\pder{y}{x}\;.
\eeqa
This back propagation can be
represented by the bnet
Fig.\ref{fig-backp-abc}$(c)$
with node TPMs

\beq\color{blue}
P(
\pder{y}{f^{\mu}}
\cond 
\pder{y}{f^{\mu+1}}
)=
\indi(
\pder{y}{f^{\mu}}
=
\pder{y}{f^{\mu+1}}
\pder{f^{\mu+1}}{f^{\mu}}) 
\;
\eeq
for $\mu=2,1,0$.

$\pder{f^{\mu+1}}{f^{\mu}}$
is a Jacobian matrix
so the order of multiplication
matters. In forward prop,
it pre-multiplies,
and in back prop it post-multiplies.

\section{Application to Neural Networks}

\subsection{Absorbing $b^\lam_i$ into $w_{i|j}$.}

\begin{figure}[h!]
$$\xymatrix
{
\rvx\ar[r]\ar[rd]
&\rvh^0_2\ar[r]\ar[rd]
&\rvh^1_2\ar[r]\ar[rd] 
& Y_2
\\
& \rvh^0_1 \ar[r]\ar[ru]
&\rvh^1_1 \ar[r]\ar[ru]
& Y_1
\\
& \rvh^0_0 \ar[u]\ar@/^1pc/[uu]
&\rvh^1_0\ar[u]\ar@/^1pc/[uu]
& \rvY_0\ar[u]\ar@/^1pc/[uu]
}$$
\caption{Nodes $\rvh^0_0, \rvh^1_0, \rvY_0$
are all set to 1. They
allow us to absorb $b^\lam_i$
into the first column of
$w^\lam_{i|j}$.}
\label{fig-backp-abs}
\end{figure}

The TPMs, printed in blue,
for a NN bnet,
as given in Chapter \ref{ch-nn},
are as follows.

For all hidden layers $\lam=0, 1, \dots,
\Lambda-2$,

\beq\color{blue}
P(h^{\lam}_i\cond h^{\lam-1}_.)=
\delta\left(h^{\lam}_i,
\cala_i^\lam(\sum_j
 w^{\lam}_{i|j}
h^{\lam-1}_j + b^{\lam}_i)\right)
\eeq
for $i=0, 1, \ldots, nh(\lam)-1$.
For the output visible layer $\lam=\Lambda-1$:

\beq\color{blue}
P(Y_i\cond h^{\Lambda-2}_.)=
\delta \left(Y_i,
\cala_i^{\Lambda-1}(\sum_j w^{\Lambda-1}_{i|j}
h^{\Lambda-2}_j + b^{\Lambda-1}_i)\right)
\;
\eeq
for $i=0, 1, \ldots, ny-1$.


For each $\lam$, replace the matrix
 $w^\lam_{\cdot|\cdot}$ 
by 
the augmented matrix
$[b^\lam., w^\lam_{\cdot|\cdot}]$
so that the new
 $w^\lam_{\cdot|\cdot}$ satisfies

\beq
w^{\lam}_{i|0}=b_i^{\lam}
\;
\eeq

Let the nodes $\rvh_0^\lam$
for all $\lam$ and $\rvY_0$ be
root nodes (so no arrows 
pointing into them).
For each $\lam$, draw arrows from 
$\rvh_0^\lam$ to all other nodes 
in that same layer.
Draw arrows from 
$\rvY_0$ to all other nodes 
in that same layer.

After performing the
above steps,
the TPMs,
printed in blue,
for the NN bnet,
 are as follows:

For all hidden layers $\lam=0, 1, \dots,
\Lambda-2$,

\beq\color{blue}
P(h^\lam_0)=\delta(h^\lam_0,1)
\;,
\eeq
and

\beq\color{blue}
P(h^{\lam}_i\cond h^{\lam-1}_., h_0^\lam=1)=
\delta\left(h^{\lam}_i,
\cala_i^\lam(\sum_{j}
 w^{\lam}_{i|j}
h^{\lam-1}_j)\right)
\eeq
for $i=1, \ldots, nh(\lam)-1$.
For the output visible layer $\lam=\Lambda-1$:

\beq\color{blue}
P(Y_0)=\delta(Y_0,1)
\;,
\eeq
and

\beq\color{blue}
P(Y_i\cond h^{\Lambda-2}_., Y_0=1)=
\delta \left(Y_i,
\cala_i^{\Lambda-1}(\sum_j w^{\Lambda-1}_{i|j}
h^{\Lambda-2}_j)\right)
\;
\eeq
for $i=1, 2, \ldots, ny-1$.

\subsection{Bnets for function
composition, 
forward propagation and back propagation
for NN}

\begin{figure}[h!]
\centering
$$
\xymatrix{
\ul{\cala^3}
&
\ul{\calb^3}\ar[l]
&
\ul{\cala^2}\ar[l]
&
\ul{\calb^2}\ar[l]
&
\ul{\cala^1}\ar[l]
&
\ul{\calb^1}\ar[l]
&
\ul{\cala^0}\ar[l]
&
\ul{\calb^0}\ar[l]
&
\rvx\ar[l]
\\
&&\text{$(a)$}
\\
\ul{\pder{\cala^3}{x}}
&
\ul{\pder{\calb^3}{x}}\ar[l]
&
\ul{\pder{\cala^2}{x}}\ar[l]
&
\ul{\pder{\calb^2}{x}}\ar[l]
&
\ul{\pder{\cala^1}{x}}\ar[l]
&
\ul{\pder{\calb^1}{x}}\ar[l]
&
\ul{\pder{\cala^0}{x}}\ar[l]
&
\ul{\pder{\calb^0}{x}}\ar[l]
&
\ul{1}\ar[l]
\\
&&\text{$(b)$}
\\
\ul{1}\ar[r]
&
\ul{\pder{Y}{\calb^3}}\ar[r]
&
\ul{\pder{Y}{\cala^2}}\ar[r]
&
\ul{\pder{Y}{\calb^2}}\ar[r]
&
\ul{\pder{Y}{\cala^1}}\ar[r]
&
\ul{\pder{Y}{\calb^1}}\ar[r]
&
\ul{\pder{Y}{\cala^0}}\ar[r]
&
\ul{\pder{Y}{\calb^0}}\ar[r]
&
\ul{\pder{Y}{x}}
\\
&&\text{$(c)$}
}
$$
\caption{bnets for $(a)$ function
composition, $(b)$
forward propagation and $(c)$ back propagation
for a neural net with 4 layers (3 hidden and
output visible).}
\label{fig-backp-nn}
\end{figure}
.\\
From here on, we will rename $y$ above
by $Y=\hat{y}$ and 
consider samples $y[i]$ for 
$i=0, 1, \ldots, nsam-1$.
The Error (a.k.a. loss or cost function) is

\beq
\cale =\frac{1}{nsam}
\sum_{\sigma=0}^{nsam-1}\sum_{i=0}^{ny-1}
|Y_i-y_i[\sigma]|^2
\eeq
To perform simple gradient descent,
one uses:

\beq
(w_{i|j}^\lam)'=w_{i|j}^\lam
-\eta\pder{\cale}{w^\lam_{i|j}}
\;.
\eeq
One has

\beq
\pder{\cale}{w^\lam_{i|j}}=
\frac{1}{nsam}
\sum_{\sigma=0}^{nsam-1}\sum_{i=0}^{ny-1}
2(Y_i-y_i[\sigma])\pder{Y}{w^\lam_{i|j}}
\;.
\eeq
Define $\calb^\lam_i$ thus


\beq
\calb^\lam_i(h^{\lam-1})=
\sum_j w^{\lam}_{i|j}h^{\lam-1}_j
\;.
\label{eq-calb}
\eeq
Then

\beqa
\pder{Y}{w^\lam_{i|j}}&=&
\pder{Y}{\calb^\lam_i}
\pder{\calb_i^\lam}{w^\lam_{i|j}}
\\
&=&\pder{Y}{\calb^\lam_i}
h^{\lam-1}_j
\eeqa

\beqa
\pder{\cale}{w^\lam_{i|j}}&=&
\pder{\cale}{\calb^\lam_j}
\pder{\calb^\lam_j}{w^\lam_{i|j}}
\\
&=&
\pder{\cale}{\calb^\lam_j}
h^{\lam-1}_j
\;.
\eeqa
This suggest that
we can calculate 
the derivatives of the error
$\cale$
with respect to the weights
$w^\lam_{i|j}$ in two 
stages, using an intermediate 
quantity $\delta^\lam_j$:

\beq
\left\{
\begin{array}{l}
\delta^\lam_j=\pder{\cale}{\calb^\lam_j}
\\
\pder{\cale}{w^\lam_{i|j}}=\delta^\lam_j h^{\lam-1}_j
\end{array}
\right.
\eeq

To 
apply what we learned in the earlier
General Theory
section of this chapter, 
consider a NN with 4
layers (3 hidden, and the
output visible one). Define the
functions $f_i$
as follows:


\beq
f^0_i=x_i
\eeq

\beqa
\text{Layer 0:}&
f^1_i=\calb_i^0(x_i),&
f^2_i=\cala_i^0(\calb_i^0)
\\
\text{Layer 1:}&
f^3_i=\calb_i^1(\cala_i^0),&
f^4_i=\cala_i^1(\calb_i^1)
\\
\text{Layer 2:}&
f^5_i=\calb_i^2(\cala_i^1),&
f^6_i=\cala_i^2(\calb_i^2)
\\
\text{Layer 3:}&
f^7_i=\calb_i^3(\cala_i^2),&
f^8_i=\cala_i^3(\calb_i^3)
\eeqa




%\beq\color{blue}
%P(\cala^\lam|\cala^{\lam-1})=
%\indi(\cala^\lam=\cala^\lam\circ\calb^\lam(\cala^{\lam-1}))
%\;
%\eeq

%\beq\color{blue}
%P(\pder{\cala^{\lam+1}}{x}\cond
%\pder{\cala^{\lam}}{x})=
%\indi(\pder{\cala^{\lam+1}}{x}=
%\pder{\cala^{\lam+1}}{\calb^{\lam+1}}
%\pder{\calb^{\lam+1}}{\cala^{\lam}}
%\pder{\cala^{\lam}}{x} 
%)\;
%\eeq

See Fig.\ref{fig-backp-nn}. The
TPMs, printed in blue,
for the bnet $(c)$
for back propagation, are
as follows:

\beq\color{blue}
P(
\pder{Y}{\calb^{\lam}}
\cond 
\pder{Y}{\calb^{\lam+1}}
)=
\indi(
\pder{Y}{\calb^{\lam}}
=
\pder{Y}{\calb^{\lam+1}}
\pder{\calb^{\lam+1}}{\cala^{\lam}}
\pder{\cala^{\lam}}{\calb^{\lam}}) 
\;.
\label{eq-backp-nn-1}
\eeq
One has

\beq
\pder{\cala^{\lam}_i}{\calb^{\lam}_j}
=
D{\cala_i}^\lam(\calb_i^\lam)\delta(i,j)
\eeq
where $D{\cala}^\lam_i(z)$
is the derivative of 
$\cala^\lam_i(z)$.

From Eq.(\ref{eq-calb})

\beq
\calb^{\lam+1}_i(\cala^{\lam})=
\sum_j w^{\lam+1}_{i|j}\cala^{\lam}_j
\eeq
so

\beq
\pder{\calb_i^{\lam+1}}{\cala_j^{\lam}}
=w^{\lam+1}_{i|j}
\;.
\eeq
Therefore, Eq.(\ref{eq-backp-nn-1})
implies

\beq\color{blue}
P(
\pder{Y}{\calb^{\lam}_j}
\cond 
\pder{Y}{\calb^{\lam+1}_j}
)=
\indi(
\pder{Y}{\calb^{\lam}_j}
=
\sum_i
\pder{Y}{\calb_i^{\lam+1}}
D{\cala}^\lam_j(\calb^\lam_j)
w^{\lam+1}_{i|j}
) 
\;,
\eeq

\beq\color{blue}
P(
\pder{\cale}{\calb^{\lam}_j}
\cond 
\pder{\cale}{\calb^{\lam+1}_j}
)=
\indi(
\pder{\cale}{\calb^{\lam}_j}
=
\sum_i
\pder{\cale}{\calb_i^{\lam+1}}
D{\cala}^\lam_j(\calb^\lam_j)
w^{\lam+1}_{i|j}
) 
\;,
\eeq

\beq\color{blue}
\boxed{
P(
\delta^{\lam}_j
\cond 
\delta^{\lam+1}_j
)=
\indi(
\delta^{\lam}_j
=
\sum_i
\delta_i^{\lam+1}
D{\cala}^\lam_j(\calb^\lam_j))
w^{\lam+1}_{i|j}
)
}
\;.
\eeq

First delta of iteration, belonging to output 
layer $\lam=\Lambda-1$:

\beqa
\delta^{\Lambda-1}_j&=&
\pder{\cale}{\calb^{\Lambda-1}_j}
\\
&=&
\frac{1}{nsam}
\sum_{\sigma=0}^{nsam-1}\sum_{i=0}^{ny-1}
2(Y_i-y_i[\sigma])
D{\cala}^{\Lambda-1}_i(\calb^{\Lambda-1}_i)\delta(i,j)
\\
&=&
\frac{1}{nsam}
\sum_{\sigma=0}^{nsam-1}
2(Y_j-y_j[\sigma])
D{\cala}^{\Lambda-1}_j(\calb^{\Lambda-1}_j)
\eeqa

Cute expression for 
derivative of sigmoid function:
\beq
D\smoid(x)=
\smoid (x)(1-\smoid(x))
\eeq

\section{General bnets instead of Markov chains
induced by layered structure of NNs}

\beq
P(
\delta_\rvx
\cond 
(\delta_\rva)_{\rva\in ch(\rvx)}
)=
\indi(
\delta_\rvx
=
\sum_{\rva\in ch(\rvx)}
\delta_\rva
D{\cala}_\rvx(\calb_\rvx))
w_{\rva|\rvx}
)
\;
\eeq

Reverse arrows of original bnet
and define the TPM
of nodes of ``time reversed" bnet by

\beq\color{blue}
\boxed{
P(
\delta_\rvx
\cond 
(\delta_\rva)_{\rva\in pa(\rvx)}
)=
\indi(
\delta_\rvx
=
\sum_{\rva\in pa(\rvx)}
\delta_\rva
D{\cala}_\rvx(\calb_\rvx))
w^T_{\rvx|\rva}
)
}
\;
\eeq

