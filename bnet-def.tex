\section{Definition of a Bayesian Network}
\label{ch-bnet-def}

A Bayesian network (bnet)
consists of a DAG 
(Directed Acyclic Graph)
and a 
Transition 
Probability Matrix (TPM)
associated 
with each node
of the graph.
A TPM is often 
called a Conditional Probability
Table 
(CPT).

In 
this book,
random  variables are
 indicated by 
underlined letters and their values 
by non-underlined letters.
 Each node of a bnet is
 labelled by a random variable.
 Thus, $\rvx=x$ means that node 
$\rvx$ is in state $x$.

\hrule\noindent
{\bf Some sets of nodes 
associated 
with each node $\rva$
of a bnet}
\begin{itemize}
\item
$ch(\rva)=$ children of $\rva$.
\item
$pa(\rva)=$ parents of $\rva$.
\item
$nb(\rva)=pa(\rva)\cup ch(\rva)=$ 
neighbors of $\rva$.
\item
$de(\rva)=\cup_{n=0}^{\infty}ch^n(\rva)=$
$ch(\rva)\cup ch\circ ch(\rva)\cup\ldots$, 
descendants of $\rva$. 
\item
$an(\rva)=\cup_{n=0}^{\infty}pa^n(\rva)=$
$pa(\rva)\cup pa\circ pa(\rva)\cup\ldots$, 
ancestors of $\rva$.
\end{itemize}
\hrule
In this book,
we will use 
$\rva.$
to indicate
a {\bf multi-node (node set,
node array)} $\rva.=
(\rva_j)_{j=0, 1, \ldots , na-1}$.
We will often
treat multinodes as if
they were sets, and
combine them with
the usual
set
operators.
For instance,
for two
multinodes $\rva.$
and $\rvb.$,
we define
$\rva.\cup\rvb.$,
$\rva.\cap\rvb.$,
$\rva.-\rvb.$
and
$\rva.\subset\rvb.$
in the obvious way.
We 
will indicate
a singleton set (single
node multi-node) $\rva.=\{\rva\}$
simply by $\rva.=\rva$.
For instance,
$\rva.-\rvb=\rva.-\{\rvb\}$.
\hrule

The TPM of a node
$\rvx$ of a bnet
is a matrix of
probabilities 
$P(\rvx=x|pa(\rvx)=a.)$.

A bnet
with nodes $\rvx.$
represents
a probability
distribution

\beq
P(x.)=
\prod_j
P(\rvx_j=x_j|pa(\rvx_j) =pa(x_j))
\;.
\eeq


