\chapter{Turbo Codes}
This chapter is based
 on Ref.\cite{mackay98}.

In this chapter, vectors
with $n$ components
will be indicated by 
an $n$ superscript. For example,
$a^n=(a_0, a_1, \ldots, a_{n-1})$.



Consider
an  n-letter message $u^n=
(u_0, u_1, \dots, u_{n-1})$,
where for all $i$,
$u_i\in \cala$ is an element of
an alphabet $\cala$,
and where 
for all $i$, the $\rvu_i$ are i.i.d..
Suppose $u^n$
is encoded 
deterministically in
two different ways, $e_1(u^n)$
and $e_2(u^n)$.
After passing through 
the same memoryless channel, the variables
$u^n,e_1, e_2$
become $\tilu^n, \tile_1,
\tile_2$, respectively.
The letter $u$ stands 
for unencoded, and $e$ for
encoded. Quantities with a tilde
$\tilu^n, \tile_1,
\tile_2$
occur after channel passage
and are visible (measurable). Quantities
without a tilde
$u^n, e_1,
e_2$ are hidden (unmeasurable).


The situation just described
can be represented by
the bnet Fig.\ref{fig-turbo-ext},
or by its abridged version
Fig.\ref{fig-turbo-simple}.
But note that the abridged version does not
show explicitly that the
$u_i$ are i.i.d. or that the 
 channel is memoryless (i.e., that
the $u_i$ for all $i$
pass independently
through the channel).

\begin{figure}[h!]
\centering
$$\xymatrix{
\rvu_0\ar[rr]\ar[rddd]\ar[rdddd]
&&\tilu_0\\
\rvu_1\ar[rr]\ar[rdd]\ar[rddd]
&&\tilu_1\\
\rvu_2\ar[rr]\ar[rd]\ar[rdd]
&&\tilu_2\\
&\rve_1\ar[r]&\tile_1\\
&\rve_2\ar[r]&\tile_2
}$$
\caption{Turbo coding Bnet
representing a message
being 
encoded two
different ways 
and then the
original 
message and the 2
encodings
pass through a memoryless channel.
}
\label{fig-turbo-ext}
\end{figure}

\begin{figure}[h!]
\centering
$$\xymatrix{
\rvu^n\ar[rr]\ar[rd]\ar[rdd]
&&\tilu^n\\
&\rve_1\ar[r]&\tile_1\\
&\rve_2\ar[r]&\tile_2
}$$
\caption{Abridged version of Fig.\ref
{fig-turbo-ext}.}
\label{fig-turbo-simple}
\end{figure}

Define 


\beq
x=(u^n, e_1, e_2)
\eeq
and

\beq
\tilx=(
\tilu^n,
\tile_1, 
\tile_2)
\;.
\eeq

Fig.\ref{fig-turbo-ext} 
implies that

\beq
P(x, \tilx)
=
P(\tilu^n|u^n)\left[\prod_{r=1,2}
 P(\tile_r|e_r)
P(e_r|u^n)\right]
P(u^n)
\;.
\eeq
Because the $u^n$ are i.i.d.,

\beq\color{blue}
P(u^n)=
\prod_i P(u_i)
\;.
\eeq
Because the channel is memoryless, 

\beq\color{blue}
P(\tilu^n|u^n)=\prod_i P(\tilu_i|u_i)
\;.
\eeq
Because the encoding
is deterministic, we must have
for  $r=1,2$
\beq\color{blue}
P(e_r|u^n)=\delta(e_r, e_r(u^n))
\;.
\eeq

Define the belief functions

\beq
BEL_i=BEL_i(\rvu_i=a)=P(\rvu_i=a|\tilx)
\;.
\eeq
The best estimate of $u_j$
given all visible evidence $\tilx$
is

\beq
\hat{u}_i=
\argmax_{u_i}BEL_i(u_i)
\;.
\eeq


Define the probability functions
\beq
\pi_i=\pi_i(u_i)=P(u_i)
\;,
\eeq
and the likelihood functions

\beq
\lam_i=\lam_i(u_i)=P(\tilu_i|u_i)
\;.
\eeq

For $r=1,2$, define the Kernel functions

\beq
K_r=K_r(u^n)=
P(\tile_r|e_r=e_r(u^n))
\;.
\eeq

In this book,
$\caln(!a)$ denotes
a normalization constant 
that does not depend
on $a$. Define
\beq
\caln_i=\caln(!u_i)
\;.
\eeq

\begin{claim}
\beq
BEL_i=\caln_i\lam_i\pi_i
\calt_i^{K_1K_2}
[\prod_{j\neq i} \lam_j\pi_j]
\label{eq-bel-exact}
\;,
\eeq
where $\calt^K_i(\cdot)$ with $K=K_1K_2$
is an operator (transform)
that acts on functions of $u^n$:

\beq
\calt_i^K(\cdot)=
\sum_{u^n}\delta(u_i,a)K(u^n)(\cdot)
\;.
\eeq

\end{claim}
\proof

\beqa
\lefteqn{P(\rvu_i=a|\tilx) =}\nonumber\\
&=&\sum_{x}\delta(u_i, a)P(x|\tilx)\\
&=&\sum_{x}\delta(u_i, a)
\frac{P(\tilx|x)P(x)}{P(\tilx)}\\
&=&\caln(!a)
\sum_{x}\delta(u_i, a)
P(\tilx|x)P(x)\\
&=&
\caln(!a)
\sum_{x}\delta(u_i, a)P(u^n)
\left[
\prod_{r=1,2}
P(\tile_r|e_r)\delta(e_r,e_r(u^n))
\right]
\prod_j P(\tilu_j|u_j)\\
&=&\caln(!a)
\lam_i(a)\pi_i(a) R
\;, 
\eeqa
where

\beqa
R&=&
\sum_{u^n}\delta(u_i, a)
\left[
\prod_{r=1,2}
P(\tile_r|e_r(u^n))\right]
\prod_{j\neq i} P(\tilu_j|u_j)P(u_j)\\
&=&
\sum_{u^n}\delta(u_i, a)
\left[
\prod_{r=1,2}
K_r(u^n)\right]
\prod_{j\neq i} \lam_j(u_j)\pi_j(u_j)\\
&=&\calt_i^{K_1K_2}
[\prod_{j\neq i} \lam_j(u_j)\pi_j(u_j)]
\;.
\eeqa
Hence

\beq
BEL_i(a)=\caln(!a)\lam_i(a)\pi_i(a)
\calt_i^{K_1K_2}
[\prod_{j\neq i} \lam_j(u_j)\pi_j(u_j)]
\;.
\eeq
\qed


\section{Decoding Algorithm}
The Turbo algorithm for
decoding the encode message
 is as follows.
For $m=0$, let

\beq
\pi_j^{(0)}(u_j) = \frac{1}{n_{\rvu_j}}
\;.
\eeq
Then for $m=1, 2, \dots $, let

\beq
\pi_i^{(m)}=\caln_i
\calt_i^{K_{m\%2}}
[\prod_{j\neq i} \lam_j\pi_j^{(m-1)}]
\;,
\eeq
 where $m\%2=1$ if $m$ is odd and 
$m\%2=2$ if $m$ is even. 
Furthermore, for $m>0$, let

\beqa
BEL_i^{(m)}&=&\caln_i \lam_i
\pi_i^{(m-1)}\pi_i^{(m)}
\\
&=&
\caln_i\lam_i \pi_i^{(m-1)}\calt_i^{K_{m\%2}}
[\prod_{j\neq i} \lam_j
\pi_j^{(m-1)}]
\;.
\label{eq-bel-approx}
\eeqa
As $m\rarrow\infty$, 
$BEL_i^{(m)}$ given 
by Eq.(\ref{eq-bel-approx}) is
 expected to 
converge to the the exact 
$BEL_i$ given
by Eq.(\ref{eq-bel-exact}).

Turbo decoding 
can be represented by the bnets 
Figs.\ref{fig-turbo-decode}
and \ref{fig-turbo-decode-ext}.


\begin{figure}[h!]
\centering
$$\xymatrix{
\ul{\tilu}^n
\ar[r]
\ar@/^1pc/[rr]
\ar@/^1pc/[rrr]
\ar@/^1pc/[rrrr]
\ar@/^1pc/[rrrrr]
&\rvd_i^{(1)}
&\rvd_i^{(2)}
&\rvd_i^{3)}
&\rvd_i^{(4)}
&\rvd_i^{(5)}
\\
\ul{\tile}_1
\ar[ur]\ar[urrr]\ar[urrrrr]\\
\ul{\tile}_2
\ar[uurr]\ar[uurrrr]
}$$
\caption{Bnet
describing Turbo code
generation of $BEL_i^{(m)}(a)$
for $m=1,2, \ldots$.}
\label{fig-turbo-decode}
\end{figure}

\begin{figure}[h!]
\centering
$$\xymatrix{
&
&\ul{BEL}^{n(1)}(\cdot)
&\ul{BEL}^{n(2)}(\cdot)
&\ul{BEL}^{n(3)}(\cdot)
&\ul{BEL}^{n(4)}(\cdot)
\\
\rvu^n
&\ul{\pi}^{n(0)}(\cdot)\ar[r]\ar[ur]
&\ul{\pi}^{n(1)}(\cdot)\ar[r]\ar[u]\ar[ur]
&\ul{\pi}^{n(2)}(\cdot)\ar[r]\ar[u]\ar[ur]
&\ul{\pi}^{n(3)}(\cdot)\ar[r]\ar[u]\ar[ur]
&\ul{\pi}^{n(4)}(\cdot)\ar[u]
\\
\ul{\tile}_1\ar[urr]\ar[urrrr]\\
\ul{\tile}_2\ar[uur]\ar[uurrr]\ar[uurrrrr]
\\
\ul{\tilu}^n\ar[r]
&\ul{\lam}^{n}(\cdot)
}$$
\caption{
Bnet
describing Turbo code
generation of $BEL^{n(m)}(\cdot)$ and
$\pi^{n(m)}(\cdot)$ 
for $m=0,1,2 \ldots$.
The following arrows 
were not drawn
for clarity:
Arrows pointing from node
 $\ul{\lam}^n(\cdot)$ to nodes 
$\ul{\pi}^{n(m)}(\cdot)$ 
and $\ul{BEL}^{n(m)}(\cdot)$ for 
$m=0,1,2, 
\ldots$.
}
\label{fig-turbo-decode-ext}
\end{figure}

The node TPMs, printed in blue,
for Fig.\ref{fig-turbo-decode}, 
are given by:


\beq\color{blue}
P(d_i^{(m)}=a\cond
\tilu^n, \tile_{m\%2})=BEL^{(m)}_i(a)
\;.
\eeq

The TPMs, printed in blue,
for Fig.\ref{fig-turbo-decode-ext}, 
are given by:



\beq\color{blue}
P((\lam^n)'(\cdot)|\tilu^n)=
\delta((\lam^n)'(\cdot),
 \lam^n(\cdot))
\eeq

\beq\color{blue}
P(\pi^{n(m)}(\cdot)|\lam^n(\cdot), 
\pi^{n(m-1)}(\cdot), \tile_{m\%2})=
\prod_i\prod_{u_i}
\delta(\pi_i^{(m)}(u_i),
\caln_i
\calt_i^{K_{m\%2}}
[\prod_{j\neq i} \lam_j\pi_j^{(m-1)}])
\eeq

\beq\color{blue}
P(BEl^{n(m)}(\cdot)|\lam^n(\cdot),
\pi^{n(m)}(\cdot),
\pi^{n(m-1)}(\cdot))=
\prod_i\prod_{u_i}
\delta(
BEL_i(u_i),
\caln_i \lam_i
\pi_i^{(m-1)}\pi_i^{(m)})
\eeq


\section{Message Passing 
Interpretation of Decoding Algorithm}

Ref.\cite{mackay98} shows that
the  Turbo code
decoding algo can be
interpreted
as an 
application of Message Passing.
We leave all talk of Message Passing to
a separate
chapter, Chapter \ref{ch-mpass}.