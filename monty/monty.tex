\chapter{Monty Hall Problem}
\begin{figure}[h!]
\centering
$$\xymatrix{
\rvc\ar[dr]&&\rvy\ar[dl]\\
&\rvm&
}$$
\caption{Monty Hall Problem.}
\label{fig-monty}
\end{figure}

Mr. Monty Hall, host of the 
game show \qt{Let’s Make a Deal},
 hides a car behind one of 
three doors and a goat 
behind each of the other two.
 The contestant picks Door No. 1,
 but before opening it, Mr. Hall 
opens Door No. 2 to reveal a goat. 
Should the contestant stick with No. 1 
or 
switch to No. 3?

The Monty Hall problem can be 
modeled by the bnet 
Fig.\ref{fig-monty}, where
\begin{itemize}
\item
$\rvc$= the door behind which the car actually is.
\item
$\rvy$= the door opened by you
 (the contestant), on your 
first selection.
\item
$\rvm$= the door opened by Monty (game host)
\end{itemize}

We label the doors 1,2,3 so
 $val(\rvc)=val(\rvy)=val(\rvm)=\{1,2,3\}$.

The TPMs, printed in blue,
for this bnet, are as follows:

\beq\color{blue}
P(c)=\frac{1}{3}\text{ for all $c$}
\eeq

\beq\color{blue}
P(y)=\frac{1}{3}\text{ for all $y$}
\eeq

\beq\color{blue}
P(m|c,y)=\indi(m\neq c)\left[
\frac{1}{2}\indi(y=c)
+
\indi(y\neq c)\indi(m\neq y)\right]
\eeq

It's easy to show that the above
 node probabilities imply that
\beq
P(c=1|m=2,y=1)=\frac{1}{3}
\eeq

\beq
P(c=3|m=2,y=1)=\frac{2}{3}
\eeq

So you are twice as likely to
 win if you switch your final
 selection to be the door 
which is neither 
your first choice nor Monty's choice.

The way I justify this to myself
is: Monty gives you a
 piece of information.
If you don't switch your choice,
you are wasting that info, whereas
if you switch, you are using the info.


