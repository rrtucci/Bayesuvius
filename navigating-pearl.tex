\chapter*{Navigating 
the ocean of Judea Pearl's Books}
\addcontentsline{toc}{chapter}{Navigating 
the ocean of Judea Pearl's Books}
\label{ch-nav-pearl}
Many
of the
greatest ideas 
in the bnet field 
were invented by Judea Pearl
and his collaborators.
Thus, this book is 
heavily indebted to
those great scientists.
Those ideas have had no clearer
and more generous 
expositor than Judea Pearl
himself.

Pearl has written
4 books that I have used
in writing Bayesuvius.
His 
1988 book Ref.\cite{pearl-1988book}
dates back to his pre-causal period.
That book I used to learn
about topics such as
d-separation, belief propagation,
Markov-blankets, and noisy-ORs.
3 other books that  he  wrote later,
in his causal period, 
are:
\begin{enumerate}
\item
In 2000 (1st ed.), and 2013 (2nd ed.),
Pearl published what
is so far
his most technical
and exhaustive book
on the subject of causality,
Ref.\cite{pearl-2013book}.
\item
In 2016,
he released 
together
with Glymour and Jewell,
a less advanced ``primer"
on causality, Ref.\cite{pearl-primer}.
\item
In 2018, 
he released 
together with
Mackenzie his
lovely  ``The Book of Why",
 Ref.\cite{book-why}.
\end{enumerate}
Those 3 books I used to learn
about causality topics
such as Do Calculus,
backdoor and frontdoor
adjustment formulae,
linear 
deterministic 
bnets with exogenous noise,
and counterfactuals.

A micro poem written by me
to celebrate Judea Pearl and 
his work:
 
\section*{I, Robot}
\begin{verse}
Let other robots {\tt talk()},\\
while I,\\
{\tt talk()}, {\tt do()} and {\tt imagine()}.
\end{verse}
