\chapter*{CI-2-3 track}
\addcontentsline{toc}{chapter}{CI-2-3 track}
\label{ci-track}

\begin{figure}[h!]
\centering
\includegraphics[width=5in]
{godzilla-kk-doge-nn-ci.jpg}
\caption{CI meme}
\label{fig-godzilla-kk-doge}
\end{figure}

As discussed in Chapter \ref{ch-counterf},
Judea Pearl has proposed 3 rungs
of Causal Inference (CI).
This book covers all 3 rungs.

Confusingly,
it has become common to use the term
CI to refer to only the highest
2 rungs of the CI hierarchy; i.e,
rung 2 (do operations)
and rung 3 (imagining/counterfactual thinking).
Also confusingly, rung 1
uses causal diagrams and
is often referred to as ``inference",
so it could reasonably have been defined as the whole
of CI, but Pearl has defined
the CI hierarchy to include two more rungs.
To patch over this linguistic confusion,
I sometimes refer to rung 1 as ``prediction",
or as ``predictive inference"
instead of calling it merely ``inference".
Also, when I want to be precise,
I use the term ``CI-2-3" to
refer to CI restricted to only rungs 2 and 3.


Here is a subset of chapters
that I call
the CI-2-3 track,
that are devoted mostly to rungs 2 and 3.


\begin{enumerate}
\item \nameref{ch-bdoor}
\item \nameref{ch-berkson}
\item \nameref{ch-counterf}
\item \nameref{ch-dec-2-3}
\item \nameref{ch-did}
\item \nameref{ch-do-calc}
\item \nameref{ch-do-calc-proofs}
\item \nameref{ch-dsep}
\item \nameref{ch-dsep-lden}
\item \nameref{ch-fwl-theo}
\item \nameref{ch-fdoor}
\item \nameref{ch-good-causal-fit}
\item \nameref{ch-granger-c}
\item \nameref{ch-inst-ineq}
\item \nameref{ch-instrumental}
\item \nameref{ch-mediation}
\item \nameref{ch-meta-learners}
\item \nameref{ch-personalized}
\item \nameref{ch-personalized-test}
\item \nameref{ch-pot-out}
\item \nameref{ch-reg-dis}
\item \nameref{ch-seq-bdoor}
\item \nameref{ch-sb-removal}
\item \nameref{ch-simpson}
\item \nameref{ch-syn-con}
\item \nameref{ch-targeted-est}
\item \nameref{ch-transport}
\item \nameref{ch-uplift}
\end{enumerate}
