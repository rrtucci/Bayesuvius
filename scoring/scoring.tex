\chapter{Scoring the Nodes of a Learned Bnet}
\label{ch-scoring}

Chapter \ref{ch-struc-learn}
discusses how 
to learn a bnet  from data.
Many algorithms for
doing this require scoring
how well
a particular bnet fits
the data.
This chapter 
is an introduction to
such scoring.
 

In this chapter,
scores are defined 
so that a higher score
is a better fit.
By taking
the negative
of such a score,
one can 
always get a score
such that a lower score is a
 better fit.

There are 2 main types of
bnet scores: Maximum Likelihood (ML) scores,
and Shannon Information Theory (SIT)
scores.
For ML scores,
one takes
the log
of a maximum
likelihood function $P(x|\theta)$:

\beq
\text{ML-score}=\ln(P(x|\theta))
\;,
\eeq
and for a SIT score, 
one averages over the ML score
to get a negative entropy:

\beq
\text{Info-score}
=\sum_x P(x|\theta)\ln P(x|\theta)
=
-H(P_{\rvx|\theta})
\;.
\eeq
Thus, maximizing a 
likelihood function
and minimizing its entropy both
have the
same goal,
to find 
good estimates of the
hidden parameters $\theta$. 

\section*{
Probability Distributions
and Special Functions}

While
writing
this chapter,
I briefly consulted 
the following Wikipedia articles
about
the definitions
and properties of
certain
probability
distributions
and special
functions.
\begin{itemize}
\item
Categorical Distribution
 Ref.\cite{wiki-categorical}
\item
Multinomial Distribution
 Ref.\cite{wiki-multi-dist}
\item
Dirichlet Distribution 
Ref.\cite{wiki-diri}
\item 
Multivariate Normal Distribution
Ref.\cite{wiki-multi-normal}

\item
Beta function 
Ref.\cite{wiki-beta-fun}
\item
Multinomial Coefficients
 Ref.\cite{wiki-multi-thm}
\item 
Gamma Function
Ref.\cite{wiki-gamma-fun}
\end{itemize}

Here are a few 
results 
from those Wikipedia
articles that will be used later on.

Below,
we will abbreviate
$q_+=\sum_i q_i$, and
$q.=(q_0, q_1, \ldots, q_{nq-1})$
for various quantities $q$

{\bf Gamma function}. If $n>0$ is an integer,
\beq
\Gamma(n+1)=n!
\eeq

The {\bf multivariate Beta function} 
is defined by
\beq
B(\alp.)=
\frac{\prod_k\Gamma(\alp_k)}
{\Gamma(\alp_+)}
\eeq
where $\alp_k>0$ for all $k$.

The {\bf multinomial coefficient}
is defined by
\beq
C(N.)=
\frac{N_+!}{\prod_k N_k!}
\eeq
where $N_k$ are non-negative
integers.

The {\bf inverse of the multinomial 
coefficient} will
be denoted by
\beq
CI(N.)
=
\frac{1}{C(N.)}
=
\frac{\prod_k N_k!}{N_+!}
\eeq

The {\bf Categorical Distribution}
is defined by
\beq
Cat(x; \pi.)
=\pi_x=
\prod_{k} \pi_k^{\indi(k=x)}
\eeq
for $k, x\in S_\rvx$,
where 
$\pi.$ is a
probability dist.(i.e., $\pi_k\geq 0$,
and $\pi_+=1$).



The {\bf Multinomial Distribution}
is defined by
\beq
Mul(N.;\pi., N)
=
C(N.)
\prod_k \pi_k^{N_k}
\;,
\eeq
where $N_k$
is a non-negative
integer for all $k$,
and $\pi.$
is a probability dist.
Mul() satisfies:
\beq
E[\ul{N_k}]=N\pi_k
\;.
\label{eq-exp-val-mul}
\eeq

The {\bf Dirichlet Distribution}
is defined by
\beq
Dir(\pi.;\alp.)=
\frac{1}{B(\alp.)}
\prod_k
\pi_k^{\alp_k-1}
\eeq
where $\alp_k>0$
for all $k$,
and $\pi.$
is a probability dist.
The
$\alp.$ are called 
{\bf concentration parameters}.
Dir() satisfies:
\beq
E[\ul{\pi_k}]=\frac{N_k}{N_+}
\;.
\label{eq-exp-val-dir}
\eeq


\hrule\noindent
Dir() is {\bf conjugate prior} of  Mul()

Note that

\beq
Mul(N.;\pi., N)Dir(\pi.;\alp.)
=
\calk( N.,\alp.)
Dir(\pi.;N.+\alp.)
\;,
\eeq
where

\beq
\calk(N., \alp.)=
\frac{B(N.+\alp.)}
{CI(N.)B(\alp.)}
\;.
\eeq
\hrule\noindent
Dir() swappable with Mul()
for large concentration parameters

Note that if $N_k$ is a positive  integer
and
$\alp_k=N_k+1$ for all $k$, then

\beqa 
Dir(\pi.;\alp_k=N_k+1)
&=&
C(N.)\prod_k \pi_k^{N_k}
\\
&=&
Mul(N.; \pi., N_+)
\;.
\eeqa

\section*{Single node with no parents}
In this
section,
we consider a 
learned bnet
consisting of
a single node with
no parents.
We will consider
arbitrary
learned
bnets
in the next section.
But we start with 
this simplified
case so as to reduce
the number of indices
in most quantities
from 3 to 1.
All the results
that we derive
in this section
will
be used in the next 
section
after 
adding the extra indices.
This way,
we will
avoid
carrying
the extra indices
throughout
the intermediate steps.

For state
$k\in\{0, 1, \ldots, 
nk-1\}$
of a single node $\rvx$, let


$\rvN_k$=
current count number
(an integer, data)

$\ul{\pi}.$=
a probability dist, the TPM for the node

$\ul{\alp}_k$=
prior count
number



\begin{figure}[h!]
$$
\xymatrix{
\rvN.&\ul{\pi.}\ar[l]&
\ul{\alp.}\ar[l]
}$$
\caption{
For a bnet
consisting
of a single node
with no parents, 
this is a
Markov chain of
current counts ($\rvN.$), TPM ($\ul{\pi.}$),
and
prior counts ($\ul{\alp.}$)
.}
\label{fig-n-pi-alp-single}
\end{figure}

Consider
the Markov chain
bnet
of Fig.\ref{fig-n-pi-alp-single},
with the following
TPMs, given in blue.

\beq\color{blue}
P(N.|\pi.)=
Mul(N.;\pi., N_+)
\eeq

\beq\color{blue}
P(\pi.|\alp.)=Dir(\pi.;\alp.)
\eeq

It follows that
\beqa
P(N., \pi.|\alp.)
&=&
P(N.|\pi.)P(\pi.|\alp.)
\\
&=&
Mul(N.; \pi., N_+) Dir(\pi.;\alp.)
\\
&=&
\calk(N., \alp.)
Dir(\pi.;N.+\alp.)
\;.
\label{eq-calk-dir}
\eeqa 
From
Eq.(\ref{eq-exp-val-dir})
for the expected value of Dir(),
we get

\beq
\hat{\pi}.= E[\ul{\pi}.]=
\frac{N.+\alp.}
{N_+ + \alp_+}
\;.
\eeq
Integrating
both 
sides
of Eq.(\ref{eq-calk-dir})
over $\pi.$, 
we find that

\beq
P(N.|\alp.)=
\calk(N., \alp.)
\;.
\eeq

If $N_k>>1$ for all $k$,
then the Dir()
in Eq.(\ref{eq-calk-dir})
can be swapped by a Mul()


\beq
P(N., \pi.|\alp.)
\approx
\calk(N., \alp.)
Mul(N.+\alp.; \pi., N_+ + \alp_+)
\;.
\eeq
Therefore,

\beqa
P(N.|\pi., \alp.)
&=&
\frac{P(N., \pi., \alp.)}
{P(N.,\alp.)}
\\
&=&
Mul(N.+\alp.; \pi., N_+ + \alp_+)
\;.
\eeqa

\begin{claim}
\beqa
\ln P(N.|\hat{\pi}., \alp.)
&=&
-
(N_+  +\alp_+)
H\left(\frac{N.+\alp.}
{N_+  +\alp_+}\right)
+
\ln C(N.+\alp.)
\\
&>&
-
(N_+  +\alp_+)
H\left(\frac{N.+\alp.}
{N_+  +\alp_+}\right)
-\frac{1}{2}(nk-1)\ln N_+
\eeqa
\end{claim}
\proof


\beqa
\ln P(N.|\hat{\pi}., \alp.)
&=&
\sum_k (N_k+\alp_k)
\ln \hat{\pi}_k
+
\ln C(N.+\alp.)
\\
&=&
\sum_k (N_k+\alp_k)
\ln 
\frac{N_k+\alp_k}
{N_+  +\alp_+}
+
\ln C(N.+\alp.)
\\
&=&
-
(N_+  +\alp_+)
H\left(\frac{N.+\alp.}
{N_+  +\alp_+}\right)
+
\ln C(N.+\alp.)
\eeqa

Recall Stirling's approximation
of a factorial
\beq
\ln n!
\approx 
(n+\frac{1}{2})\ln n -n
\;.
\eeq
Assume $N_k>>1$ for all $k$.
Applying Stirling's
approximation
to all factorials in $C(N)$,
we get


\beqa
\ln C(N.)
&\approx &
(N_+ + \frac{1}{2})\ln N_+ -
N_+
-
\sum_k
\left[
(N_k+\frac{1}{2})\ln N_k
-N_k
\right]
\\
&=&
(N_+ + \frac{1}{2})\ln N_+ 
-
\sum_k
(N_k+\frac{1}{2})\ln N_k
\;.
\eeqa
Next assume that

\beq
N_k\approx \frac{N_+}{nk}
\;.
\eeq
Then

\beqa
\ln C(N.)
&=&
(N_+ + \frac{1}{2})\ln N_+
-nk(\frac{N_+}{nk} +
 \frac{1}{2})[\ln N_+
-\ln nk]
\\
&=&
-\frac{1}{2}(nk-1)\ln N_+
+(N_+ +\frac{nk}{2})\ln nk
\\
&>&
-\frac{1}{2}(nk-1)\ln N_+
\;.
\eeqa
\qed


\section*{
Multiple nodes with any number of parents}

In the previous
section,
we considered a 
bnet consisting of a single node
with no parents,
so we only
needed a single index $k$
for the states of the single node.
In this section,
we consider an arbitrary 
bnet with multiple nodes
each of which may have
multiple parents. Most
of the results in
the previous section
are valid for the 
general case if we make the
following replacements:
$\pi.\rarrow \pi^i_\dotbarmu$
$N.\rarrow N^i_\dotmu$
$\alp.\rarrow \alp^i_\dotmu$.
Upon this replacement, 
Fig.\ref{fig-n-pi-alp-single}
becomes
Fig.\ref{fig-n-pi-alp-many}.
The TPMs, printed in blue, of the new 
Markov chain, are as follows: 


\begin{figure}[h!]
$$
\xymatrix{
\ul{N^i_\dotmu}
&\ul{\pi^i_\dotbarmu}\ar[l]
&
\ul{\alp^i_\dotmu}\ar[l]
}$$
\caption{
Generalization
of Fig.\ref{fig-n-pi-alp-single}.
For a bnet with multiple nodes
each of which may have multiple parents,
this is  a
Markov chain of
current counts ($\ul{N^i_\dotmu}$),
TPM ($\ul{\pi^i_\dotbarmu}$),
and
prior counts ($\ul{\alp^i_\dotmu}$)
.}
\label{fig-n-pi-alp-many}
\end{figure}

\beq\color{blue}
P(N^i_\dotmu|\pi^i_\dotbarmu)=
Mul(N^i_\dotmu;
\pi_\dotbarmu^i, N^i_\plusmu)
\eeq

\beq\color{blue}
P(\pi^i_\dotbarmu)=
Dir(\pi^i_\dotbarmu; \alp^i_\dotmu)
\eeq

In these TPMs,

$i\in
S_\rvi=\{0, 1, \dots, ni-1\}$= node index

$\rvx.=(\rvx_i)_{i\in S_\rvi}$= the 
nodes of the learned bnet.

$k\in
S_{\rvk^i}=\{0, 1, \dots, nk^i-1\}$= states 
of node $\rvx_i$

$\mu\in
S_{\ul{\mu}^i}=\{0, 1, \dots, n\mu^i-1\}$=
states of parents of node $\rvx_i$.

In the previous section,
we assumed a single node ($ni=1$)
with no parents ($n\mu=1$)
so that 
we could drop  the $i, \mu$
indices.
In this section, we eliminate
that restriction.


It is 
convenient
to define the magnitude of a  bnet
 $G$
to equal the sum
over nodes of the number 
of free parameters in each TPM:

\beq
|G|=\sum_i
(nk^i-1)n\mu^i
\;.
\eeq

Suppose that 
we are given $nsam$ samples 
$
\vec{\rvx}_i=
(\rvx_i[\sigma])_
{\sigma=0, 1, \ldots, nsam-1}
$ of 
our learned bnet. The count numbers
$N^i_\kmu$ are defined 
in terms of those samples 
as follows:
 

\beq
N^i_\kmu=\sum_\sigma
\indi(\rvx_i[\sigma]=k, pa(\rvx_i[\sigma])=\mu )
\;.
\eeq
It is also convenient
to
defined
count number ratios

\beq
N^i_\kbarmu=
\frac{N^i_\kmu}{N^i_{+, \mu }}
\;.
\eeq
Note that
$N^i_\kmu$
is a positive integer whereas
$N^i_\kbarmu\in[0,1]$.

Let's denote
the components of the TPMs by
$\pi^i_\kbarmu$:


\beq
\pi_\kbarmu^i=
P(\rvx_i=k\cond pa(\rvx_i)=\mu )
\approx
N^i_\kbarmu
\;.
\eeq


The rest
of this section
lists equations that
we obtained 
from the previous
section, by adding the new indices
$i, \mu$:

\beq
\calk(N^i_\dotmu,
 \alp^i_\dotmu)=
\frac{B(
N^i_\dotmu+
\alp^i_\dotmu)
}
{CI(N^i_\dotmu)B(\alp^i_\dotmu)}
\eeq


\beq
\hat{\pi}^i_\kbarmu=
\frac{N^i_\kmu+\alp^i_\kmu}
{N^i_\plusmu + \alp^i_\plusmu}
\eeq


\beq
P(N^i_\dotmu|
\alp^i_\dotmu
)
=
\calk(N^i_\dotmu, \alp^i_\dotmu)
\eeq

\beq
P(N^i_\dotmu| 
\pi^i_\dotbarmu, 
\alp^i_\dotmu)
\approx
Mul(N^i_\dotmu+
\alp^i_\dotmu; 
\pi^i_\dotbarmu, 
N^i_\plusmu + \alp^i_\plusmu)
\eeq

\begin{claim}
\beqa
\ln P(N^i_\dotmu|\hat{\pi}^i_\dotbarmu, 
\alp^i_\dotmu)
&=&
\sum_k
(N^i_\kmu +\alp^i_\kmu)
\ln\left(\frac{N^i_\kmu + \alp^i_\kmu}
{N^i_\plusmu  +\alp^i_\plusmu}\right)
+
\ln C(N^i_\dotmu+\alp^i_\dotmu)
\\
&>&
\sum_k
(N^i_\kmu +\alp^i_\kmu)
\ln\left(\frac{N^i_\kmu + \alp^i_\kmu}
{N^i_\plusmu  +\alp^i_\plusmu}\right)
-\frac{1}{2}(nk^i-1)\ln N^i_\plusmu
\eeqa
\end{claim}

\section*{Bayesian Scores}

\begin{itemize}
\item Bayesian Information Criterion (BIC)
\beqa\color{red}
\text{BIC-score}
&=&
-\sum_i\sum_{k, \mu}
N^i_\kmu
\ln\left(
\frac{N^i_\kmu}{N^i_\plusmu}
\right)+
\underbrace{
\left[-\frac{|G|}{2}\ln N^+_{+,+}\right]
}_{\sum_i\sum_\mu\ln C(N^i_\dotmu)
\text{ would be more accurate}}
\\
&\approx&
\sum_i
\sum_\mu
\ln P(N^i_\dotmu|\hat{\pi}^i_\dotbarmu, 
\alp^i_\dotmu=0)
\eeqa

\item Bayesian Dirichlet (BD)
\beqa
\color{red}
\text{BD-score}
&=&
\sum_{i}\sum_\mu\ln 
\frac
{B(N^i_\dotmu+\alp^i_\dotmu)}
{B(N^i_\dotmu)}
\\
&=&
\sum_i \sum_\mu
\ln
\left[
CI(N^i_\dotmu)
P(N^i_\dotmu|
\alp^i_\dotmu)
\right]
\eeqa

\item BD equivalent (BDe)
\beqa
\color{red}
\text{BDe-score}
&=&
\text{BD-score}
\left(
\alp^i_\kmu=
\alp'N^i_\kmu
\right)
\;,
\eeqa
where $\alp'$
is a free parameter.

\item BD equivalent unified (BDeu)
\beqa
\color{red}
\text{BDeu-score}
&=&
\text{BD-score}
\left(
\alp^i_\kmu=
\frac{\alp'}{nk^in\mu^i}
\right)
\;,
\eeqa
where $\alp'$
is a free parameter.
The BDeu 
score satisfies {\bf score equivalence};
i.e.,
it is the same
for all DAGs in 
an equivalence class
of observational
equivalent DAGs. See
Chapter \ref{ch-obs-equi}
for more information about observational 
equivalence.

\end{itemize}

\section*{Information Theoretic scores}


\begin{itemize}
\item Maximum likelihood
\beqa
\color{red}\text{ML-score}
&=&
\sum_i \sum_\kmu
N^i_\kmu\ln N^i_\kbarmu
\\
&=&
-\sum_i H(N^i_\dotbarmu)
\eeqa

\item Bayesian Information
Criterion(BIC), aka Minimum Description Length (MDL)
\beqa\color{red}
\text{BIC-score}
&=&\text{ML-score} -\frac{|G|}{2}\ln N^+_{+,+}
\\
&\approx&
\sum_i \sum_\kmu
N^i_\kmu\ln \frac{N^i_\kbarmu}
{\sqrt{N^+_{+,+}}}
\eeqa



\item Akaike Information Criterion (AIC)

\beqa\color{red}
\text{AIC-score}
&=&
\text{ML-score} -|G|
\\
&\approx&
\sum_i \sum_\kmu
N^i_\kmu\left[\ln N^i_\kbarmu-1
\right]
\eeqa
\end{itemize}