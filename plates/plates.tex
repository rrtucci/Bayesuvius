\chapter{Plate Notation}
\label{ch-plates}

In this chapter, we
will use the Numpy-like tensor notation
discussed in Section 
\ref{sec-numpy-tensors}. In particular, note that $[n] = [0:n] = \{0, 1,\ldots, n-1\}$ and that $T^{[n], [m]}$ is an $n\times m$ matrix.

Plate notation is often used in Machine Learning. See, for instance,
Chapter \ref{ch-transformer} on Transformers Networks, for  examples of plate notation.


{\bf Plate notation} is used to describe, in a compact way, a family of 
equal, disjoint  sub-bnets of a bnet that are connected in parallel or in series.
Suppose you have a bnet containing as a subset, $\Lam$ disjoint node sets
$S_\lam$ (\qt{sub-bnets}), where
$\lam\in [\Lam]$. Suppose
any two $S_\lam$ have the same number of equivalent nodes,
and two equivalent nodes have the same TPM.

In case the $S_\lam$ are {\bf connected in parallel (CIP)}:
Rather than drawing all $\Lam$ sets, 
we think of them as {\bf layers} of a {\bf stack}
that come out of the page like a stack of pancakes with the pancakes
lying flat on the page.
That way we only have to draw one pancake instead of $\Lam$.
We only draw once instead of $\Lam$ times, each node and the arrows entering and exiting
that node. Quite a saving in labor and bnet complexity! 
And a bnet can have more 
than one plate, and a node can belong
to more than one plate!

In case the $S_\lam$ are {\bf connected in series (CIS)}: Rather than drawing all $\Lam$ sets, 
we think of them as {\bf links} in a {\bf chain}.
That way we only have to draw one link instead of $\Lam$.
We only draw once instead of $\Lam$ times, each node and the arrows entering and exiting
that node. 


The simplest possible use
of CIS plates is for representing
a Markov chain. This is
illustrated in  Fig.\ref{fig-texnn-for-markov}.
\begin{figure}[h!]\centering
$$\xymatrix{
{\underline{y}}&&*+[F*:SkyBlue]{\underline{A}}\ar[ll]&&{\underline{x}}\ar[ll]
\save
\POS"1,2"."1,2"."1,4"."1,4"!C*+<1.0em>\frm{-,}
\restore
\\
*+[F-,]{\;\;}&\text{3 links}
} \quad\quad=\quad\quad \xymatrix{\rvy & \rvA\ar[l] &\rvA\ar[l] &\rvA\ar[l] & \rvx\ar[l]}$$
\caption{3-link Markov chain represented in plate notation
and without plates.}
\label{fig-texnn-for-markov}
\end{figure}


Fig.\ref{fig-texnn-for-rock_layers}
gives an example of a bnet with 2 nested CIP plates\footnote{Fig.\ref{fig-texnn-for-rock_layers}
and the blue TPMs were
rendered with my free, open source software 
{\tt texnn} (see Ref.\cite{texnn})
{\tt texnn} 
can keep track of 
the tensor shapes
of each node, for bnets with one or more plates.
}. 
The TPMs for this bnet are of the following
form (we print them in blue).

\begin{figure}[h!]\centering
$$\xymatrix{
&&*+[F*:SkyBlue]{\underline{Q}^{[20],[10]}}\ar[dr]&&&
\\
&&*+[F*:SkyBlue]{\underline{K}^{[20],[10]}}\ar[r]&*+[F*:SkyBlue]{\underline{A}^{[20]}}\ar[ddrr]&&
\\
&&*+[F*:SkyBlue]{\underline{V}^{[20],[10]}}\ar[ur]&&&
\\
*+[F*:pink]{\underline{X}}\ar[uurr]\ar[uuurr]\ar[urr]\ar[rrrrr]&&&&&*+[F*:pink]{\underline{Y}}
\save
\POS"1,3"."3,3"."1,3"."3,3"!C*+<1.0em>\frm{--}
\POS"1,3"."3,3"."1,4"."3,4"!C*+<2.0em>\frm{-,}
\restore
\\
*+[F--]{\;\;}&\text{$10$ layers}
\\
*+[F-,]{\;\;}&\text{$20$ layers}
}$$
\caption{Example of a bnet with 2 nested CIP plates.  In general, multiple plates need not be 
nested.}
\label{fig-texnn-for-rock_layers}
\end{figure}

\begin{subequations}

\begin{equation}\color{blue}
P(A^{[20]}|Q^{[20],[10]},K^{[20],[10]},V^{[20],[10]})=
\label{eq-A-fun-rock_layers}
\end{equation}

\begin{equation}\color{blue}
P(K^{[20],[10]}|X)=
\label{eq-K-fun-rock_layers}
\end{equation}

\begin{equation}\color{blue}
P(Q^{[20],[10]}|X)=
\label{eq-Q-fun-rock_layers}
\end{equation}

\begin{equation}\color{blue}
P(V^{[20],[10]}|X)=
\label{eq-V-fun-rock_layers}
\end{equation}

\begin{equation}\color{blue}
P(X)=
\label{eq-X-fun-rock_layers}
\end{equation}

\begin{equation}\color{blue}
P(Y|X,A^{[20]})=
\label{eq-Y-fun-rock_layers}
\end{equation}

\end{subequations}


