\chapter{Frisch-Waugh-Lovell (FWL) theorem}
\label{ch-fwl-theo}

The Frisch-Waugh-Lovell (FWL) theorem
(see Ref.\cite{wiki-fwl-theo})
(mnemonic: FoWL Theorem)
is a method used in Linear Regression (LR).
It allows us to
calculate
a regression coefficient
by doing LR with two residuals. Let's call them
the {\bf independent residual} and the
{\bf dependent or target residual}.\footnote{Standard 
LR is performed with two features,
the independent feature and the dependent or target feature.}
These  two residuals are calculated 
with the help of two previously
performed LR steps.

As in Section [\nameref{sec0-conv-lr}]
on LR, we will consider
two cases: $x^\s$ non-random, and $x^\s$ random i.i.d..

\section{FWL, assuming $x^\s$ are non-random}
Suppose
\beq
y=  X_1\beta_1 + X_2\beta_2 + \eps
\label{eq-y-x1-x2-eps}
\eeq
where

$y, \eps\in \RR^{nsam}$

$X_a\in \RR^{nsam\times k_a}$,
$\beta_a\in \RR^{k_a}$ for $a=1,2$

Define the matrices $U_1$ and $A_1$ by

\beq
U_1 = X_1(X_1^TX_1)^{-1}X_1^T
\eeq
and

\beq
A_1 = 1-U_1
\;.
\eeq
Note that

\beq
U_1X_1=X_1\;,\;\;A_1X_1=0
\eeq
(mnemonic: $A_1$ Annihilates $X_1$,
and $U_1$ acts like Unity on $X_1$).

Applying $A_1$ to Eq.(\ref{eq-y-x1-x2-eps}) gives

\beq
A_1 y = A_1X_2\beta_2  + A_1\eps
\eeq
so we can estimate $\beta_2$ by

\beq
\boxed{
\hat{\beta}_2=
(A_1X_2)^{-1}A_1 y}
\label{eq-fwl-nonrand}
\;.
\eeq
\section{FWL, assuming $x^\s$ are random}


Assume for simplicity that 
$k_1=k_2=1$
in Eq.(\ref{eq-y-x1-x2-eps}). Let
$\beta_1=\alp\in\RR$, 
$\beta_2=\beta\in\RR$.
When the $x^\s$ are random and i.i.d.,
and
$X_1, X_2$ are
replaced by 
the random variables
$\rvx, \rvd\in \RR$,
Eq.(\ref{eq-y-x1-x2-eps})
is equivalent to


\beq
\rvy=\alpha \rvx + \beta \rvd
+ \rvu_\rvy
\label{eq-y-ax-bd-u}
\;.
\eeq

As usual, we will denote
the linear operator
$\av{\rvx,\rvx}^{-1}\av{\rvx, \cdot}$
by a derivative operator:

\beq
\frac{\av{\rvx, \cdot}}
{\av{\rvx,\rvx}}=
\frac{d\cdot}{d\rvx}
\;.
\eeq
Define operators $U_\rvx$ and $A_\rvx$ by

\beq
U_\rvx(\cdot)= \rvx\frac{d\cdot}{d\rvx}
\eeq
and

\beq
A_\rvx = 1 - U_\rvx
\;.
\eeq
Note that

\beq
U_\rvx\rvx=\rvx\;,\;\;
A_\rvx \rvx=0
\;.
\eeq

If we apply $A_\rvx$ to Eq.(\ref{eq-y-ax-bd-u}), we get

\beq
A_\rvx\rvy = \beta A_\rvx \rvd
+A_\rvx\rvu_\rvy
\;.
\eeq
Hence,

\beq
\av{A_\rvx\rvd, A_\rvx\rvy}=
\beta\av{A_\rvx\rvd, A_\rvx\rvd}
\;.
\eeq
Thus,

\beq
\boxed{
\beta = \frac{d (A_\rvx \rvy)}{d(A_\rvx\rvd)}
}
\;.
\label{eq-fwl-rand}
\eeq
\hrule
Eqs.(\ref{eq-fwl-nonrand})
and (\ref{eq-fwl-rand})
constitute our statement
of the FWL theorem. They
are the main result of this chapter.
But before ending this chapter,
let us show how the
FWL theorem for random $x^\s$
can be interpreted
graphical using LDEN diagrams
(LDEN diagrams
are discussed in Chapter \ref{ch-linear-sys}).


\begin{figure}[h!]
$$
\xymatrix{
&\rvx\ar[dr]^\alpha&\rvu_\rvy\ar[d]
\\
\rvd\ar[rr]_\beta&&\rvy
\\
&(a)
}\;\;\;
\xymatrix{\rvu_\rvd\ar[d]
&\rvx\ar[dr]^{\alpha'}\ar[dl]_\gamma
&\rvu'_\rvy\ar[d]
\\
\rvd&&\rvy
\\
&(b)}
\;\;\;
\xymatrix{
\\
A_\rvx\rvd\ar[rr]_\beta&&A_\rvx\rvy
\\
&(c)
}$$
\caption{LDEN diagrams for discussing the FWL theorem.
}
\label{fig-fwl-abc}
\end{figure}

Fig.\ref{fig-fwl-abc}$(a)$
is a graphical
representation of the following equation:
\begin{subequations}
\beq
\rvy = \alpha \rvx + \beta\rvd + \rvu_\rvy
\;.
\label{eq-fwl-a}
\eeq
Likewise, \ref{fig-fwl-abc}$(b)$
represents

\beq
\left\{
\begin{array}{l}
\rvd = \gamma \rvx + \rvu_\rvd
\\
\rvy = \alpha'\rvx + \rvu'_\rvy
\end{array}
\right.
\label{eq-fwl-b}
\eeq
and \ref{fig-fwl-abc}$(c)$
represents

\beq
A_\rvx\rvy = \beta A_\rvx \rvd
\;.
\label{eq-fwl-c}
\eeq
\end{subequations}

Applying $d/d\rvx$ to Eq.(\ref{eq-fwl-a}
yields

\beq
\frac{d\rvy}{d\rvx}=
\alp
+\beta\frac{d\rvd}{d\rvx}
\;.
\eeq
Applying $d/d\rvx$ to Eq.(\ref{eq-fwl-b}
yields


\beq
\gamma=\frac{d\rvd}{d\rvx}
\;,\;\;
\alp'=\frac{d\rvy}{d\rvx}
\;.
\eeq
Therefore,

\beq
\alp'=\alp +\beta\gamma
\;.
\eeq

Applying $A_\rvx$ to Eq.(\ref{eq-fwl-b})
yields
\beq
A_\rvx\rvy=(1-\rvx\frac{d}{d\rvx})\rvy=
\rvy-\alp'\rvx
\eeq
and

\beq
A_\rvx\rvd=
(1-\rvx\frac{d}{d\rvx})\rvd
=
\rvd-\gamma\rvx
\;.
\eeq
Therefore, Eq.(\ref{eq-fwl-c})
is equivalent to

\beq
\rvy-\alpha'\rvx = \beta(\rvd-\gamma\rvx)
\;.
\eeq