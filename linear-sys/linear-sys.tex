\chapter{Linear Deterministic Bnets with Exogenous 
Noise}

In this chapter, we will consider 
bnets which were referred to,
prior to the invention of bnets, as:
Sewall Wright's Path Analysis (PA) and
linear Structural Equations Model (SEM).
Judea Pearl in his
books calls them
linear Structural Causal Models (SCM),
because they are very 
convenient for doing causal analysis.
We will follow 
Judea's convention
and refer to them as scum.


To 
build a SCM,
start with a deterministic bnet $G$.
Now  
add to each node $\rva$ of $G$ a 
root node $\rvU_\rva$
pointing into $\rva$ only.
The nodes $\rvU_\rva$ are called
the {\bf exogenous (external) variables}.
The exogenous
variables make their children noisy.
Their TPMs are
priors and are assumed 
to be unobserved.
Since they are 
root nodes, they are 
mutually independent.
When
drawing
bnets for SCM,
we will
never draw the exogenous nodes,
leaving them implicit.

Examples:
\begin{enumerate}
\hrule
\item


\begin{figure}[h!]
$$\xymatrix{
&\rvx\ar[dl]_\beta\ar[dr]^\alp
\\
\rvw\ar[dr]_\epsilon&&\rvz\ar[ll]_\gamma\ar[dl]^\delta
\\
&\rvy
}$$
\caption{
Example of a linear
SCM wherein
$\rvx$ splits
into two nodes $\rvz$
and $\rvw$,
then merges into node $\rvy$.
There is also an arrow
$\rvz\rarrow \rvw$.
Exogenous
nodes are not shown.}
\label{fig-scm-diamond}
\end{figure}

The TPMs, printed in blue,
for the nodes of the bnet 
Fig.\ref{fig-scm-diamond},
are as follows.

\beq\color{blue}
P(y|w, z, U_\rvy)=
\indi(y=\epsilon w +\delta z
+ U_\rvy)
\eeq

\beq\color{blue}
P(w|x, z, U_\rvw)=
\indi(w=\beta x +\gamma z + U_\rvw)
\eeq

\beq\color{blue}
P(z|x, U_\rvz)=
\indi(z=\alpha x + U_\rvz)
\eeq

\beq\color{blue}
P(x|U_\rvx)=
\indi(x=U_\rvx)
\eeq

Hence,
\beqa
y&=&
\epsilon w +\delta z
+ U_\rvy
\\
&=&
\epsilon (\beta x +\gamma z + U_\rvw)
 +\delta z
+ U_\rvy
\\
&=&
(\epsilon\gamma + \delta)z
+ \epsilon\beta x
+\epsilon U_\rvw U_\rvy
\\
&=&
(\epsilon\gamma + \delta)z
+ \epsilon\beta U_\rvx
+\epsilon U_\rvw U_\rvy
\eeqa
and

\beq
\left(\pder{y}{z}\right)_{U.-U_\rvz}=
\epsilon\gamma + \delta
\;,
\eeq
where the
partial
derivative holds fixed
all
exogenous
variables except
$U_\rvz$.
Note that
this partial
derivative is a 
sum of terms,
and that each of those terms
represents a different
causal path (one with
all arrows
pointing
in
the same direction,
a Markov chain)
from $\rvz$ to $\rvy(\rvz)$.
This
is a general
property
of linear SCM
diagrams.

\hrule
\item
Our next example
uses the notation
$\av{\rvx, \rvy}$
for 
covariances 
of any two random variables $\rvx, \rvy$.
The $\av{\rvx, \rvy}$ notation
is defined in the 
Notational
Conventions Chapter \ref{ch-not-cons}.

Consider a
bnet
with 
deterministic nodes
$\rvx.=(\rvx_k)_{k=0, 1, \ldots nx-1}$
and 
corresponding exogenous nodes 
$\rvU.=(\rvU_k)_{k=0, 1, \ldots nx-1}$.
Assume $\av{\rvU_i, \rvU_j}=0$
if $i\neq j$. The {\bf structural
coefficient} $\alp_{j|i}> 0$
measures the strength of
the connection 
$\rvx_i\rarrow \rvx_j$.


\begin{itemize}

\item {\bf Fully connected graph with $nx=2$}

\begin{figure}[h!]
$$
\xymatrix{
\rvx_0
\ar[d]^{\alp{1|0}}
\\
\rvx_1
}$$
\caption{
Fully connected graph with two $\rvx_j$
nodes
(exogenous nodes $\rvU_j$
not shown).}
\label{fig-fully-conn-2}
\end{figure}

Note that
\begin{subequations}
\label{eq-fully-conn-2}
\beqa
\rvx_0&=&\rvU_0
\\
\rvx_1&=&\alp_{1|0}\rvx_0  + \rvU_1
\eeqa
\;.
\end{subequations}
Eqs.\ref{eq-fully-conn-2}
constitute a system of 2 
linear equations in 2 unknowns
(the $\rvx$'s) so we can solve
for the $\rvx$'s in terms 
of the $\alpha$'s and $\rvU$'s.

Note also that
\beq
\av{\rvx_1, \rvx_0}=
\alp_{1|0}\av{\rvx_0, \rvx_0}
\;.
\eeq
Thus, $\alp_{1|0}$
can be estimated  
from the covariances $\av{\rvx_i, \rvx_j}$.

\item {\bf Fully connected graph with $nx=3$}

\begin{figure}[h!]
$$
\xymatrix{
\rvx_0\ar[d]_{\alp_{1|0}}
\ar[dr]^{\alp_{2|0}}
\\
\rvx_1\ar[r]^{\alp_{2|1}}
&\rvx_2
}$$
\caption{
Fully connected graph with 
three $\rvx_j$ nodes
(exogenous nodes $\rvU_j$
not shown).}
\label{fig-fully-conn-3}
\end{figure}

Note that
\begin{subequations}
\label{eq-fully-conn-3}
\beqa
\rvx_0 &=& \rvU_0
\\
\rvx_1&=&\alp_{1|0}\rvx_0 + \rvU_1
\\
\rvx_2&=&\alp_{2|1}\rvx_1 +
\alp_{2|0}\rvx_0 +\rvU_2
\;.
\eeqa
\end{subequations}
Eqs.\ref{eq-fully-conn-3}
constitute a system of
3 linear  equations in 3 unknowns
(the $\rvx$'s) so we can solve
for the $\rvx$'s in terms 
of the $\alpha$'s and $\rvU$'s.

Note also that
\begin{subequations}
\label{eq-fully-conn-3-covs}
\beqa
\av{\rvx_1, \rvx_0}&=&
\alp_{1|0}\av{\rvx_0, \rvx_0}
\\
\av{\rvx_2, \rvx_0}&=&
\alp_{2|1}\av{\rvx_1, \rvx_0}
+
\alp_{2|0}\av{\rvx_0, \rvx_0}
\\
\av{\rvx_2, \rvx_1}&=&
\alp_{2|1}\av{\rvx_1, \rvx_1}
+
\alp_{2|0}\av{\rvx_0, \rvx_1}
\eeqa
\end{subequations}
Eqs.\ref{eq-fully-conn-3-covs}
constitute a system of
3 linear  equations in 3 unknowns
(the $\alpha$'s) so we can solve
solve for the $\alpha$'s in terms
of covariances $\av{\rvx_i, \rvx_j}$.
This gives an estimate
for the $\alp$'s. 

\end{itemize}
\end{enumerate}