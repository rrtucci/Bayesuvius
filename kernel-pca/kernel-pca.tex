\chapter{Kernel Principal Component Analysis (Kernel PCA)}
\label{ch-kerne-pca}

This chapter is based on Ref.\cite{wiki-kernel-pca}.

We will use vector sign
as in $\vec{x}$ to denote 
vectors in $\RR^{na}$. We will use kets as in  $\ket{\Phi}=\Phi\in \RR^{N\times 1}$ and bras as in $\bra{\Phi}=\Phi^T\in \RR^{1\times N}$ to denote a column vector and row vector in $\RR^N$.

$\vec{x}_\s\in \RR^{na}$ for  $\s=1,2,\ldots N$

$\Phi: \RR^{na}\rarrow \RR^N$. $\Phi$ 
can be a matrix in $\RR^{N\times na}$ so that $
\Phi (\vec{x}_\s)=\Phi \vec{x}_\s\in\RR^N$,
but the most advantageous use of this
kernel arises when $\Phi(\cdot)$ is a nonlinear map.
 
{\bf ``Kernel trick" kernel}
\beqa
K(\vec{x}_\s, \vec{x}_{\rho}) &=& \av{\Phi(\vec{x}_\s) |
\Phi(\vec{x}_{\rho})}
\eeqa

\beq
K = [K(\vec{x}_\s, \vec{x}_{\rho})]
\eeq

\beqa
C=\frac{1}{N}\sum_\s 
\ket{\Phi(\vec{x}_\s)}
\bra{\Phi(\vec{x}_\s)}
\eeqa

{\bf eigenvalue problem for C}
\beq
C \ket{v_j} = \lam_j \ket{v_j}
\eeq



\beq
\bra{\Phi(\vec{x_\s})} C\ket{ v_j}=
\lam_j\av{\Phi(\vec{x_\s}) | v_j}
\eeq


\beq
\ket{v_j} = \sum_\rho A_{j,\rho}\ket{\Phi(\vec{x_\rho})}
\eeq

\beqa
\boxed{
\av{\Phi(\vec{x_\s}) |v_j}}
&=&
\sum_\rho A_{j,\rho}K_{\s, \rho}
\\
&=&
\boxed{(K\ket{A_j})_\s=
\av{\s|K|A_j}}
\eeqa

\beq
\sum_\rho A_{j,\rho}\underbrace{\bra{\Phi(\vec{x_\s})} C
\ket{\Phi(\vec{x_\rho})}}_{\frac{1}{N}K^2_{\s,\rho}}=
\lam_j\sum_\s A_{j,\s}
\underbrace{\av{\Phi(\vec{x_\s}) |\Phi(\vec{x_\rho})}}_{K_{\s, \rho}=K_{\rho, \s}}
\eeq

\beq 
\ket{A_j}_\s = A_{j, \s}
\eeq

\beq
K^2\ket{A_j} =  N\lam_j K\ket{A_j}
\eeq

{\bf eigenvalue problem for K}
\beq
 K\ket{A_j} =
N\lam_j \ket{A_j} 
\eeq

\beqa
\delta_{i,j}&=&\av{v_i|v_j}
\\
&=&
\sum_\rho\sum_\s 
A_{i,\s}A_{j, \rho}K_{\s, \rho}
\\
&=&
\bra{A_i}K\ket{A_j}
\\
&=&
N\lam_j\av{A_i|A_j}
\eeqa



{\bf Examples}

\begin{itemize}
\item Radial basis function
\beq
K(\vec{x}, \vec{y})= \exp\left(-\; 
\frac{\norm{\vec{x}-\vec{y}}^2}{2\s^2}
\right)
\eeq

\item Sigmoid kernel 
\beq
K(\vec{x}, \vec{y}) =
\tanh(a\vec{x}\cdot \vec{y} + b)
\eeq
where $a,b\in\RR$
\item Polynomial of order $d$

\beq
K(\vec{x}, \vec{y}) = (\vec{x}\cdot\vec{y})^d
\eeq

Consider $d=2$, $na=2$, $N=3$.
\begin{claim}
If

\beq
\Phi(\vec{x})= (x_1^2, x_2^2, \sqrt{2} x_1, x_2)
\label{eq-rings-to-lines}
\eeq
then


\beq 
(\vec{x}\cdot\vec{y})^2 =\av{\Phi(\vec{x})|\Phi(\vec{y})}
\eeq
\end{claim}
\proof

\beqa
( \vec{x}\cdot\vec{y})^2 &=&
(x_1y_1+x_2y_2)^2
\\
&=&
(x_1^2, x_2^2, \sqrt{2} x_1x_2)\cdot
(y_1^2, y_2^2, \sqrt{2} y_1y_2)
\\
&=&
\av{\Phi(\vec{x})|\Phi(\vec{y})}
\eeqa
\qed

The map defined by Eq.(\ref{eq-rings-to-lines})
is illustrated in Fig.\ref{fig-rings-to-lines}.
\begin{figure}[h!]
\centering
\includegraphics[width=2.9in]
{kernel-pca/kernel-pca-rings}
\includegraphics[width=2.9in]
{kernel-pca/kernel-pca-lines}
\caption{The map $\Phi(x_1,x_2)= (x_1^2, x_2^2, \sqrt{2} x_1, x_2)$ maps a ring in $\RR^2$ to a line in $\RR^3$.(plots from \cite{boram-lee-notes})}
\label{fig-rings-to-lines}
\end{figure}





\end{itemize}