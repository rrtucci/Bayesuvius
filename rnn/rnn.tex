\chapter{Recurrent Neural
 Networks}\label{ch-rnn}

This chapter is mostly
based on Ref.\cite{ng-rnn}.

This chapter
assumes you are
familiar 
with the material
and notation of Chapter \ref{ch-nn}
on plain Neural Nets.


\begin{figure}[h!]
\centering
$$\xymatrix{
\rvx(\cdot)\ar[d]\ar[dr]\ar[drr]\\
\rvx(0)\ar[d]&
\rvx(1)\ar[d]&
\rvx(2)\ar[d]\\
\rvh(0)\ar[d]\ar[r]&
\rvh(1)\ar[d]\ar[r]&
\rvh(2)\ar[d]\\
\rvY(0)&
\rvY(1)&
\rvY(2)
}$$
\caption{Simple example of 
RNN witb $T=3$}
\label{fig-rnn}
\end{figure}

Suppose

$T$ is a positive integer.

$t=0, 1, \ldots, T-1$,

$\rvx_i(t)\in \RR$ for
 $i=0,1, \ldots,numx-1$,

$\rvh_i(t)\in \RR$ for
 $i=0,1, \ldots,numh-1$,

$\rvY_i(t)\in \RR$ for
 $i=0,1, \ldots,numy-1$,

$W^{h|x}\in\RR^{numh\times numx}$,

$W^{h|h}\in\RR^{numh\times numh}$,

$W^{y|h}\in\RR^{numy\times numh}$,

$b^y\in \RR^{numy}$,

$b^h\in \RR^{numh}$.

Henceforth, $x(\cdot)$ will
mean the array of $x(t)$ for all $t$.

The simplest kind of
recurrent neural network (RNN)
has
the bnet Fig.\ref{fig-rnn}
with arbitrary $T$.
The node
TPMs, printed in
blue, for this bnet, are as follows.

\beq\color{blue}
P(x(\cdot))\text{ = given}
\eeq

\beq\color{blue}
P(x(t))=\delta(x(t), [x(\cdot)]_t)
\eeq

\beq\color{blue}
P(h(t)\cond h(t-1), x(t))=
\delta(h(t),
\cala(W^{h|x}x(t) +
 W^{h|h}h(t-1) + b^h))
\;,
\eeq
where
$h(-1)=0$.

\beq\color{blue}
P(Y(t)\cond h(t))=
\delta(Y(t),
\cala(W^{y|h}h(t) + b^y))
\eeq

Define

\beq
W^h=[W^{h|x}, W^{h|h}, b^h]
\;,
\eeq
and

\beq
W^y=[W^{y|h}, b^y]
\;.
\eeq

The bnet of Fig.\ref{fig-rnn}
can be used for
classification once 
its parameters 
$W^h$ and $W^y$
have been optimized.
To optimize
those parameters via gradient
descent,
one can use the bnet 
of Fig.\ref{fig-rnn-ext}.

Let $\sigma=0,1, \ldots, nsam(\vecx)-1$
be the labels for a minibatch of samples.
The node TPMs,
 printed in blue,
for bnet Fig.\ref{fig-rnn-ext},
 are as follows.



\begin{figure}[h!]
\centering
$$\xymatrix{
&\ranvec{x}(\cdot)\ar[d]\ar[dr]\ar[drr]
\ar@/^1pc/[dddd]
\\
&\vec{\rvx}(0)\ar[d]&
\vec{\rvx}(1)\ar[d]&
\vec{\rvx}(2)\ar[d]\\
\rvW^h\ar[r]\ar@/^2pc/[rr]\ar@/^2pc/[rrr]
\ar@/_3pc/[dddddd]
&\vec{\rvh}(0)\ar[d]\ar[r]&
\vec{\rvh}(1)\ar[d]\ar[r]&
\vec{\rvh}(2)\ar[d]\\
\rvW^y\ar[r]\ar@/^2pc/[rr]\ar@/^2pc/[rrr]
\ar@/_3pc/[dddddd]
&\vec{\rvY}(0)\ar@/^1pc/[dd]&
\vec{\rvY}(1)\ar@/^1pc/[dd]&
\vec{\rvY}(2)\ar@/^1pc/[dd]
\\
&
\ranvec{y}(\cdot)\ar[d]\ar[dr]\ar[drr]
\\
\ul{\cale}\ar[ddd]
\ar@/_2pc/[dddd]&
\ul{\cale}(0)\ar[l]&
\ul{\cale}(1)\ar@/_1pc/[ll]&
\ul{\cale}(2)\ar@/_2pc/[lll]
\\
\\
\\
(\rvW^h)'\\
(\rvW^y)'
}
$$
\caption{RNN bnet used
to optimize parameters $W^h$
and $W^y$ of RNN bnet Fig.\ref{fig-rnn}.}
\label{fig-rnn-ext}
\end{figure}

\beq\color{blue}
P(x(\cdot)[\sigma])\text{ = given}
\eeq

\beq\color{blue}
P(x(t)[\sigma])=\delta(x(t)[\sigma],
[x(\cdot)]_t[\sigma])
\eeq

\beq\color{blue}
P(h(t)[\sigma]\cond h(t-1)[\sigma], x(t)[\sigma])=
\delta(h(t)[\sigma],
\cala(W^{h|x}x(t)[\sigma] + W^{h|h}h(t-1)[\sigma] + b^h)
\eeq

\beq\color{blue}
P(Y(t)[\sigma]\cond h(t-1)[\sigma])=
\delta(Y(t)[\sigma],
\cala(W^{y|h}h(t-1)[\sigma] + b^y)
\eeq

\beq\color{blue}
P(y(\cdot)[\sigma]\cond x(\cdot)[\sigma])\text{ = given}
\eeq

\beq\color{blue}
P(\cale(t)\cond \vecy(\cdot), \vec{Y}(t))
=\frac{1}{nsam(\vecx)}
\sum_\sigma d(y(t)[\sigma], Y(t)[\sigma])
\;,
\eeq
where 

\beq
d(y,Y)=|y-Y|^2
\;.
\label{eq-d-err-sq}
\eeq
If $y, Y\in [0,1]$, 
one can use this instead

\beq
d(y,Y)=XE(y\rarrow Y)=
-y\ln Y - (1-y)\ln (1-Y)
\;.
\eeq

\beq\color{blue}
P(\cale\cond [\cale(t)]_{\forall t})=
\delta(\cale, \sum_t \cale(t))
\eeq

For $a=h,y$,
\beq\color{blue}
P(W^a)\text{ = given}
\;.
\eeq
The first time it is used,
$W^a$ is fairly arbitrary. Afterwards,
it is determined by previous 
horizontal
stage.

\beq\color{blue}
P((W^a)'|\cale, W^a)=
\delta((W^a)', W^a -
\eta ^a\partial_{W^a}\cale)
\;.
\eeq
$\eta ^a>0$ is the learning rate
for $W^a$.

\section*{Language Sequence Modeling}

Figs.\ref{fig-rnn}, and \ref{fig-rnn-ext}
with arbirary $T$ can be used 
as follows to do
Language Sequence Modeling.

For this usecase, one must
train with the following
TPM for node $\vecy(\cdot)$:

\beq\color{blue}
P(y(\cdot)[\sigma]\cond x(\cdot)[\sigma])=
\prod_t\indi(\;\;\;y(t)[\sigma]=
P(x(t)[\sigma]\cond [x(t')[\sigma]]_{t'<t})
\;\;\;)
\eeq

With such training, one gets

\beq
P(Y(t)|h(t))=
\indi(\;\;\;
Y(t)=P(x(t)\cond [x(t')]_{t'<t})\;\;\;)
\;.
\eeq
Therefore,

\beq
Y(0)=P(x(0))
\;,
\eeq

\beq
Y(1)=P(x(1)|x(0))
\;,
\eeq

\beq
Y(2)=P(x(2)|x(0), x(1))
\;,
\eeq
and so on.

We can use this to: 
\begin{itemize}
\item
predict the probability 
of a sentence,

example: Get $P(x(0), x(1), x(2))$.
\item
predict 
the most likely 
next word in a sentence,

example: Get $P(x(2)| x(0), x(1))$.
\item generate fake sentences.

example: 

Get $x(0)\sim P(x(0))$.

Next get $x(1)\sim P(x(1)|x(0))$.

Next get $x(2)\sim P(x(2)|x(0), x(1))$.


\end{itemize}

 
\section*{Other types of RNN}

\begin{figure}[h!]
\centering
$$\xymatrix{
\rvx(\cdot)\ar[d]\ar[dr]\\
\rvx(0)\ar[d]&
\rvx(1)\ar[d]&
\\
\rvh(0)\ar[r]&
\rvh(1)\ar[r]&
\rvh(2)\ar[d]\ar[r]&
\rvh(3)\ar[d]\\
&
&
\rvY(2)&
\rvY(3)
}$$
\caption{RNN bnet of the
many to many kind. This
one can be used for  translation.
$x(0)$ and $x(1)$ might
denote two words of an English
sentence, and $Y(2)$ 
and $Y(3)$ might be
their Italian translation.}
\label{fig-rnn-translation}
\end{figure}

Let $\calt=\{0,1, \dots , T-1\}$,
and
$\calt^x, \calt^y\subset \calt$.
Above, 
we assumed that 
$\rvx(t)$ and $\rvY(t)$
were both defined 
for all $t\in \calt$.
More generally, they 
might be defined only
for subsets of $\calt$:
$\rvx(t)$ for $t\in \calt^x$
and 
$\rvY(t)$ for $t\in \calt^y$.
If $|\calt^x|=1$ and
$|\calt^y|>1$, 
we say the RNN bnet is of
the 1 to many kind.
In general, can have 
{\bf 1 to 1, 1 to many, many to 1, 
many to many} RNN bnets.

Plain RNNs can suffer 
from the
{\bf vanishing or exploding
 gradients problem}.
There are various ways to
mitigate this (good choice of initial
$W^h$ and $W^y$, 
good choice of activation 
functions, regularization).
Or by using GRU or LSTM (discussed below).
 {\bf GRU and LSTM}
were designed to mitigate the
vanishing or exploding gradients problem.
They are very popular in NLP (Natural
Language Processing).



\newpage

\section*{Long  
Short Term Memory (LSTM) unit (1997)}

This section
is based on Wikipedia article 
Ref.\cite{lstm}. In this section,
$\odot$
will denote the Hadamard matrix product
(elementwise product).

\begin{figure}[h!]
\centering
$$\xymatrix{
&&\rvx(t)\ar[ddd]\ar[dl]\ar[ldd]\ar[lddd]
\\
&\rvi(t)\ar[rddd]
&\\
&\rvf(t)\ar[rdd]&\\
&\rvo(t)\ar[ddr]
&\tilde{\rvc}(t)\ar[d]
\\
\rvc(t-1)\ar[rr]
&&\rvc(t)\ar[d]
\\
\rvh(t-1)\ar[ruuuu]\ar[ruuu]\ar[ruu]\ar[rruu]
&&\rvh(t)\ar[d]\\
&&\rvY(t)
}$$
\caption{
bnet for a Long Short Term Memory
 (LSTM) unit.}
\label{fig-rnn-lstm}
\end{figure}

Let

$\rvx(t)\in \RR^{numx}$: 
input vector to the LSTM unit

$\rvf(t)\in \RR^{numh}$:
forget gate's activation vector

$\rvi(t)\in \RR^{numh}$: 
input/update gate's activation vector

$\rvo(t)\in \RR^{numh}$: 
output gate's activation vector

$\rvh(t)\in \RR^{numh}$: 
hidden state vector also known as
 output vector of the LSTM unit

${\tilde {\rvc}}(t)\in \RR^{numh}$: 
cell input activation vector

$\rvc(t)\in \RR^{numh}$: 
cell state vector

$\rvY(t)\in \RR^{numy}$: 
classification of $x(t)$.

$W\in \RR^{numh\times numx}$, 
$U\in \RR^{numh\times numh}$
and 
$b\in \RR^{numh}$: 
weight matrices and bias vectors,
 parameters learned by training.

$\calw^{y|h}\in \RR^{numy\times numh}$:
 weight matrix


Fig.\ref{fig-rnn-lstm}
is a bnet net
for a LSTM unit.
The node TPMs, printed in blue,
for this bnet, are
as follows.

\beqa\color{blue}
P(f(t)|x(t), h(t-1))=\indi(\;\;\;&
f(t)=\sig(W^{f|x}x(t)+
U^{f|h}h(t-1)+b^{f})
\;\;\;)
\;,
\eeqa
where $h(-1)=0$.

\beqa\color{blue}
P(i(t)|x(t), h(t-1))=\indi(\;\;\;&
i(t)=\sig(W^{i|x}x(t)
+U^{i|h}h(t-1)+b^{i})
\;\;\;)
\eeqa

\beqa\color{blue}
P(o(t)|x(t), h(t-1))=\indi(\;\;\;&
o(t)=\sig(W^{o|x}x(t)
+U^{o|h}h(t-1)+b^{o})
\;\;\;)
\eeqa

\beqa\color{blue}
P(\tilde{c}(t)|x(t), h(t-1))=\indi(\;\;\;&
\tilde{c}(t)=\tanh
(W^{c|x}x(t)+U^{c|h}h(t-1)+b^{c})
\;\;\;)
\eeqa

\beqa\color{blue}
P(c(t)|f(t), c(t-1), i(t),
 \tilde{c}(t))
=\indi(\;\;\;&
c(t)=f(t)\odot c(t-1)+
i(t)\odot {\tilde {c}}(t)
\;\;\;)
\eeqa

\beqa\color{blue}
P(h(t)|o(t), c(t))=\indi(\;\;\;&&
h(t)=o(t)\odot \tanh
(c(t))
\;\;\;)
\eeqa



\beqa\color{blue}
P(Y(t)|h(t))=\indi(\;\;\;&
Y(t)= \cala(\calw^{y|h}h(t) + b^y)
\;\;\;)
\eeqa

\newpage
\section*{Gated Recurrence Unit
 (GRU) (2014)}

This 
section is based 
on Wikipedia article Ref.\cite{gru}. In this section,
$\odot$
will denote the Hadamard matrix product
(elementwise product).

GRU is a more recent (17 years later)
attempt at simplifying LSTM unit.

\begin{figure}[h!]
\centering
$$\xymatrix{
&\rvr(t)\ar[dr]&\rvx(t)\ar[d]\ar[dl]\ar[l]
\\
&\rvz(t)\ar[dr]
&\hat{\rvh}(t)\ar[d]
\\
\rvh(t-1)\ar[ur]\ar[rr]\ar[urr]\ar[uur]
&&\rvh(t)\ar[d]
\\
&&\rvY(t)
}$$
\caption{bnet for a Gated
Recurrent Unit (GRU).}
\label{fig-rnn-gru}
\end{figure}

Let

$\rvx(t)\in \RR^{numx}$: input vector

$\rvh(t)\in\RR^{numh}$: output vector

$\hat{\rvh}(t)\in\RR^{numh}$: candidate activation vector

$\rvz(t)\in\RR^{numh}$: update gate vector

$\rvr(t)\in\RR^{numh}$: reset gate vector

$\rvY(t)\in \RR^{numy}$: 
classification of $x(t)$.

$W\in \RR^{numh\times numx}$, 
$U\in \RR^{numh\times numh}$
and 
$b\in \RR^{numh}$: 
weight matrices and bias vectors,
 parameters learned by training.

$\calw^{y|h}\in \RR^{numy\times numh}$:
 weight matrix

Fig.\ref{fig-rnn-gru}
is a bnet net
for a GRU.
The node TPMs, printed in blue,
for this bnet, are
as follows.


\beqa\color{blue}
P(z(t)|x(t),h(t-1))=\indi(\;\;\;&
z(t) = \sig(W^{z|x} x(t) + U^{z|h} h(t-1) + b^z)
\;\;\;)
\;,
\eeqa
where $h(-1)=0$.

\beqa\color{blue}
P(r(t)|x(t), h(t-1))=\indi(\;\;\;&
r(t) = \sig(W^{r|x} x(t) + U^{r|h} h(t-1) + b^r)
\;\;\;)
\eeqa

\beqa\color{blue}
P(\hat{h}(t)|x(t), r(t), h(t-1))=
\indi(\;\;\;&
\hat{h}(t) = \tanh(W^{h|x} x(t) +
 U^{h|h} (r(t) \odot h(t-1)) + b^h)
\;\;\;)
\eeqa

\beqa\color{blue}
P(h(t)|z(t), h(t-1),\hat{h}(t))=\indi(\;\;\;&
h(t) =  (1 - z(t)) \odot h(t-1) +
 z(t) \odot \hat{h}(t)
\;\;\;)
\eeqa

\beqa\color{blue}
P(Y(t)|h(t))=\indi(\;\;\;&
Y(t)= \cala(\calw^{y|h}h(t) + b^y)\;\;\;)
\eeqa