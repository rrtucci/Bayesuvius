\chapter{Do Calculus proofs}
\label{ch-do-calc-proofs}

In Chapter \ref{ch-do-calc},
we explained Do Calculus,
but referred to this
chapter for proofs
of claims that
use Do Calculus.
In this chapter, we've
aggregated
 all proofs, from
throughout the book,
of claims that use Do Calculus.

Note that even though the 3
rules of Do Calculus
are great for proving
adjustment formulae
for general classes of DAGs,
they are sometimes overkill
for proving
 adjustment formulae
for a single specific DAG.
Indeed, since the
 3 rules of Do Calculus
are a consequence
of the d-separation theorem, it follows that
all adjustment
formulae should be
provable from first principles,
assuming only
the d-separation theorem
and the standard rules of
probability theory.

In this chapter, we use the
 following conventions for bnet diagrams.

\bnetInstantiations

\hiddenNodes

Selection diagrams
with bnet-switches
 are discussed
in Chapter \ref{ch-transport}.
\selectionGraphs

Some {\bf node summation identities}
that are used in this chapter:

\hrule
\begin{enumerate}
\item

\beq
\sum_a
P(a|x_1, x_2)=1
\eeq

\beq
\xymatrix@R=.3pc{
x_1\ar[dr]
\\
&\sum a&=1
\\
x_2\ar[ur]
}
\label{eq-collider-sum}
\eeq
\hrule
\item
\beq
\sum_a P(y|a, x_1, x_2)P(a|x_1, x_2)P(x_2|x_1)P(x_1)
= P(y,x_1, x_2)
\;.
\eeq

\beq
\xymatrix{\\P(x_1)}
\xymatrix{
x_1\ar[rd]\ar[drr]\ar[dd]
\\
&\sum a\ar[r]
&y
\\
x_2\ar[ru]\ar[rru]
}
\xymatrix
{\\ =P(x_1)}
\xymatrix{
x_1\ar[rrd]\ar[dd]
\\
&&y
\\
x_2\ar[urr]
}
\label{eq-univ-bdoor}
\;.
\eeq
This identity can be interpreted
geometrically by noting that the 
bnet on the left hand side (LHS) is fully connected 
so one can reverse the direction of the 
arrow from $\sum a \rarrow y$ without changing the numerical value of the LHS. The the sum over $a$
yields 1 so the $\sum a$ vertex 
with its 3 incoming arrows can be deleted.

The following identity has a similar geometrical interpretation

\beq
\sum_a P(y|x_2, a, x_1)P(x_2|a, x_1)P(x_1|a)P(a)
= P(y,x_1, x_2)
\;.
\eeq

\beq
\xymatrix{
x_1\ar[drr]\ar[dd]
\\
&\EE a\ar[r]\ar[ul]\ar[dl]
&y
\\
x_2\ar[rru]
}
\xymatrix
{\\ = P(x_1)}
\xymatrix{
x_1\ar[rrd]\ar[dd]
\\
&&y
\\
x_2\ar[urr]
}
\label{eq-univ-bdoor-ee}
\;.
\eeq


\hrule
\item
\beq
\sum_a P(y|a, x)P(a|x)P(x)=P(x,y)
\eeq

\beq
\xymatrix@R=1pc{
\\P(x)}
\xymatrix@R=1pc{
&\sum a\ar[rd]
\\
x\ar[ru]\ar[rr]
&&y
&=&P(x)\quad
x\ar[r]
&y}
\label{eq-med-sum}
\eeq
Once again, the bnet on the LHS is fully
connected so the arrow $\sum a\rarrow y$ can be reversed and then the sum over $a$ yields 1.

The following identity has a similar geometrical interpretation

\beq
\sum_a P(y|x,a)P(x|a)P(a)=P(x,y)
\eeq

\beq
\xymatrix@R=1pc{
&\EE a\ar[rd]\ar[ld]
\\
x\ar[rr]
&&y
&=&P(x)\quad
x\ar[r]
&y}
\label{eq-med-sum-ee}
\eeq

\hrule

\item
\beq
\sum_a P(y|a)P(a)=P(y)
\eeq

\beq
\xymatrix{
\EE a\ar[r]
&y
&=&P(y)
}
\label{eq-diff-priors}
\eeq
Furthermore,

\beq
\sum_a P(a|y)P(y)=P(y)
\eeq

\beq
\xymatrix{
P(y)\quad
\sum a
&y\ar[l]
&=&P(y)
}
\label{eq-diff-priors-ee}
\eeq

\end{enumerate}
\hrule

The moral of the above summation identities is that
if one has a fully connected bnet, or a clique
(i.e., fully connected subgraph) within a bnet,
one can remove a sum vertex $\sum a$ or an average
vertex $\EE a$ of the clique. Not only that! We can also 
replace a sum vertex $\sum a$  of the clique by a different  sum vertex $\sum b$. Likewise, we can replace
an average vertex $\EE a$  of the clique by a different  average vertex $\EE b$.

An important caveat concerning the above
summation identities is that 
the instantiations that they produce depend on the
original bnet whence they come from. 
This caveat is illustrated in Fig.\ref{fig-dep-on-origing-bnet}.
This caveat is especial relevant when dealing 
with do operators. In that case, the unamputated
graph and the amputated graph are different bnets, with
different full probability distributions.

\begin{figure}[h!]
$$
\begin{array}{c|c}
\sum_a P_{G}(x_2|a)P_{G}(a|x_1)=P_{G}(x_2|x_1)
&
\sum_a P_{G'}(x_2|a, x_1)P_{G'}(a|x_1)=P_{G'}(x_2|x_1)
\\
\xymatrix@C=1.5pc{
x_1\ar[r]
&\sum a\ar[r]
&x_2
&=&
x_1\ar[r]
&x_2}
&
\xymatrix@C=1.5pc{
x_1\ar[r]\ar@/_1pc/[rr]
&\sum a\ar[r]
&x_2
&=&
x_1\ar[r]
&x_2}
\end{array}
$$
\caption{Note that there are cases when $P_G(x_2|x_1)
\neq P_{G'}(x_2|x_1)$. Hence,
 $P(x_2|x_1)$ depends on the bnet of origin.
}
\label{fig-dep-on-origing-bnet}
\end{figure}






We define a {\bf do-adjustment formula}
for {\bf do-query}
$P(y|\cald\rvx=x)$
to be an algebraic expression  
constructed from the TPMs for {\it observed} (i.e., not hidden) nodes of the original bnet $G$.
If a do-adjustment formula
exists for a
particular do-query,
then we say the do-query is
 {\bf do-identifiable (DI)}.
% \footnote{
%To prove that a do-query $P(y|\cald\rvx=x, z)$
%is do-identifiable
%for a graph $G$,
%just prove that $\rvy\perp\rvx|\rvz$
%in $\call_\rvx G$.
%This is called Rule 2 of Do Calculus,
%but it is easy to understand just from
%the d-separation theorem.
%Info can be transmitted
%between $\rvy$ and $\rvx$ by
%either (1) paths
%in $\cald_\rvx G$
%or (2) paths in $\call_\rvx G$.
%$P(y|\cald\rvx=x, z)=
%P(y|x, z)$
%means the info is being transmitted only
%by (1).
%So the Rule 2 premise is checking
%that no info is being transmitted
%by (2). }
A {\bf do-transport formula}
is a relationship between 2  do-queries.
This chapter deals with both
do-adjustment and do-transport
formulae.

Let $G$ be a graph before
amputation
of the arrows entering node $\rvx$,
and let $G_{amp(x)}=\cald_{\rvx=x}G$
be the same graph after amputation
at $\rvx$.
Also let $P_G()$ be
the full probability distribution
for graph $G$, and $P_{G_{amp(x)}}()$
be that for $G_{amp(x)}$.
In general, the
following is always true,
for bnets
with or bnets without hidden nodes:\footnote{Note that $P_{G_{amp(x)}}(y|x)\neq P_G(y|x)=P(y|x)$.
In fact,
$P_{G_{amp(x)}}(y|x)= P(y|x)$
iff there is no confounding,
so $P_{G_{amp(x)}}(y|x)\neq P(y|x)$
indicates confounding.
}

\beq
P_{G_{amp(x)}}(y|\cald\rvx=x)=P_{G_{amp(x)}}(y|x)\neq P_G(y|x)
\;
\eeq
However, $P_{G_{amp(x)}}(y|x)$ is
not a valid adjustment formula
for this query
because it's not expressed in terms
of observed TPMs of $G$.

To make matters even more confusing, let $G|x$
be the graph $G$ conditioned on $\rvx=x$. If $G$ has full distribution $P_G()$,
then  $P_{G|x)}(\cdot) = P_G(\cdot|x)$. 
Note that $G$, $G_{amp(x)}$ and $G|x$ have different
full probability distributions so they should not
be considered the same $P()$ function. 

IMPORTANT: For any adjustment formula, we assume
a hypothesis bnet $G_{hypo}$.
Then for a query
$P(y|\cald\rvx=x)$, we express the adjustment
formula in terms of the full probability distribution $P()=P_{G_{hypo}|x}()=P_{G_{hypo}}(\cdot|x)$.


%Note that
%it's always true that $P(y|\cald \rvx = x)=P(y|x)$.
%However, if the bnet contains more nodes than just $\rvx$ and $\rvy$, then the bnet is identifiable
%iff it can be stratified (i.e., 
%expressed as $P(y|x)=\sum_{z.}P(y|x, z.)P(z.)$
%for some observed multinode $\rvz.$).
%Thus, an alternative  name for identifiability is
%{\bf stratifiability}. Note also that there mightt be more than
%one adjustment formula, but they must all 
%be numerically equal to $P(y|x)$ for all $x$ and $y$.


The following 3 step heuristic technique appears to be equivalent to Pearl's 3 rules of Do Calculus.\footnote{Three rules to match his three :)  This heuristic technique was invented solely by me. I haven't proven that it is equivalent to Pearl's 
3 rules of Do Calculus. I believe it is,
but it might not be. It 
correctly produces the backdoor and frontdoor adjustment formulae.}

\begin{mdframed}[hidealllines=true,backgroundcolor=blue!10]

\begin{enumerate}
\item 
Write down a bnet instantiation
that has a DAG structure identical
to the DAG structure of $G$,
except that arrows entering node
$\rvx$ have been amputated.
All nodes  of that instantiation, except
nodes  $x$ and $y$,
are summed (indicated by $\sum$) or averaged
(indicated by $\EE$) over.



\item Replace each sum $\sum h$ over a hidden variable $h$ by a sum over an observed variable. This is accomplished using the node summation 
identities.

\item 
Replace each average $\EE h$ over a hidden variable $h$ by an average over an observed variable. This is accomplished using the node summation
identities.
\end{enumerate}
It's not always possible to perform Steps 2 and 3 
and obtain a valid adjustment formula.
In such a case,
we declare the  query not-identifiable.

\end{mdframed}


\begin{figure}[h!]
$$
\begin{array}{ccc}
\xymatrix{
&\rvz\ar[ld]\ar[rd]
\\
\rvx\ar[rr]&&\rvy
}
&\quad\quad&
\xymatrix{
&*++[F-o]{\rvz}\ar[ld]\ar[rd]
\\
\rvx\ar[rr]&&\rvy
}
\\
(a) \text{$P(y|\cald\rvx=x)$ is DI}
&&(b) \text{$P(y|\cald\rvx=x)$ is not-DI}
\end{array}
$$
\caption{Examples of
bnets for which
the do-query $P(y|\cald\rvx=x)$
is
DI
and not-DI.
}
\label{fig-iden-noniden}
\end{figure}
See Fig.\ref{fig-iden-noniden}
for some simple
examples of
bnets for which
the do-query $P(y|\cald\rvx=x)$
is
DI
and not-DI.
Using our heuristic technique,
we can easily see why the
query $P(y|\cald\rvx=x)$
is DI
for bnet $(a)$ 
and not-DI for bnet $(b)$
of Fig.\ref{fig-iden-noniden}.

For bnet $(a)$, after amputating arrow
$z\rarrow x$ and averaging over node $z$,
we get

\beqa
P(y|\cald\rvx=x)
&=&
\xymatrix{
\EE z\ar[dr]
\\
x\ar[r]
&y
}
\label{eq-simplest-identifiable}
\eeqa
The right hand
side of Eq.(\ref{eq-simplest-identifiable})
is a valid
adjustment formula
because it's expressed in terms
of observed TPMs of $G$.

For bnet $(b)$,
if we amputate arrow
$z\rarrow x$
and average over node $z$,
we get

\beqa
P(y|\cald \rvx =x)
&=&
\xymatrix{
*++[F-o]{\EE z}\ar[rd]
\\
x\ar[r]
&y
}
\label{eq-simplest-not-DI}
\eeqa
The right hand
side of Eq.(\ref{eq-simplest-not-DI})
is not a valid
adjustment formula
because
if we replace $\EE z$ by $\EE x' $, we get 
$\sum_{x'}P(y|x, x')P(x')$,
and $P(y|x, x')$ is undefined.






\begin{claim}
\label{cl-decBackDoor}
\decBackDoor
\end{claim}

\proof

{\bf * proof 1:}
%\beq
%P(y|\cald\rvx=x, z)
%=
%\xymatrix{
%z\ar[dr]
%\\
%x\ar[r]
%&y
%}
%\eeq

\beq
P(y|\cald\rvx=x)
=
\xymatrix{
\EE z\ar[dr]
\\
x\ar[r]
&y
}
\eeq


{\bf * proof 2:}
\begin{longtable}{l}
\color{red}
$P(y|\cald\rvx=x)
=
\sum_z
P(y|\cald\rvx=x, z)
P(z|\cald\rvx=x)$
\\
\quad by Probability Axioms
\\
\color{red}
$=\sum_z
P(y|x, z)
P(z|\cald\rvx=x)$
\\
\quad $P(y|\cald \rvx=x, z)\rarrow
P(y|x, z)$
\\
\quad  by Rule 2:
\ruletwoshort
\\
\quad
$\rvy\perp\rvx|\rvz$ in
$\call_\rvx\cald_\emptyset G:
\xymatrix{
\rvz\ar[d]\ar[rd]
\\
\rvx
&\rvy
}$
\\
\color{red}
$=\sum_z
P(y|x, z)
P(z)$
\\
\quad $P(z|\cald \rvx=x)\rarrow
P(z)$
\\
\quad  by Rule 3:
\rulethreeshort
\\
\quad
$\rvz\perp \rvx$ in
$\cald_\rvx \cald_\emptyset G:
\xymatrix{
\rvz\ar[rd]
\\
\rvx\ar[r]
&\rvy
}
$
\end{longtable}
\qed




\begin{claim}
\label{cl-decFrontDoor}
\decFrontDoor
\end{claim}

\proof


{\bf * proof 1:}

\begin{longtable}{l}
$P(y|\cald\rvx=x) =\sum_c\sum_m P(y|m,c)P(m|x)P(c)$
\\
\hspace{2cm}
\xymatrix{
&\EE c\ar[dr]
\\
x\ar[r]&\sum m\ar[r]
&y
}
\\
\quad$=\sum_m \left[\sum_c\frac{1}{P(m|c)} P(y|m,c)P(m|c)P(c)\right]P(m|x)$
\\
\hspace{2cm}
\xymatrix{\\
\sum_m\sum_c\frac{P(c)}{P(m|c)}
}
\xymatrix{
&&c\ar[dr]\ar[d]
\\
x\ar[r]&m&m\ar[r]
&y
}
\\
\quad$=\sum_m \frac{1}{P(m)}\left[\sum_c P(y|m,c)P(m|c)P(c)\right]P(m|x)$ \\
\hspace{2cm}(because $P(m|c)=P(m)$ by d-separation
on $G_{hypo}|x$)
\\
\hspace{2cm}
\xymatrix{\\
\sum_m\frac{1}{P(m)}
}
\xymatrix{
&&\EE c\ar[dr]\ar[d]
\\
x\ar[r]&m&m\ar[r]
&y
}
\\
\quad$=\sum_m P(y|m)P(m|x)$ 
\\
\hspace{2cm}
(because square bracketed expression equals $P(y,m)$)
\\
\hspace{2cm}
\xymatrix{
x\ar[r]&\sum m \ar[r]& y
}\quad (this simplifies frontdoor AF)
\\
\quad
$=\sum_{x'}\sum_m P(y|m,x')P(m|x)P(x')$
\\
\hspace{2cm}
(because can reverse steps that led to last expression)
\\
\hspace{2cm}
\xymatrix{
&\EE x'\ar[dr]
\\
x\ar[r]&\sum m\ar[r]
&y
}
\end{longtable}

{\bf * proof 2:}
\begin{longtable}{l}
$\color{red}
P(y|\cald\rvx=x)=
\sum_m
P(y|\cald\rvx=x, m)
P(m|\cald\rvx=x)$
\\
\quad by Probability Axioms
\\
$\color{red}=
\sum_m
P(y|\cald\rvx=x, \cald\rvm=m)
P(m|\cald\rvx=x)$
\\
\quad $P(y|\cald\rvx=x, m)\rarrow
P(y|\cald\rvx=x, \cald m=m)$
\\
\quad by Rule 2:
\ruletwoshort
\\
\quad $\rvy\perp \rvm|\rvx$ in
$\call_\rvm\cald_\rvx G:$
$\xymatrix{
&*++[F-o]{\rvc}\ar[rd]
\\
\rvx\ar[r]
&\rvm
&\rvy
}$
\\
$\color{red}=
\sum_m
P(y|\cald\rvx=x, \cald\rvm=m)
P(m| x)$
\\
\quad $P(m|\cald\rvx=x)\rarrow P(m|x)$
\\
\quad by Rule 2:
\ruletwoshort
\\
\quad
$\rvm\perp \rvx$ in
$\call_\rvx\cald_\emptyset G:$
$\xymatrix{
&*++[F-o]{\rvc}\ar[ld]\ar[rd]
\\
\rvx
&\rvm\ar[r]
&\rvy
}$
\\
$\color{red}=
\sum_m
P(y|\cald\rvm=m)
P(m|x)$
\\
\quad $P(y|\cald\rvx=x, \cald\rvm=m)
\rarrow
 P(y|\cald\rvm=m)$
\\
\quad by Rule 3:
\rulethreeshort
\\
\quad
$\rvy\perp\rvx|\rvm$ in
$\cald_\rvx\cald_\rvm G:$
$\xymatrix{
&*++[F-o]{\rvc}\ar[rd]
\\
\rvx
&\rvm\ar[r]
&\rvy
}$
\\
$\color{red}=
\sum_{x'}
\sum_m
P(y|\cald\rvm=m, x')
P(x'|\cald\rvm=m)
P(m|x)$
\\
\quad by Probability Axioms
\\
$\color{red}=
\sum_{x'}
\sum_m
P(y|m, x')
P(x'|\cald\rvm=m)
P(m|x)$
\\
\quad $P(y|\cald\rvm=m, x')
\rarrow
\quad P(y|m, x')$
\\
\quad by Rule 2:
\ruletwoshort
\\
\quad
$\rvy\perp\rvm|\rvx$ in
$\call_\rvm \cald_\emptyset G:$
$\xymatrix{
&*++[F-o]{\rvc}\ar[rd]\ar[ld]
\\
\rvx\ar[r]
&\rvm
&\rvy
}$
\\
$\color{red}=
\sum_{x'}
\sum_m
P(y|m, x')
P(x')
P(m|x)$
\\
\quad $P(x'|\cald\rvm=m)
\rarrow
P(x')$
\\
\quad by Rule 3:
\rulethreeshort
\\
\quad
$\rvx\perp\rvm$ in
$\cald_\rvm \cald_\emptyset G:$
$\xymatrix{
&*++[F-o]{\rvc}\ar[rd]\ar[ld]
\\
\rvx
&\rvm\ar[r]
&\rvy
}$
\end{longtable}
\qed



\begin{claim}
\label{cl-decNapkin}
\decNapkin
\end{claim}
\proof
\\
\begin{align}
P(y|\cald\rvx=x)
&=
\xymatrix{
&*++[F-o]{\EE u_1}\ar[ddl]\ar[ddrr]
&&
\\
&*++[F-o]{\EE u_2}\ar[dl]
&&
\\
\sum w\ar[r]
&\sum z
&x\ar[r]
&y
}
\\
&=
\xymatrix{
\EE u_1\ar[dr]
\\
x \ar[r]&  y}
\\
&\quad This last diagram is not identifiable. 
\end{align}
\qed


\begin{claim}
\label{cl-decWhy}
\decWhy
\end{claim}
\proof
\beqa
P(y, w|\cald\rvz=z, x)
&=&
\xymatrix{
x\ar[d]\ar[drr]
\\
 w
&z\ar[r]
&y
\\
*++[F-o]{\EE u}\ar[u]\ar[urr]
}
\eeqa
\beqa
P(y|\cald\rvz=z,x)
&=&
\xymatrix{
x\ar[d]\ar[drr]
\\
\sum w
&z\ar[r]
&y
\\
*++[F-o]{\sum u}\ar[u]\ar[urr]
&\EE w'\ar[l]
}
\\
&=&
\xymatrix{
x\ar[d]\ar[drr]
\\
\sum w
&z\ar[r]
&y
\\
{\EE w'}\ar[u]\ar[urr]
}
\\
&=&
\xymatrix{
x\ar[drr]
\\
\EE w\ar@/_1pc/[rr]
&z\ar[r]
&y
}
\eeqa
\qed

\begin{claim}
\label{cl-decTransportTrivial}
\decTransportTrivial
\end{claim}
\proof
\begin{longtable}{l}
\color{red}
$P(y|\cald\rvx=x, z, \rvs=1)=
P(y|x, z, \rvs=1)$
\\
\xymatrix{
&z\ar[rd]
&\rvs=1\ar[d]
\\
\cald\rvx=x\ar[rr]
&&y
}
\xymatrix{
\\=
}
\xymatrix{
&z\ar[rd]
&\rvs=1\ar[d]
\\
x\ar[rr]
&&y
}
\end{longtable}
\qed

\begin{claim}
\label{cl-decTransportDirect}
\decTransportDirect
\end{claim}
\proof
\begin{longtable}{l}
\color{red}
$P(y|\cald\rvx=x, z, \rvs=1)=
P(y|\cald\rvx=x, z)$
\\
\xymatrix{
\rvs=1\ar[r]
&z\ar[dr]
&
\\
\cald\rvx=x\ar[rr]
&&y
}
\xymatrix{\\=}
\xymatrix{
&z\ar[dr]
&
\\
\cald\rvx=x\ar[rr]
&&y
}
\begin{tabular}{l}
Because $\rvs\perp\rvy|\rvz$
\end{tabular}
\end{longtable}
Furthermore,
\begin{longtable}{l}
\color{red}
$P(y|\cald \rvx=x, \rvs=1)
=\sum_z P(y|\cald\rvx=x,z)P(z|\rvs=1)$
\\
\xymatrix{
\rvs=1\ar[r]
&\sum z\ar[rd]
\\
\cald\rvx=x\ar[rr]
&&y
}
\end{longtable}
\qed

\begin{claim}
\label{cl-decTransportBox}
\decTransportBox
\end{claim}
\proof
\begin{longtable}{l}
\color{red}$
P(y|\cald\rvx=x, \rvs=1)
=\sum_{a}
P(y|\cald\rvx=x,a)P(a|\rvs=1)$
\\
\xymatrix{
\rvs=1\ar[r]
&*++[F-o]{\sum z}\ar[r]
&\sum a\ar[d]
\\
&\cald \rvx=x\ar[r]
&y
}\xymatrix{\\=}
\xymatrix{
\rvs=1\ar[r]
&\sum a\ar[d]
\\
\cald \rvx=x\ar[r]
&y
}
\end{longtable}
\qed

%\decTransportOne merged
%with decDirectTransport

\begin{claim}
\label{cl-decTransportNon}
\decTransportNon
\end{claim}
\proof
\begin{longtable}{l}
\color{red}
$P^*(y|\cald\rvx=x)=P^*(y|\cald\rvx=x)$
\\
\\
\xymatrix{
&*++[F-o]{\EE h}\ar[dr]
&\rvs=1\ar[d]
\\
\cald\rvx=x\ar[rr]
&&y
}
\xymatrix{\\=}
\xymatrix{
&&\rvs=1\ar[d]
\\
\cald\rvx=x\ar[rr]
&&y
}
\end{longtable}
Can't replace $\cald\rvx=x$
by $x$ because
$\rvy\not\perp \rvx$ in
$\call_\rvx G$.
Hence, Rule 2 not satisfied.
\qed


\begin{claim}
\label{cl-decTransportTwo}
\decTransportTwo
\end{claim}
\proof
\begin{longtable}{l}
\color{red}
$P(y|\cald \rvx=x, \rvs=1)=
\sum_h P(y|\cald\rvx=x, h)P(h)$
\\
\\
\xymatrix{
\rvs=1\ar@/^1.5pc/[rr]
&*++[F-o]{\EE h}\ar[r]\ar[rd]
&\sum z
\\
\cald\rvx=x\ar[rr]
&&y
}
\xymatrix{\\=}
\xymatrix{
&*++[F-o]{\EE h}\ar[rd]
&
\\
\cald\rvx=x\ar[rr]
&&y
}
\\
\color{red}
$=P(y|\cald\rvx=x)$
\\
\xymatrix{=}
\xymatrix{
\cald\rvx=x\ar[rr]
&&y
}
\end{longtable}
\qed
\begin{claim}
\label{cl-decTransportThree}
\decTransportThree
\end{claim}
\proof

\begin{longtable}{l}
\color{red}
$P(y|\cald\rvx=x, \rvs=1)=
\sum_h
\sum_z P(y|h, z)P(h)
P(z|\cald\rvx=x, \rvs=1)$
\\
\\
\xymatrix{
\rvs=1\ar[rd]
&*++[F-o]{\EE \rvh}\ar[dr]
\\
\cald\rvx=x\ar[r]
&\sum z\ar[r]
&y
}
\\
\\
\color{red}
$=\sum_h\sum_z P(y|h,z)P(h|\cald\rvx=x)
P(z|x, \rvs=1)$
\\
\\
\xymatrix{\\=}
\xymatrix{
\rvs=1\ar[rd]
&*++[F-o]{\sum \rvh}\ar[dr]
&\cald\rvx=x\ar[l]
\\
x\ar[r]
&\sum z\ar[r]
&y
}
\\
\color{red}
$=\sum_z P(y|\cald\rvx=x, z)P(z|x, \rvs=1)$
\\
$\xymatrix{\\=}\xymatrix{
\rvs=1\ar[rd]
&&\cald\rvx=x\ar[d]
\\
x\ar[r]
&\sum z\ar[r]
&y
}$
\end{longtable}
\qed

\begin{claim}
\label{cl-decMediationSimple}
\decMediationSimple
\end{claim}
\proof
\begin{longtable}{l}
\color{red}
$P(y|\cald\rvd=d,\cali\rvd=d')=
\sum_{m}
P(y|d,m)P(m|d')$
\\
\\
\xymatrix{
\cali\rvd=d'\ar[r]
&\sum m\ar[rd]
\\
\cald\rvd=d\ar[rr]
&&y}
\end{longtable}
\qed

\begin{claim}
\label{cl-decMediationPlus}
\decMediationPlus
\end{claim}
\proof
\begin{longtable}{l}
\color{red}
$P(y|\cald\rvd=d,\cali\rvd=d')=
\sum_{\xi, u}
\sum_{m}
P(y|d,m,\xi, u)P(m|d', \xi, u)
\underbrace{P(\xi|u)P(u)}_{P(\xi, u)}$
\\
\\
\xymatrix{
&&*++[F-o]{\EE u}\ar[ddr]
\ar[dl]\ar[d]
\\
\cali\rvd=d'\ar@/_1pc/[rr]
&\sum\xi\ar[rrd]\ar[r]
&\sum m\ar[rd]
\\
\cald\rvd=d\ar[rrr]
&&&y}
\\
\color{red}
$=
\sum_{\xi}
\sum_{m}
P(y|d,m,\xi)P(m|d', \xi)
P(\xi)$
\\
\xymatrix{\\=}
\xymatrix{
\cali\rvd=d'\ar@/_1pc/[rr]
&\EE\xi\ar[rrd]\ar[r]
&\sum m\ar[rd]
\\
\cald\rvd=d\ar[rrr]
&&&y}
\begin{tabular}{l}
We switch from averaging over\\ the
prior of $\xi, u$\\
to averaging over the\\
prior of $\xi$.
\end{tabular}
\end{longtable}
\qed

\begin{claim}
\label{cl-decSeqBackDoor}
\decSeqBackDoor
\end{claim}
\proof
\begin{longtable}{l}
\color{red}
$P(y|\cald\rvx^3=x^3)=
\calq(y|x^3)$
\\
\xymatrix@C=.5pc{
\EE z_0\ar[r]\ar@/^1pc/[rr]
\ar[drrr]
&\sum z_1\ar[r]
\ar[drr]
&\sum z_2
\ar[dr]
\\
&&&y
\\
\cald\rvx_0=x_0\ar[uur]\ar[uurr]
\ar[urrr]
&\cald\rvx_1=x_1\ar[uur]
\ar[urr]
&\cald\rvx_2=x_2
\ar[ur]
}
\xymatrix{\\=}
\xymatrix@C=.9pc{
\EE z_0\ar[r]\ar@/^1pc/[rr]
\ar[drrr]
&\sum z_1\ar[r]
\ar[drr]
&\sum z_2
\ar[dr]
\\
&&&y
\\
x_0\ar[uur]\ar[uurr]
\ar[urrr]
&x_1\ar[uur]
\ar[urr]
&x_2
\ar[ur]
}
\begin{tabular}{l}
We can replace\\
$\cald\rvx_i=x_i$
by $x_i$
\\once all nodes
\\in bnet are
\\observed nodes.
\end{tabular}
\end{longtable}
\qed

\begin{claim}
\label{cl-decSBBackDoor}
\decSBBackDoor
\end{claim}
\proof
\begin{longtable}{l}
\color{red}
$P(y|\cald\rvx =x, \rvs=1)=
\sum_z
P(y|\cald\rvx=x, z)P(z^{<\rvx}|\rvs=1)
P(z^{>\rvx}|x,z^{<\rvx}, \rvs=1)$
\\
\\
\xymatrix{
\rvs=1\ar[r]\ar@/^1pc/[rr]
&\sum z^{<\rvx}\ar[d]\ar[r]
&\sum z^{>\rvx}
\\
\cald\rvx=x\ar[rru]\ar[r]
&y
}
\\
\color{red}
$= \sum_{z^{<\rvx}} P(y|\cald\rvx=x, z^{<\rvx})
P(z^{<\rvx}|\rvs=1)$
\\
\xymatrix{\\=}
\xymatrix{
\rvs=1\ar[r]
&\sum z^{<\rvx}\ar[d]
&
\\
\cald\rvx=x\ar[r]
&y
}
\\
\color{red}
$= \sum_z P(y|x, z)
P(z|\rvs=1)$
\\
\xymatrix{\\=}
\xymatrix{
\rvs=1\ar[r]
&\sum z\ar[d]
&
\\
x\ar[r]
&y
}
\begin{tabular}{l}
$\cald$ can be removed because there are\\
no sums over unobserved nodes.
\end{tabular}
\\
\color{red}
$= \sum_z P(y|x, z)
P(z)$
\\
\xymatrix{\\=}
\xymatrix{
&\EE z\ar[d]
&
\\
x\ar[r]
&y
}
\begin{tabular}{l}
$\rvs=1$ node can be removed\\ because
this expression must\\ equal
$P(y|x, \rvs=1)$. Furthermore,\\
$\rvy\perp\rvs|(\rvx,\rvz)$
in the hypothesis bnet.\\
Hence,  this expression must also
\\ equal $P(y|x)$.
\end{tabular}
\end{longtable}
\qed


%

\begin{claim} (from Ref.\cite{hunermund2021})

If
$
\xymatrix{
&\rvw\ar[d]\ar[ddr]\ar@/^1.5pc/@{<-->}[ddr]|\rvu
\\
&\rvh\ar[dr]
\\
\rvx\ar[rr]\ar[uur]\ar@/_1.5pc/@{<-->}[dr]
&&\rvy
\\
&\rve\ar[ru]\ar[lu]
}$
then
\beq
P(y|\cald \rvx = x)=
\sum_e P(y|x,e)P(e)
\eeq

\beq
\xymatrix{
\cald\rvx=x\ar[r]
&y
}\xymatrix{\\=}
\xymatrix{
x\ar[rr]
&&y
\\
&\sum e\ar[ru]
}
\eeq
\end{claim}
\proof
\\
\begin{longtable}{l}
\color{red}
$P(y|\cald\rvx=x)=$
\\
$
\xymatrix{
&\sum w\ar[d]\ar[ddr]\ar@/^1.5pc/@{<-->}[ddr]|{\sum u}
\\
&\sum h\ar[dr]
\\
\cald\rvx=x\ar[rr]\ar[uur]
&&y
\\
&\sum e\ar[ru]
}
\xymatrix{
\\
\\
=}
\xymatrix{
\\
\cald\rvx=x\ar[rr]
&&y
\\
&\sum e\ar[ru]
}$
\\
\color{red}
$=\sum_e
P(y|\cald\rvx=x, e)
P(e|\cald\rvx=x)$
\\
\xymatrix{
\\=}
$
\xymatrix{
\cald\rvx=x\ar[rr]
\ar[dr]|0
&&y
\\
&\sum e\ar[ru]
}$
\\
\color{red}
$=\sum_e
P(y|x, e)
P(e|\cald\rvx=x)$
\\
\xymatrix{
\\=}
$
\xymatrix{
x\ar[rrd]
\\
\cald\rvx=x
\ar[dr]|0
&&y
\\
&\sum e\ar[ru]
}$
\begin{tabular}{l}\\
by Rule 2
\end{tabular}
\\
\color{red}
$=\sum_e
P(y|x, e)
P(e)$
\\
$
\xymatrix{\\=}
\xymatrix{
x\ar[rr]
&&y
\\
&\sum e\ar[ru]
}$
\begin{tabular}{l}\\
by Rule 3.
\\ No information
transmission\\
between $\cald \rvx$
and $\rve$.
\end{tabular}
\end{longtable}
\qed


\begin{claim}(from Ref.\cite{hunermund2021})

If $
\xymatrix{
&\rvw\ar[dl]\ar[d]
&\rvs\ar[l]
\\
\rvz\ar[r]
&\rvx\ar[r]
&\rvy
}$ then
\beq
P(y|\cald \rvx = x)
=
 \sum_z P(y|x,z,w,\rvs=1)
P(z|w, \rvs=1)
\eeq

\beq
\xymatrix{
\cald\rvx=x\ar[r]
&y
}\xymatrix{\\=}
\xymatrix{
&w\ar[dl]\ar[dr]
&\rvs=1\ar[d]\ar[dll]
\\
\sum z \ar@/_1.5pc/[rr]
&\rvx=x\ar[r]
&y
}
\eeq
\end{claim}
\proof
\begin{longtable}{l}
\color{red}
$P(y|\cald\rvx=x)=P(y|\cald\rvx=x, w, \rvs=1)$
\\
\xymatrix{&\sum w\ar[dl]
&\sum s \ar[l]
\\
\sum z
&\cald\rvx=x\ar[r]
&y
}
\xymatrix{
\\
=
}
\xymatrix{
&w\ar[dl]\ar[dr]
&\rvs=1\ar[l]
\\
\sum z
&\cald\rvx=x\ar[r]
&y
}
\begin{tabular}{l}
by Rule 1
\end{tabular}
\\
\\
\color{red}
$=\sum_z P(y|\cald\rvx=x,z,w,\rvs=1)
P(z|\cald\rvx=x, w, \rvs=1)$
\\
\xymatrix{\\=}
\xymatrix{
&w\ar[dl]\ar[dr]
&\rvs=1\ar[l]\ar[d]\ar[dll]
\\
\sum z\ar@/_1.5pc/[rr]
&\cald\rvx=x\ar[r]\ar[l]|0
&y
}
\\
\\
\color{red}
$=\sum_z P(y|\rvx=x,z,w,\rvs=1)
P(z|\cald\rvx=x, w, \rvs=1)$
\\
\xymatrix{\\=}
\xymatrix{
&w\ar[dl]\ar[dr]
&\rvs=1\ar[l]\ar[d]\ar[dll]
\\
\sum z\ar@/_1.5pc/[rr]
&\cald\rvx=x\ar[l]|0
&y
\\
&x\ar[ur]
}
\begin{tabular}{l}
\\
by Rule 2
\end{tabular}
\\
\\
\color{red}
$=\sum_z P(y|\rvx=x,z,w,\rvs=1)
P(z| w, \rvs=1)$
\\
\xymatrix{\\=}
\xymatrix{
&w\ar[dl]\ar[dr]
&\rvs=1\ar[d]\ar[dll]
\\
\sum z \ar@/_1.5pc/[rr]
&x\ar[r]
&y
}
\begin{tabular}{l}\\
by Rule 3.
\\
No info transmission\\
between $\cald \rvx$ and $\rvz$.
\end{tabular}
\end{longtable}
\qed

