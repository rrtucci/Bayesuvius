\chapter{Junction Tree Algorithm}
\label{ch-junc-tree}
The Junction Tree (JT)
algorithm
is an algo
for evaluating
exact marginals
of a bnet, 
including cases in which
some nodes are fixed to a single state.
(fixed nodes
are called the a priori evidence.)

The JT algo
starts by  
clustering
the loops of a bnet into bigger nodes
so as  to
transform the bnet into a polytree bnet.
Then it applies
Pearl Belief Propagation (see
Chapter \ref{ch-mpass}) to the ensuing polytree.
The first breakthrough 
paper to achieve this agenda in full
was Ref.\cite{lauritzen1988}
by Lauritzen, and Spiegelhalter in 1988.
See the Wikipedia article 
Ref.\cite{wiki-junc-tree}
for more info and
 references on the JT algorithm. 

I won't describe
the JT algo
any further here,
because it would take too
long for this brief book
to give a complete treatment
of it, including the mathematical proofs.
If all you want to do is to
code the JT algo, without
delving into the mathematical theorems
and proofs behind it, I 
strongly recommend 
Ref.\cite{huang1996}.
Ref.\cite{huang1996} is 
an excellent cookbook
for programmers of the JT algo. My
open source  
program QuantumFog (see Ref.\cite{qfog})
implements the
JT algo in Python, 
following the recipe of
Ref.\cite{huang1996}.