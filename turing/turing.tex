\chapter{Turing Machine COMING SOON}
\label{ch-turing}

\newcommand{\TA}[0]{\Sigma^+}

This chapter is based on Ref.\cite{wiki-turing-machine}.

A {\bf Turing Machine} (TM)  can be explained as a 
special case of a Petri net (pnet), and that
is the approach that we take in this chapter.
Petri nets are discussed in Chapter \ref{ch-petri}. We will assume
that the reader has read that chapter before tackling this one.

In fact, a TM can be viewed as a generalization
of a special case of a Petri net called a Finite
State Machine (FSM).  FSM are discussed
in Chapter \ref{ch-finite-state}.
We will assume
that the reader has read that chapter too before tackling this one.


\begin{figure}[h!]
\centering
\includegraphics[width=6in]
{turing/turing-physical.jpg}
\caption{View of Mount Vesuvius from
  Pompeii}
\label{fig-turing-phsical}
\end{figure}

\section{Example}

\begin{figure}[h!]
$$
\begin{array}{cc}
\xymatrix@C=6pc@R=3pc{
\PinkCircle{\rva}
\ar[rr]|{\rvx_1=(0,P,R)}
\ar[rd]|{\rvx_3=(1,P, L)}
&&\Circle{\rvb}
\ar@/_2pc/[ll]|{\rvx_2=(0,P,L)}
\\
\ar[u]
&\Circle{\rvc}
\ar[ru]|{\rvx_4=(0,P, L)}
\ar[r]|{\rvx_5=(1,P, R)}
&\DCircle{\rvh}
}
&
\begin{array}{c|c|c|}
\rvp\rarrow \rvx
&\ket{\rvp'}=\rvx\ket{\rvp}
& \lam=(\lam_r,
\lam_w,
\lam_m)
\\
\hline\hline
\rva\rarrow\rvx_1
&\rvb
&(0,P,R)
\\ \hline
\rvb\rarrow\rvx_2
&\rva
&(0,P,L)
\\ \hline
\rva\rarrow\rvx_3
&\rvc
& (1,P,L)
\\ \hline
\rvc\rarrow\rvx_4
&\rvb
& (0,P,L)
\\\hline
\rvc\rarrow\rvx_5
&\rvh
&(1,P,R)
\\ \hline
\end{array}
\end{array}
$$
\caption{The \qt{3-state busy beaver} Turing machine represented as a FSM Petri net.}
\label{fig-3-state-bb}
\end{figure}

Fig.\ref{fig-3-state-bb} shows an example of
a TM represented as a FSM Petri net. Everything in this diagram
should be familiar after reading the chapters
on pnets and FSM, except for the labelling of the 
transition nodes.



\section{Precise Definition}

\hrule
Same as in FSM:

$\calx =\{\rvx_i\}_{i=1}^{nx}$ are the {\bf transition nodes} of the pnet.

$\calp =\{\rvp_i\}_{i=1}^{np}$ are the {\bf place (or state) nodes} of the pnet.

$\rvp(0)\in \calp$, {\bf starting place }

$\calp_{acc}\subset \calp$ are the 
{\bf acceptor places}

$\Delta:\calx\times\calp\rarrow \calp$ is the 
{\bf transition function}
\hrule
New for TM

$HALT\in \calp_{acc}$

$\TA =$ {\bf tape alphabet}

$B\in \TA$ {\bf blank space}.

$\Sigma=$ {\bf input alphabet}. $\Sigma\subset\TA-\{B\}$. 

$\lam_{r}:\calx\rarrow \TA$ {\bf read symbol assigning function}

$\lam_{w}:\calx\rarrow \TA$ {\bf write symbol assigning function}

$\lam_{m}:\calx\rarrow \{L,R\}$ {\bf move command
assigning function}

$\lam = (\lam_r, \lam_w, \lam_m)$ is the
{\bf transition labelling function}

$\lam_\rvp: \calx(\rvp\rarrow)\rarrow \TA$ is the
{\bf restriction of $\lam()$ to $\calx(\rvp\rarrow)$}

\hrule
$\Phi_{TM}=\av{\Phi_{pnet},\calp_{acc}, \Sigma,
\Sigma^+, \lam, \Delta}$



