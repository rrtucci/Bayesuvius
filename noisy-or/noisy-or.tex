\chapter{Noisy-OR gate}
The Noisy-OR gate was first
proposed by Judea Pearl in his 1988 book
 Ref.\cite{pearl-1988book}.
\begin{figure}[h!]
$$\xymatrix{
\rvx_0\ar@/_1pc/[ddrr]
&&\rvx_1\ar@/_1pc/[dd]
&&\rvx_2\ar@/_1pc/[ddll]
\\
&\ul{\pi}_0\ar[dr]
&\ul{\pi}_1\ar[d]
&\ul{\pi}_2\ar[dl]
\\
&\ul{\lam}\ar[r]
&\rvy
}$$
\caption{Noisy-OR gate $\rvy\in\bool$
with $n=3$, Boolean inputs $(\rvx_i)_{i=0,1,2}$
and parameters 
$\ul{\lam}, (\ul{\pi})_{i=0,1,2}$.
}
\label{fig-noisy-or-1}
\end{figure}

Let

$\ul{\lam}\in [0, 1]=${\bf gate leakage}.

$y\in \bool$ = gate  output

$\rvx^n=(\rvx_i)_{i=0,1, \ldots, n-1}$, where
$\rvx_i\in\bool$ are
 gate inputs.

$\ul{\pi}^n=(\ul{\pi}_i)_{i=0,1, \ldots, n-1}$, where
$\ul{\pi}_i\in[0,1]$ are
 gate parameters.


The TPM, printed in blue,
 for the Noisy-OR gate $\rvy$ 
shown in Fig.\ref{fig-noisy-or-1}, is

\beq\color{blue}
P(y=1|x^n, \lam, \pi^n)=
1-(1-\lam)\prod_i[1-\pi_i x_i]
\label{eq-noisy-or-tmp-1}
\eeq

Note
that if $\lam=0$ and $\pi_i=1$ for all $i$,
then this becomes 
a deterministic OR-gate. Indeed,

\beq
P(y=1|x^n, \lam=0, \pi^n=1^n)= 1-\prod_i[1-x_i]=
\V_{i=0}^{n-1}x_i
\;,
\eeq
so 

\beq
P(y|x^n, \lam=0, \pi^n=1^n)=
\delta(y, \V_{i=0}^{n-1}x_i)
\;.
\eeq


Note that if $\lam=0$ and $x^n$ is one hot (i.e., 
$x^n=e^n_i$, where $e^n_i$
is the vector with all components 
zero except for the $i$-th
component which equals 1), then

\beq
P(y=1|x^n=e^n_i, \lam=0, \pi^n)=
1-[1-\pi_i]=\pi_i
\;.
\eeq
This gives a meaning to the
parameters $\pi_i$.

Another way to
give a meaning to the 
parameters $\pi_i$
is to associate each of 
them with an
empirical probability 
distribution associated 
with a vector of samples
$\vec{\rva}_i$.
Suppose
$\vec{\rva}_i=(\rva_i[s])_
{s=0, 1, \ldots, nsam-1}$ 
and  the samples $\rva_i[s]$
 are i.i.d. with
$\rva_i[s]\sim P_{\rva_i|\rvx_i}$
for all $s$.
Define

\beq
P_{\rva_i|\rvx_i}(1|x_i)=1-\pi_i x_i
\;.
\eeq
Note that for each $i$,
$P_{\rva_i|\rvx_i}(1|x_i)$
can be 
obtained
from the vector of samples
$\vec{\rva}_i$ 
as follows:


\beq
P_{\rva_i|\rvx_i}(1|x_i)=
\frac{1}{nsam}
\sum_{s=0}^{nsam-1} \indi(a_i[s]=1)
\;.
\eeq
The same noisy-or gate $\rvy$
that we first defined with
Fig.\ref{fig-noisy-or-1}
can also be defined using
Fig.\ref{fig-noisy-or-2}.
The TPM, printed in blue,
for  
the nodes $\rvy$
and $(\vec{\rva}_i)_{i=0,1,\ldots, n-1}$
of Fig.\ref{fig-noisy-or-2}, are  as follows:
From Eq.(\ref{eq-noisy-or-tmp-1}),
we get:

\beq\color{blue}
P(y=1|x^n, \lam, \veca^n)=
1-(1-\lam)\prod_i P_{\rva_i|\rvx_i}(1|x_i)
\;.
\eeq
Furthermore, 

\beq\color{blue}
P(\veca_i|x_i)=
\prod_s P_{\rva_i|\rvx_i}(a_i[s]\cond x_i)
\;.
\eeq

\begin{figure}[h!]
$$\xymatrix{
\rvx_0\ar[dr]\ar@/_1pc/[ddrr]
&&\rvx_1\ar[d]\ar@/_1pc/[dd]
&&\rvx_2\ar@/_1pc/[ddll]\ar[dl]
\\
&\vec{\rva}_0\ar[dr]
&\vec{\rva}_1\ar[d]
&\vec{\rva}_2\ar[dl]
\\
&\ul{\lam}\ar[r]
&\rvy
}$$
\caption{ Fig.\ref{fig-noisy-or-1}
after replacing parameters 
$(\ul{\pi}_i)_{i=0,1,2}$
by 
vectors 
of samples $(\vec{\rva}_i)_{i=0,1,2}$.}
\label{fig-noisy-or-2}
\end{figure}